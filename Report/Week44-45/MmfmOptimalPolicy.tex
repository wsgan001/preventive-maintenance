\documentclass[a4paper]{thesis}
\usepackage[english]{babel}
\usepackage[utf8x]{inputenc}
\usepackage{amsfonts}
\usepackage{amsmath} % Equation numbering
\usepackage[square,sort,comma,numbers]{natbib}
\usepackage[toc,page]{appendix} % For appendices

\usepackage{graphicx} % Figures
\usepackage{grffile} % For recognising the png-extension
\usepackage{float} % For floating figures

\usepackage{changes} % For track changed extensions
\usepackage{todonotes}

\usepackage{dsfont} % For indicator function

\usepackage{hyperref}
\bibliographystyle{plainnat}

% For example environment
\usepackage{amsthm}
\theoremstyle{definition}
\let\example\relax
\newtheorem{example}{Example}[chapter]
\newtheorem{remark}{Remark}[chapter]
\newtheorem{corollary}{Corollary}[chapter]

% For theorem environment
\newtheorem{theorem}{Theorem}[section]

% For lemma environment
\newtheorem{lemma}{Lemma}

\begin{document}
\chapter{MMFM Optimal Policy}
In this section, the structure of the optimal policy for a machine which deteriorates according to the Markov Modulated Fluid Model will be determined.
We will start with the simple model from earlier and will incrementally add complexity.

\section{One CTMC-state, no jumps}
The initial fluid level is given by $Q_0\sim F(q)$ and the fluid rate in this single state equals $r$.
The hazard rate then equals
\begin{equation}
\begin{split}
\lim\limits_{\delta\rightarrow0}\frac{1}{\delta}\mathbb{P}(rx<Q_0<r(x+\delta)|rx<Q_0)&=\lim\limits_{\delta\rightarrow0}\frac{\mathbb{P}(rx<Q_0<r(x+\delta))}{\delta\mathbb{P}(rx<Q_0)}\\
&=rh(rx).
\end{split}
\end{equation}
The rest of the problem can be solved in the same way as the simple problem.
Hence, the optimal control limit $\mu^*$ is the first $\mu$ such that
\begin{equation}
rh(r\mu)=\beta\frac{c+V_\mu}{a}
\end{equation}
(where $V_\mu$ is the expected total discounted cost when using control limit $\mu$), or $\mu^*=\infty$ else.
\section{One CTMC-state, constant jumps at exponentially distributed time intervals}
We extend the previous model by adding constant fluid jumps to the model at exponentially distributed intervals.
The time intervals between jumps have rate $\lambda$ and fluid quantity $J$.
As earlier, we denote the state of the machine at time $t$ by
$$
X(t)=(S(t),L_0(t),L_c(t)),
$$
but in this simple one-state problem $S(t)=s_0$ for all $t$.
The current state is completely observable, so at time $t$, $L_0(t)$ and $L_c(t)$ are known.
The hazard rate at time $t$ is given by
\begin{equation}
h(t)=\begin{cases}
0&\text{if }L_c(t)>0\\
rh(rL_0(t))&\text{else.}
\end{cases}
\end{equation} 
This suggests that preventive maintenance should be chosen at time $t$ when $L_c(t)=0$ and
\begin{equation}
rh(rL_0(t))=\beta\frac{c+V_{L_0(t)}}{a},
\end{equation}
where $V_{l}$ equals the expected total discounted cost when the machine is repaired whenever $L_0(t)$, the observed lower bound of $Q_0$, equals $l$.
Note that when there are no jumps ($J=0$), this corresponds to the optimal policy (and cost) of the previous problem.
Hence, we will refer to this limit for the lower bound as the control limit $\mu$ and use the notation $V_\mu$ for the corresponding cost.

In the simple problem, we could easily calculate at each time $t$, how long it takes until preventive repair should take place ($\mu-t$, assuming $\mu<Q_0$).
But in this case, the presence of fluid jumps complicates this.
A lower bound for the time at which preventive maintenance should be scheduled is given by $L_c(t)+\mu-L_0(t)$ and this is also the optimal time to do preventive maintenance if no jumps occur in the meantime.

\subsection{Calculation of the expected total discounted cost}
For a given $\mu$, $V_\mu$ is still difficult to compute.
The presence of jumps complicates the calculation of the time between repairs.
Let $T_t(q)$ be the random variable denoting the time until the fluid level is $q$ lower than it was at time $t$, i.e.
$$
T_t(q)=\min\{\tau\geq0|Q(t+\tau)\leq Q(t)-q\}.
$$
Note that, using this definition, $T_0(\mu)$ equals the time until the control limit is reached ($L_0(t)=\mu$) and $T(Q_0)$ equals the time until the machine fails.
\begin{lemma}
$Q_0\leq\mu\Leftrightarrow T_t(Q_0)\leq T_t(\mu)$
\end{lemma}
\begin{proof}
$\Rightarrow$: 
\begin{equation}
\begin{split}
Q_0\leq\mu&\Rightarrow Q(t)-\mu\leq Q(t)-Q_0\\
&\Rightarrow (Q(t+\tau)\leq Q(t)-\mu\Rightarrow Q(t+\tau)\leq Q(t)-Q_0)\\
&\Rightarrow T_t(Q_0)\leq T_t(\mu)\\
\end{split}
\end{equation}
$\Leftarrow$: We will prove that $Q_0>\mu\Rightarrow T_t(Q_0)> T_t(\mu)$:\\
We know that
$$
Q_0>\mu\Rightarrow Q(t)-\mu > Q(t)-Q_0.
$$
Since $Q(t)$ is piecewise continuous and does not decrease at the discontinuities, we know that 
$$
Q(t+T_t(\mu))=Q(t)-\mu>Q(t)-Q_0\Rightarrow T_t(Q_0)> T_t(\mu).
$$
\end{proof}
Also, let $N_t(q)$ be the random variable denoting the number of jumps that occur in the interval $(t,t+T_t(q)]$.
We will now compute its distribution.
For simplicity, we will assume that $r=1$.
The probability that zero jumps occur equals the probability that the exponentially distributed time interval is larger than $q$:
$$
\mathbb{P}(N_t(q)=0)=e^{-\lambda q}.
$$
The probability that exactly one jump occurs equals the probability that exactly one Poisson event happens in the interval $(t,t+q]$ while none happen in $(t+q,t+q+J]$. Resulting in
$$
\mathbb{P}(N_t(q)=1)=\lambda q e^{-\lambda q} e^{-\lambda J}=\lambda q e^{-\lambda (q+J)}.
$$

For each $k\geq0$, by conditioning on the time until the first jump, we get the following recursion
\begin{equation}\label{eq:recursionFormula}
\begin{split}
\mathbb{P}(N_t(q)=k+1)&=\int\limits_0^{q}\lambda e^{-\lambda x}\mathbb{P}(N_t(q-x+J)=k)dx\\
&=\int\limits_0^{q}\lambda e^{-\lambda (q-y)}\mathbb{P}(N_t(y+J)=k)dy,
\end{split}
\end{equation}
since after this first jump, the fluid level equals $q-x+J$ and $k$ jumps should occur. The second equality follows after the substition $y=q-x$.
The solution of this recursion, is given by
$$
\mathbb{P}(N_t(q)=k)=\lambda q\frac{(\lambda (q+kJ))^{k-1}}{k!}e^{-\lambda(q+kJ)}.
$$
Which can be seen by substituting it into \eqref{eq:recursionFormula} and setting $k=1$ to see that it also satisfies the calculated expression for $\mathbb{P}(N_t(q)=1)$.

Now we know the distribution of the amount of jumps that occur in a run where the initial fluid level equals $q$, we can derive the distribution of the time length of a run with initial fluid level $Q_0$ random with density function $f_{Q_0}(q)$.
This distribution $f(x)$, is given by
\begin{equation}
\begin{split}
f(x)&=\lim\limits_{\delta\rightarrow 0}\frac{1}{\delta}\mathbb{P}(x<T_0(Q_0)<x+\delta)\\
&=\sum\limits_{k=0}^{\lfloor\frac{x}{J}\rfloor}\mathbb{P}(N_0(q-kJ)=k)\lim\limits_{\delta\rightarrow 0}\frac{1}{\delta}\mathbb{P}(x-kJ<Q_0<x-kJ+\delta)\\
&=\sum\limits_{k=0}^{\lfloor\frac{x}{J}\rfloor}\mathbb{P}(N_0(q-kJ)=k)f_{Q_0}(x-kJ)\\
\end{split}
\end{equation}

The total discounted cost $V_\mu$ when using control limit $\mu$ can, similarly as with the simple problem, be calculated in the following way
\begin{equation}
V_\mu=\frac{(c+a)\mathbb{P}(Q_0<\mu)\mathbb{E}[e^{-\beta Q_0}|Q_0<\mu]+c\mathbb{P}(Q_0\geq \mu)e^{-\beta\mu}}{1-\mathbb{E}e^{-\beta T(\mu\wedge Q_0)}}.
\end{equation}
The same iteration approaches can also be applied.

\section{Multiple CTMC-states with equal fluid rates}
TODO
\section{Multiple CTMC-states with different fluid rates}
TODO
\end{document}