\documentclass[a4paper]{report}
\usepackage[english]{babel}
\usepackage{amsfonts}
\usepackage{amsmath} % Equation numbering
\usepackage[square,sort,comma,numbers]{natbib}

\usepackage{graphicx} % Figures
\usepackage{grffile} % For recognising the png-extension
\usepackage{float} % For floating figures

\usepackage{changes} % For track changed extensions
\usepackage{todonotes}

\usepackage{dsfont} % For indicator function

\usepackage{hyperref}
\bibliographystyle{plainnat}

% For example environment
\usepackage{amsthm}
\theoremstyle{definition}
\newtheorem{example}{Example}[section]

% For theorem environment
\newtheorem{theorem}{Theorem}

\begin{document}
\chapter{Maximum Likelihood estimators for MMFM fluid rates and constant jump quantities}
In these notes, a maximum likelihood estimator will be presented to estimate the fluid rates and jump quantities of a Markov Modulated fluid model using the lifetimes.
We assume that all transition rates $\lambda_{ij}$ and the initial distribution $f(q)$ are already known.

Suppose we have observed a run of the machine and have seen that it started in state $s_{i_1}$, stayed there for a period time of length $\tau_1$.
Suppose also that after this time, a transition occurred to $s_{i_2}$ and the machine stayed there for a time $\tau_2$ and so forth.
Hence we have observations in the following form
$$
\sigma=\left[[i_1,\tau_1],...,[i_L,\tau_L] \right].
$$
We assume that no preventive maintenance is done so that after the last observation in the trace, it fails.

For a given MMFM model $M$ with rates $r_i$ and jump quantities $J_{ij}$, this would mean that initially the fluid level was
$$
q_0(M,\sigma)=\tau_1r_{i_1}+\sum\limits_{k=2}^{L}\tau_kr_{i_k}-J_{i_{k-1}i_k}
$$
Such that the likelihood of this trace would be
$$
L(\sigma)=f(q_0(M,\sigma))\left[\prod\limits_{k=1}^{L-1}\lambda_{i_ki_{k+1}}e^{-\lambda_{i_k}\tau_k}\right]e^{-\lambda_{i_L}\tau_L}
$$
using $\lambda_i=\sum_j\lambda_{ij}$.

To determine the ratein $s_i$, we can derive this to $r_i$.
For simplicity, we take the log likelihood.
$$
\frac{d}{dr_i}\log L(\sigma)
=\frac{d}{dr_i}\left(\log f(q_0(M,\sigma))
+\left[\sum\limits_{k=1}^{L-1}\log(\lambda_{i_ki_{k+1}}e^{-\lambda_{i_k}\tau_k}\right]
+\log(e^{-\lambda_{i_L}\tau_L})\right).
$$
Note that only $\log f(q_0(M,\sigma))$ depends on $r_i$ such that all other terms vanish and we get
$$
\frac{d}{dr_i}\log L(\sigma)=\frac{d}{dr_i}\log f(q_0(M,\sigma))=\frac{f'(q_0(M,\sigma))}{f(q_0(M,\sigma))}\tau(i,\sigma).
$$
Where 
$$
\tau(i,\sigma)=\sum\limits_{k|i_k=i}\tau_k.
$$
Similarly, for the jump quantities $J_{ij}$, we'll get
$$
\frac{d}{dJ_{ij}}\log L(\sigma)=\frac{d}{dr_i}\log f(q_0(M,\sigma))=\frac{f'(q_0(M,\sigma))}{f(q_0(M,\sigma))}\#(i,\sigma)
$$
where $\#(i,\sigma)$ is the number of times a jump to $s_i$ was observed in $\sigma$.

For multiple measurements  $\sigma_1,...,\sigma_K$, where
$$
\sigma_n=\left[[i_1^{(n)},\tau_1^{(n)}],...,[i_{L^{(n)}}^{(n)},\tau_{L^{(n)}}^{(n)}] \right],
$$
the MLE will be the solution of the following equations
$$
\frac{d}{dr_i}\log L(\sigma_1,...,\sigma_K)=\sum\limits_{n=1}^K\frac{f'(q_0(M,\sigma_n))}{f(q_0(M,\sigma_n))}\tau(i,\sigma_n)=0
$$
for all $i$ and 
$$
\frac{d}{dJ_{ij}}\log L(\sigma_1,...,\sigma_K)=\sum\limits_{n=1}^K\frac{f'(q_0(M,\sigma_n))}{f(q_0(M,\sigma_n))}\#(i,\sigma_n)=0
$$
for all $J_{ij}$.
\end{document}