\section{Simplified models}
Instead of tackling the complete problem at once, we start with a simplified version of the problem, solve this and then make the model incrementally more realistic.\\
\\
\subsection{First model}
In our first and simplest model, the machine has a lifetime that follows some distribution $f(l)$ (cumulative function $F(l)$). When the machine breaks, it needs to be repaired and a certain cost for corrective maintenance needs to be paid ($c_c$). We start with an open loop problem where we have to choose at the beginning at what time the machine will be repaired. 
When the machine is repaired, the problem starts again with a new lifetime according to the same distribution.\\
We want to minimize the average cost. Since the machine is renewed after each repair, we can do this by minimizing the expected average cost until the first repair.

If we decide to repair the machine at some time $u>0$, then the expected cost will be
$$
J(u)=\frac{(1-F(u))c_p}{u}+\int\limits_{0}^u\frac{f(l)c_c}{l}dl
$$
\added{You also want to check the value at $u=0$ and $u=\infty$. Derivative $=0$ is condition for extreme value only inside the domain where the function is analytic. And it says nothing about minima or maxima. As you see it on your hazard plot figure, there are going to be minima and mixima in there, so checking if this is an extreme value is not always enough.}
We want to minimize this, so we search for a zero of its derivative
$$
\frac{d}{du}J(x)=-\frac{c_p}{u^2} - \frac{f(u)c_p}{u} +\frac{F(u)c_p}{u^2}+\frac{f(u)c_c}{u}=0
$$
We multiply by $u^2$ ($u>0$ since preventive maintenance at time 0 would result in infinite average cost)
$$
c_pF(u)-c_p+uf(u)(c_c-c_p)=0
$$
Which can be solved either numerically or algebraic for some specific distribution. For example, for lifetimes uniformly distributed on the interval $[0,L]$, this would result in solving
$$
c_p\frac{u}{L}-c_p+\frac{u(c_c-c_p)}{L}=0\Rightarrow u=L\frac{c_p}{c_c}
$$

\subsection{Closed loop}
We now make the problem a little more difficult by discretizing the time and having a limited set of options at each stage.
There are two states, one where the machine is broken ($s_b$), and one where it is not ($s_0$). If the machine is broken, the only available action is $u_c$ (corrective maintenance) with cost $c_c$. After $u_c$, the machine is renewed. When the machine is not broken, there are two actions:
\begin{itemize}
\item Preventive maintenance ($u_p$) with cost $c_p<c_c$, renewing the machine.
\item Do nothing ($u_w$) with cost 0.
\end{itemize}
For simplicity we introduce a discount $\alpha$ and are interested in the discounted cost instead of the average cost. The value function would then be 
$$
J_{k}(s_0)=\min\{c_p+\alpha J_0(s_0),\alpha p_kJ_{k+1}(s_0)+(1-p_k)J_{k+1}(s_b)\}
$$
$$
J_{k}(s_b)=c_c+\alpha J_0(s_0)
$$
Where $p_k$ denotes the probability that the machine does not break until the next stage. If we discretize time as $t_k=k\Delta$, this probability would be $\mathbf{P}(L>t_{k+1}|L>t_k)=\frac{1-F(t_{k+1})}{1-F(t_k)}$.
\added{Although you say that this is model with two states, it is not really. The machine spends exactly zero time in $s_b$, so that state can be eliminated. So effectively this is one state.
$$
J_{k}(s_0)=\min\{c_p+ J_0(s_0),p_k J_{k+1}(s_0)+(1-p_k) (c_c+ J_0(s_0))\}
$$}

\subsection{Fluid models}
The last problem could be seen as a fluid model with one state with rate $-1$ and a random initial fluid level (ignoring the broken state for now). We could extend this to fluid models of more states.
We introduce a fluid model with two states: 
\begin{itemize}
\item $s_0$: With fluid rate $r_0<0$
\item $s_1$: With fluid level $r_1<0$.
\end{itemize}
\added{Now this is proper two states, because the third $s_b$ is not listed :)}
The system transitions in the fluid model occur as a CTMC when $u_w$ is chosen and the machine does not break, when $u_p$ is chosen, the system is renewed and transitions to $s_0$. When the machine breaks, it transitions to $s_b$ from which the only available action is $u_c$ which transitions to $s_0$.\\

To calculate the probability that the machine breaks between two stages, we need to have the distribution of the fluid decrease in a period of length $\Delta$. Let $\overline{r}=\max{r_0,r_1}$ and $\underline{r}=\min{r_0,r_1}$. Let $\Delta\underline{r}<q<\Delta\overline{r}$, then the probability that the fluid decreases less than $q$ in the next period, equals the probability that the machine spends less than $\frac{q-\underline{r}\Delta}{\overline{r}-\underline{r}}$ time in the state with the lowest rate.
Let $f_{i}^j(t^*,t)$ denote the density of spending $t^*$ out of $t$ time in state $j$ given that it starts in state $i$. We have that for small $h$:
$$
f_{0}^1(t^*,t)=\lambda_0hf_1^1(t^*,t-h)+(1-\lambda_0h)f_{0}^1(t^*,t-h)
$$
$$
\Rightarrow \frac{f_{0}^1(t^*,t)-f_{0}^1(t^*,t-h)}{h}=\lambda_0(f_1^1(t^*,t-h)-f_{0}^1(t^*,t-h))
$$
And in the limit this results in
$$
\frac{d}{dt}f_{0}^1(t^*,t)=\lambda_0(f_1^1(t^*,t)-f_{0}^1(t^*,t))
$$
Similarly, for $f_1^1(t^*,t)$ we have
$$
(\frac{d}{dt}+\frac{d}{dt^*})f_1^1(t^*,t)=\lambda_1(f_0^1(t^*,t)-f_1^1(t^*,t))
$$
The first equation is an ordinary differential equation and can be solved using the method of the integrating factor, this leads to
$$
f_0^1(t^*,t)=\int\lambda_0f_1^1(t^*,t)e^{\lambda_0t}dte^{-\lambda_0t}
$$
(constants omitted for readability). The second equation is a partial differential equation and could be approached using the method of characteristics, which leads to
$$
f_1^1(t_0,t_0+x)=\int\limits^{x}\lambda_1f_0^1(t_0,t_0+y)e^{\lambda_1y}dyd^{-\lambda_1x}
$$
(Lower limit of the integral must be set such that $f_1^1(t_0,t_0)=e^{-\lambda_1t_0}$)
I currently have not found a way to finish these differential equations.

\added{This is the brute force way of getting the distribution that we want and probably not in the simplest form. I suggest the following approach. Assume you start from a state and you want to get the distribution of the proportion of time spent in that state before $t$. Consider the following: The time spent in state $i$ is always the part of the trajectory that starts in $i$ and you spend there an exponential time. The distribution can be calculated based on what is the next state after $i$ and what is the distribution of time until return to $i$. This return distribution is exactly a phase-type distribution with properly chosen parameters. So you can work this out without solving those nasty integrals. After this is done, you can probably use law of total provability on the number of excursions from $i$ to $i$. My guess is that this approach will give you something that can be treated a bit better. Of you get the solution from this approach, even in form of infinite sums and such, you can check if that satisfies the d.e., as a sanity check.}

\added{There is a reason why this approach works only for this simplified setup with two states. And why this cannot be extended for more states. Try to think about it.}