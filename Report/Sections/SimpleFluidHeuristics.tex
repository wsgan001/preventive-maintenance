\section{Heuristic policies}
Instead of directly trying to find the optimal policy for the proposed simple fluid model, we will first try a heuristic policy.
If we compare the current problem with the age-based maintenance problem and look at the differences concerning the control limit that the jumps introduce, we see the following two differences:
\begin{enumerate}
	\item The presence of jumps increases the time until the control limit is reached, which decreases the discount factors, which decreases the expected total discounted cost.
	\item The presence of jumps changes the structure of the control limit slightly.
	The fact that a jump can occur between now and the breakdown of the asset, increases the control limit.
\end{enumerate}
The first difference can be anticipated using the new formula for the expected total discounted cost \eqref{eq:SimpleFluidExplicitTDC}.
The second difference could be dealt with by simply assuming no jump will occur before the next failure.
In this way, the control limit will be given by the same equation as for the simple discounted problem \eqref{eq:AgeBasedHazardBound}, but with the new total discounted cost $V(x_{NEW},\mu^*)$ on the right hand side.
This results in a heuristic policy for which the same iteration methods can be used with the same convergence properties as in the previous chapter.
Appendix \ref{AppendixComputationsTable} contains total discounted costs and control limits using this heuristic policy for various problem parameters.