\section{The optimal policy}
In this section, we will analytically derive the control limits for the optimal MMFM maintenance policy.
This will be done using the Bellman equations.

If the optimal control limit in state $s_i$ is given by $\mu_i^*$, then in the state $x=(s_i,\mu_i^*,0)$ where repair is chosen, it holds that the expected cost of waiting one more time step of size $\delta$ is at least as large as the expected cost of repairing.
The repair cost equals $c+V(x_{NEW},\pi^*)$ and by \eqref{eq:MmfmNextState}, the cost of waiting equals
\[
\begin{split}
&(1-\lambda_i \delta-\delta r_ih(\mu_i^*))V_\delta(i,\mu_i^*+r_i\delta,0,\pi^*)\\
&+ \delta r_ih(\mu_i^*)V_\delta(x_{BREAK},\pi^*)\\
&+\sum\limits_{j\neq i}\lambda_{ij} \delta V_\delta(j,\mu_i^*,J_{ij}-r_j\delta,\pi^*)+o(\delta^2).
\end{split}
\]
So we know that
\[
\begin{split}
c+V(x_{NEW},\pi^*)\leq 
&(1-\lambda_i \delta-\delta r_ih(\mu_i^*))V_\delta(i,\mu_i^*+r_i\delta,0,\pi^*)\\
&+ \delta r_ih(\mu_i^*)V_\delta(x_{BREAK},\pi^*)\\
&+\sum\limits_{j\neq i}\lambda_{ij} \delta V_\delta(j,\mu_i^*,J_{ij}-r_j\delta,\pi^*)+o(\delta^2).
\end{split}
\]
Also, $V_\delta(x_{BREAK},\pi^*)=c+a+V_\delta(x_{NEW},\pi^*)$ and $V_\delta(i,\mu_i^*+r_i\delta,0,\pi^*)=c+V_\delta(x_{NEW},\pi^*)$ as repair is chosen next.
Substituting this, we get
\[\begin{split}
c+V(x_{NEW},\pi^*)\leq
&(1-\lambda_i \delta-\delta r_ih(\mu_i^*))(c+V(x_{NEW},\pi^*))\\
&+ \delta r_ih(\mu_i^*)(c+a+V(x_{NEW},\pi^*))\\
&+\sum\limits_{j\neq i}\lambda_{ij} \delta V_\delta(j,\mu_i^*,J_{ij}-r_j\delta,\pi^*)+o(\delta^2).
\end{split}
\]
Subtracting $c+V(x_{NEW},\pi^*)$ from both sides, dividing by $\delta$ and rewriting yields
\[
r_ih(\mu_i^*)a+\sum\limits_{j\neq i}\lambda_{ij} V_\delta(j,\mu_i^*,J_{ij}-r_j\delta,\pi^*)\geq
(\beta+\lambda_i)(c+V(x_{NEW},\pi^*))+o(\delta).
\]
If we were to do the same but starting at a state $x'=(s_i,\mu_i^*-r_i\delta,0)$ (i.e. just before the control limit is reached) so we know that the cost of waiting is smaller than the cost of preventive maintenance, we would get
\[
r_ih(\mu_i^*)a+\sum\limits_{j\neq i}\lambda_{ij} V_\delta(j,\mu_i^*,J_{ij}-r_j\delta,\pi^*)<
(\beta+\lambda_i)(c+V(x_{NEW},\pi^*))+o(\delta).
\]
Which together proves the following theorem:
\begin{theorem}
	If for the optimal control limit policy $\pi^*=[\mu_1^*,...,\mu_N^*]$ the control limit in CTMC-state $s_i$ is finite (i.e. $\mu_i^*<\infty$), then the following equation holds 
	
	\begin{equation}\label{eq:MmfmHazardBounds}
	r_ih(\mu_i^*)a+\sum\limits_{j\neq i}\lambda_{ij} V(j,\mu_i^*,J_{ij},\pi^*)=
	(\beta+\lambda_i)(c+V(x_{NEW},\pi^*)).
	\end{equation}
\end{theorem}
\begin{remark}
	We can rewrite the expected costs $V(j,\mu_i^*,J_{ij},\pi^*)$ in \eqref{eq:MmfmHazardBounds} to
	\[
	V(j,\mu_i^*,J_{ij},\pi^*)=\sum\limits_k D_{jk}(J_{ij},\pi,\mu_i^*)V(j,\mu_i^*,0,\pi^*),
	\]
	using the discounted probability $D_{jk}(J_{ij},\pi,\mu_i^*)$ that the process will be in CTMC-state $s_k$ when the buffer $L_c$ is emptied.
\end{remark}

\begin{remark}
	It is difficult to compute the control limits $\mu_i^*$ analytically since the total discounted costs depend on the control limit and the control limit depends on the total discounted costs.
\end{remark}

\begin{remark}
Note that using the generator matrix $\Lambda^D$ defined by \eqref{eq:MmfmDiscountGenerator}, equation for the optimal policy \eqref{eq:MmfmHazardBounds} could also be written as

\begin{equation}\label{eq:MmfmHazardBoundsShort}
h(\mu_i^*)a+\sum\limits_j\Lambda^D_{ij}V(j,\mu_i^*,0,\pi^*)=0.
\end{equation}
\end{remark}

\begin{remark}
Note that equation \eqref{eq:MmfmHazardBounds} is similar to the equation for the optimal control limit of the previous problems in the sense that it has a constant right-hand side and an increasing left hand side.
\end{remark}