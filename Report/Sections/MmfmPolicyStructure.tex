\section{Structure of optimal policy}
In this section, we will establish that for the MMFM preventive maintenance problem, the optimal policy is a stationary policy to repair in CTMC-state $s_i$ whenever the buffer $L_c$ is empty and the used initial fluid $L_0$ exceeds a certain control limit $\mu_i^*$.
Note that 'stationary' in this sense means independent of the time. 
The control limit does depend on the state of the CTMC.

\subsection{Stationary policy}
By the same reasoning as for the previous chapter, we can prove that every discrete approximation of the continuous-time MDP has a countable state space so that section 6.2.4 of \cite{Puterman2008} again proves that the there exists a stationary policy that is discount-optimal.
%By the same reasoning as for the previous problems, we can prove that in each state $x\in X$, the optimal choice depends only on the distribution $F_x$ of the remaining fluid, which only depends on the state $x$ of the process.

\subsection{Empty buffer}
By the same reasoning as for the simple fluid problem, an optimal policy will never perform preventive maintenance if the buffer $L_c$ is non-empty.

\subsection{Control limit}
Compared to the previous problem, proving that the optimal policy is a control limit policy is slightly more difficult:
In the simple fluid problem, it was never possible to reach states 'after the control limit' so that it wasn't relevant whether for higher fluid levels repair should also always be chosen.
For this problem, it is possible that these states are reached.
This is illustrated by the following example:
\begin{example}
	Consider a MMFM with two states and control limits $\mu_1<l<\mu_2$ for some $q$.
	If at some time $t$, $X(t)=(2,l,L_c(t))$, then it is possible that for some $t'>t$, $X(t')=(1,l,0)$ so that $L_0(t')>\mu_1$ and the control limit is exceeded.
\end{example}
Although no rigorous proof was found that asserts the optimal policy is a control limit on $L_0$, results from value iteration seem to suggest that this is always the case.
This also seems natural as for states with used initial fluid $L_0$ higher than the control limit, the asset is in a worse condition than at the control limit and is more likely to fail earlier so that repair would still be chosen after the control limit.

Hence, we will consider stationary policies of the form $\pi=[\mu_1,...,\mu_N]$, where preventive repair is chosen in states $x=(i,l_0,l_c)\in X$ if and only if $l_c=0$ and $l_0\geq \mu_i$.