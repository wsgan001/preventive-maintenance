\section{Structural properties}\label{section:AgeBasedStructuralProperties}
In this section, the effect of changing problem parameters is investigated.

\subsection{Effect on control limit}
From \eqref{eq:AgeBasedHazardBound}, we can see what effects changing $h,\beta,c$ and $a$ has on the control limit:
\[h(\hat{\mu})=\beta\frac{c+V(0^+,\mu^*)}{a}.\]
\begin{remark}
	Decreasing  $a$ would result in an increase of the right hand side (the hazard bound) of this equation, increasing the control limit.
	This is expected, as this would decrease the incentive to prevent the machine from failing.
\end{remark}
\begin{remark}
	Decreasing $c$ would result in a decrease of the hazard bound, decreasing the control limit.
	This is also expected as this would make $a$ relatively larger.
\end{remark}
\begin{remark}
	If we would multiply $c$ and $a$ by a constant, then the control limit would not change as the sizes of these costs relative to each other remains the same.
	This can be proven using remark \ref{remark:AgeBasedCostRatio}.
\end{remark}
\begin{remark}\label{remark:AgeBasedControlLimitDiscountIncrease}
	Increasing $\beta$ will result in an increase in hazard bound, resulting in an increase in the control limit.
	This can be explained by the fact that with a higher discount, costs further in the future weigh less such that you would want to move costs as far into the future as possible.
\end{remark}
\begin{remark}
	A higher hazard (or a faster increasing hazard) results in a lower control limit as expected.
\end{remark}
\begin{remark}\label{remark:AgeBasedControlLimitDiscountAndHazardIncrease}
	If the hazard rate and the control limit are both multiplied by a constant, then the hazard rate remains the same as this constant can just be divided out of this equation.
\end{remark}

\subsection{Effect on expected total discounted cost}
From \eqref{eq:AgeBasedOptimalTDC}, we can see what effects changing $h,\beta,c$ and $a$ has on the expected total discounted cost:
\[V(0^+,\mu)=\frac{aF(\mu)\mathbb{E}[e^{-\beta Q_0}|Q_0\leq \mu]+c\mathbb{E}[e^{-\beta(Q_0\wedge\mu)}]}{1-\mathbb{E}[e^{-\beta (Q_0\wedge\mu)}]}.\]
\begin{remark}
	Decreasing $c$ or $a$ results in a decrease of this total cost as expected.
\end{remark}
\begin{remark}\label{remark:AgeBasedCostRatio}
	If we multiply both $c$ and $a$ by a constant, the cost will also be multiplied by this constant.
	This is natural as expressing the valuta the costs are expressed in should not be relevant.
\end{remark}
\begin{remark}
	A higher hazard would result in a higher expected value for the discount, decreasing the denominator and increasing the total discounted cost.
	This is natural as a machine that fails quicker is more expensive to maintain.
\end{remark}
\begin{remark}\label{remark:AgeBasedTDCDiscountIncrease}
	Increasing the discount increases the denominator and the numerator so that the discounted cost decreases, as expected.
\end{remark}
For various parameters and distributions, the optimal control limit and total discounted cost are summarized in appendix \ref{AppendixComputationsTable}.