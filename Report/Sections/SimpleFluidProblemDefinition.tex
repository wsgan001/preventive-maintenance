\section{Problem formulation and definition}
In this section, we extend the definition of age-based maintenance from section \ref{section:AgeBasedDefinition}.
First we define the underlying stochastic process of the the asset degrading over time and instantaneously improve at the occurrence of jumps.
After that, we define a Markov decision process similarly to section \ref{section:AgeBasedDefinition}.

\subsection{Stochastic asset degradation}
We define the random process $Q(t)$ as the fluid level at age $t$.
Initially, the fluid level is given by $Q(0)=Q_0\sim F$.
Then over time this level decreases at a constant rate of $1$.
The jumps occur according to a Poisson process with rate $\lambda$, i.e. the time interval between two consecutive jumps is exponentially distributed with rate $\lambda$.
The fluid process is absorbing at $Q(t)=0$ as this resembles the asset failing.
Hence, the asset breaks at time $T^*$ given by
\[
T^*=\inf\{t|Q(t)=0\}.
\]
When the asset is repaired, it starts with an age of zero again, with an initial fluid level from the same distribution $F$.

To derive the distribution of $Q(t)$ for a certain $t$ from the observed jumps, it seems we need to keep track of the exact times at which these jumps occurred.
This would be a very inconvenient format of the state of the asset.
Luckily, this information can be condensed into a simpler state description.
First, we will illustrate this with the following example:
\begin{example}
	If the first jump of the process would occur at some time $t$ (and the fluid level did not reach $0$ already), then we would know the following at this time $t$:
	\begin{enumerate}
		\item $Q(t)\geq J$.
		Hence, we have a lower bound on the current fluid level.
		\item Initially, the fluid level was at least $t$, i.e. $Q_0\geq t$.
		Hence, we have a certain lower bound on the initial fluid level.
	\end{enumerate}
	If now, after some time $\tau<J$ another jump occurs, we know that:
	\begin{enumerate}
		\item At time $t+\tau$, $Q(t)\geq 2J-\tau$.
		The passage of time has hence decreased our lower bound of the current fluid level by $\tau$ and the jump has increased this bound by $J$.
		\item Our lower bound of the initial fluid level has remained unchanged.
		When our lower bound of the current fluid level is positive, our lower bound of the initial fluid level remains the same.
	\end{enumerate}
\end{example}
This suggests that the only two parameters we need to keep track of, are the lower bound of the current fluid level $L_c(t)$ and the lower bound of the initial fluid level $L_0(t)$.
We will refer to this $L_0$ as the drained initial fluid.
The asset cannot break if 
\[L_c(t)>0,\]
since we know with certainty that there is still remaining fluid.
We will refer to this quantity $L_c(t)$ as the fluid buffer.
The asset breaks whenever
\[L_0(t)=Q_0,\]
i.e. when the initial fluid level is drained.
Hence, we maintain the two quantities $L_c(t)$ and $L_0(t)$ as the description of the state the asset is in.
\[
X(t)=(L_0(t),L_c(t)).
\]
Initially
\[
X(0)=x_{NEW}:=(0,0).
\]
Furthermore, we also define a state $x_{BREAK}$ for when the asset is broken.
When $X(t)=x_{BREAK}$, $L_0(t)$ and $L_c(t)$ are undefined.
These two quantities evolve in the following way:
\begin{itemize}
	\item When a jump occurs at time $t$, $L_c$ increases by $J$, i.e.
	\begin{equation}\label{eq:SimpleJumpEvolution}
	X(t^+)=(L_0(t),L_c(t)+J).
	\end{equation}
%	\item When a time $\tau$ passes without a jump occurring while $\tau\leq L_c(t)$, $L_c$ decreases by $\tau$, i.e.
%	\begin{equation}\label{eq:SimpleAgeEvolution1}
%	X(t+\tau)=(L_0(t),L_c(t)-\tau).
%	\end{equation}
%	\item When at time $t$ $L_c(t)=0$ and a time $\tau$ passes without a jump occurring, $L_0$ increases by $\tau$, i.e.
%	\begin{equation}\label{eq:SimpleAgeEvolution2}
%	X(t+\tau)=(L_0(t)+\tau,0).
%	\end{equation}
	\item When no jump or failure occurs in a neighborhood of $t$, $S(t)$ remains constant and $L_0$ and $L_c$ evolve according to
		\begin{equation}\label{eq:SimpleAgeEvolution}
		\begin{split}
		\frac{d}{dt}L_c(t)&=\begin{cases}
		0&\ \text{if }L_c(t)=0,\\
		-1&\ \text{else.}
		\end{cases}\\
		\frac{d}{dt}L_0(t)&=\begin{cases}
		0&\ \text{if }L_c(t)>0,\\
		1&\ \text{else.}
		\end{cases}
		\end{split}
		\end{equation}
		Note that we always have
		\begin{equation}\label{eq:SimpleSameDecrease}
		\frac{d}{dt}\left[L_c(t)-L_0(t)\right]=\frac{d}{dt}Q(t)=-1.
		\end{equation}
	\item When the fluid drains, i.e. $Q(t)=0$ or equivalently $L_0(t^-)=Q_0$, the asset breaks so that
	\[
	X(t)=x_{BREAK},
	\]
	until it is being repaired.
\end{itemize}
\begin{theorem}\label{theorem:SimpleCurrentLevel}
	Using the $L_c(t)$ and the $L_0(t)$ as defined above.
	For $x\neq x_{BREAK}$, the fluid level is given by
	\begin{equation}\label{eq:SimpleCurrentLevel}
	Q(t)=L_c(t)+Q_0-L_0(t).
	\end{equation}
	\begin{proof}
		At the start of the process, $X(0)=(0,0)$ so that
		\[
		Q(0)=0+Q_0-0=Q_0.
		\]
		When a jump occurs, we know that $Q(t)$ increases by $J$.
		By \eqref{eq:SimpleJumpEvolution}, $L_c$ also increases by $J$ such that jumps preserve \eqref{eq:SimpleCurrentLevel}.
		When no jump occurs, we have by \eqref{eq:SimpleSameDecrease} that $Q(t)$ and $L_c(t)-L_0(t)$ decrease at the same speed.
		This completes the proof.
	\end{proof}
\end{theorem}

\begin{example}\label{example:SimpleFluidState}
	For example, if we have the jump quantity $J=3$, the initial fluid level for a run of the asset equals $5$ and after 1.5 and 2.5 time units a jump occurred, $Q,L_0$ and $L_c$ would evolve as in figure \ref{figure:SimpleFluidExampleQuantities}.
	As you can see, when $Q(t)=0$, it holds that $L_0(t)=Q(0)$.
\end{example}
\begin{figure}[H]
\centering
\setlength\fwidth{0.7\textwidth}
% This file was created by matlab2tikz.
%
%The latest updates can be retrieved from
%  http://www.mathworks.com/matlabcentral/fileexchange/22022-matlab2tikz-matlab2tikz
%where you can also make suggestions and rate matlab2tikz.
%
\definecolor{mycolor1}{rgb}{0.00000,0.44700,0.74100}%
\definecolor{mycolor2}{rgb}{0.85000,0.32500,0.09800}%
\definecolor{mycolor3}{rgb}{0.92900,0.69400,0.12500}%
%
\begin{tikzpicture}

\begin{axis}[%
width=0.951\fwidth,
height=0.75\fwidth,
at={(0\fwidth,0\fwidth)},
scale only axis,
xmin=0,
xmax=12,
xlabel style={font=\color{white!15!black}},
xlabel={Time},
ymin=0,
ymax=8,
ylabel style={font=\color{white!15!black}},
ylabel={Fluid},
axis background/.style={fill=white},
legend style={legend cell align=left, align=left, draw=white!15!black}
]
\addplot [color=mycolor1]
  table[row sep=crcr]{%
0	5\\
1.5	3.5\\
1.51	6.5\\
3.5	4.5\\
3.51	7.5\\
7.5	3.5\\
11	0\\
};
\addlegendentry{$Q$}

\addplot [color=mycolor2]
  table[row sep=crcr]{%
0	0\\
1.5	1.5\\
1.51	1.5\\
3.5	1.5\\
3.51	1.5\\
7.5	1.5\\
11	5\\
};
\addlegendentry{$L_0$}

\addplot [color=mycolor3]
  table[row sep=crcr]{%
0	0\\
1.5	0\\
1.51	3\\
3.5	1\\
3.51	4\\
7.5	0\\
11	0\\
};
\addlegendentry{$L_c$}

\end{axis}
\end{tikzpicture}%
\caption{$Q(t),L_0(t),L_c(t)$ for example \ref{example:SimpleFluidState}.}
\label{figure:SimpleFluidExampleQuantities}
\end{figure}

\begin{corollary}\label{corollary:SimpleFluidDistribution}
	Theorem \ref{theorem:SimpleCurrentLevel} implies that $Q(t)\sim F_{X(t)}$, where
	\begin{equation}\label{eq:SimpleCurrentDistribution}
	\begin{split}
	F_{X(t)}(q)&:=\mathbb{P}(Q(t)<q)\\
	&=\mathbb{P}(L_c(t)+Q_0-L_0(t)<q|Q_0>L_0(t))\\
	&=\mathbb{P}(Q_0<q+L_0(t)-L_c(t)|Q_0>L_0(t))\\
	&=\frac{F(q+L_0(t)-L_c(t))-F(L_0(t))}{\bar{F}(L_0(t))}.\\
	\end{split}
	\end{equation}
\end{corollary}
%\begin{remark}
%	Let $t^*$ denote the last time the machine was repaired.
%	Note that we can write $L_0(t)$ in the following way:
%	\[
%	L_0(t)=\max\limits_{t^*\leq\tau\leq t} Q_0-Q(\tau).
%	\]
%	As by \eqref{eq:SimpleCurrentLevel}, 
%	\[
%	L_0(t)-L_c(t)=Q_0-Q(t),
%	\]
%	and $L_0$ cannot decrease and only increases whenever $L_c=0$.
%\end{remark}
%\begin{corollary}
%	From the previous remark, it can also be seen easily that the asset will fail when the used initial fluid equals the initial fluid level as
%	\[
%	L_0(t)=Q_0\Rightarrow Q(t)=0.
%	\]
%\end{corollary}

We use the same discretization $t_k=\delta k$ as in the previous chapter.
For $x_k=X(t_k)$, we describe the random variables $\omega_k=\omega_k(x_k)$ of the Markov decision process in the following way
\[
\omega_k(x_k):=\begin{cases}
\Omega_{SURVIVE}&\ \begin{split}&\text{if the asset does not break and no jump occurs}\\
&\text{in the $k$'th time interval.}\end{split}\\
\Omega_{BREAK}&\ \text{if the asset breaks in the $k$'th time interval.}\\
\Omega_{JUMP}&\ \begin{split}&\text{if the asset does not break and a jump occurs}\\
&\text{in the $k$'th time interval.}\end{split}\\
\end{cases}
\]
Note that the asset can only break whenever $L_c(t)=0$.
Assuming only one jump can occur in a time interval and letting $x_k=X(t_k)=(l_0,l_c)$, we get the following probabilities:
\[
\begin{split}
\mathbb{P}(\omega_k(x_k)=\Omega_{SURVIVE})&=\begin{cases}
e^{-\lambda \delta}=1-\delta\lambda+o(\delta^2)&\text{ if }l_c>0,\\
\begin{split}
e^{-\lambda \delta} & \bar{F}_{x_k}(l_0+\delta)\\
=1&-\delta h(l_0)-\delta\lambda+o(\delta^2)
\end{split}&\text{ if }l_c=0.\\
\end{cases}\\
\mathbb{P}(\omega_k(x_k)=\Omega_{BREAK})&=\begin{cases}
0&\text{ if }l_c>0,\\
\begin{split}
e^{-\lambda \delta}F_{x_k}&(l_0+\delta)\\
=&\delta h(l_0)+o(\delta^2)
\end{split}&\text{ if }l_c=0.\\
\end{cases}\\
\mathbb{P}(\omega_k(x_k)=\Omega_{JUMP})&=\begin{cases}
1-e^{-\lambda \delta} & \text{ if }l_c>0,\\
\begin{split}
1-&e^{-\lambda \delta}\\
=&\delta\lambda+o(\delta^2)
\end{split}&\text{ if }l_c=0.\\
\end{cases}
\end{split}
\]
Where $F_x$ is given by corollary \ref{corollary:SimpleFluidDistribution}.

\subsection{Control actions}
We introduce a state $x_{BREAK}$ for when the asset is broken.
and in this state the only available action is $a_R$.
In all other states, both actions $a_W$ and $a_R$ may be chosen.
The definitions of these actions remains the same as in the definition of the age-based maintenance problem.

\subsection{State evolution}
Initially $x_{NEW}=(0,0)$.
The state of the Markov decision process now evolves in the following way:
\[
x_{k+1}=f(x_k, u_k, \omega_k):=\begin{cases}
x_{NEW}&\ \text{if }u_k=a_R,\\
\begin{split}(l_0+\delta-\min\{l_c,\delta\}&,\\l_c-\min\{l_c,\delta\}&)\end{split}&\ \text{if }u_k=a_W\text{ and }\omega_k=\Omega_{SURVIVE},\\
(l_0,l_c+J-\delta)&\ \text{if }u_k=a_W\text{ and }\omega_k=\Omega_{JUMP},\\
x_{BREAK}&\ \text{if }u_k=a_W\text{ and }\omega_k=\Omega_{BREAK}.\\
\end{cases}
\]
Here we assumed that jumps occur at the start of time intervals.
For small $\delta$, this approximates the fluid model defined above.
Again, we use the definition of the random variable $Z(x_k):=f(x_k,a_W,\omega_k(x_k))$ as the state after $x_k$.

\subsection{Costs and discounting}
The costs and discounting remain the same as in the age-based maintenance problem.

\subsection{Optimal policy and Bellman equations}
In the next section, we will prove that the optimal policy is a stationary policy.
Hence, we want to find a stationary policy $\mu:X\rightarrow \{a_W,a_R\}$ that chooses the action $u_k=\mu(x_k)$ that minimizes the expected total discounted cost $V_\delta(x_k,\mu)$ for each state.
Similarly as in the definition of age-based maintenance, $V_\delta(x_k,\mu)$ is given by
\[V_\delta(x_k,\mu)=g(x_k,\mu(x_k))+\alpha_\delta \mathbb{E}[V_\delta(Z(x_k),\mu)].\]
The Bellman equations for the optimal cost $V_\delta(x_k,\mu^*)$ read
\begin{equation}\label{eq:SimpleFluidBellman}
V_\delta(x_k,\mu^*)=\begin{cases}
c+a+\alpha_\delta V_\delta(x_{NEW},\mu^*),&\text{ if }x_k=x_{BREAK},\\
\min\left\{\begin{split}&c+\alpha_\delta V_\delta(x_{NEW},\mu^*),\\&\alpha_\delta \mathbb{E}[V_\delta(Z(x_k),\mu^*)]\end{split}\right\},&\text{else.}\\
\end{cases}
\end{equation}
$\mu$ is optimal if $V_\delta(x,\mu)=V_\delta(x,\mu^*)$ for all $x$.
$\mathbb{E}[V_\delta(Z(x_k),\mu^*)]$ when $x_k\neq x_{BREAK}$ is given by 
\begin{equation}\label{eq:SimpleFluidNextState}
\begin{split}
&\mathbb{E}[V_\delta(Z(l_0,l_c),\mu^*)]\\
&=\begin{cases}
\begin{split}
&(1-e^{-\lambda \delta})V_\delta(l_0,l_c+J-\delta,\mu^*)\\
&+e^{-\lambda \delta}V_\delta(l_0,l_c-\delta,\mu^*),
\end{split}&\ \text{If $l_c=0$,}\\
\begin{split}
&e^{-\lambda \delta} \bar{F}_{t_k}(l_0+\delta)V_\delta(l_0+\delta,0,\mu^*)\\
&+ e^{-\lambda \delta}F_{t_k}(l_0+\delta)V_\delta(x_{BREAK},\mu^*)\\
&+(1-e^{-\lambda \delta})V_\delta(l_0,J-\delta,\mu^*),
\end{split}&\ \text{If $l_c>0$.}\\
\end{cases}\\
&=\begin{cases}
\begin{split}
&\lambda\delta V_\delta(l_0,l_c+J-\delta,\mu^*)\\
&+(1-\lambda \delta)V_\delta(l_0,l_c-\delta,\mu^*)+o(\delta^2),
\end{split}
&\ \text{If $l_c=0$,}\\
\begin{split}
&(1-\lambda \delta-\delta h(l_0))V_\delta(l_0+\delta,0,\mu^*)\\
&+ \delta h(l_0)V_\delta(x_{BREAK},\mu^*)\\
&+\lambda \delta V_\delta(l_0,J-\delta,\mu^*)+o(\delta^2),
\end{split}&\ \text{If $l_c>0$.}\\
\end{cases}
\end{split}
\end{equation}
Similar to the previous chapter, we will consider a continuous MDP by letting $\delta\rightarrow0$.

\subsection{Alternative models}
The occurrence of fluid jumps can be modeled in many different ways.
We will briefly mention some alternatives to design choices that were made in the definition above.
\subsubsection{Decisions at jumps only}
We could also model the problem such that the choice to repair the asset can only be made at the instant after a jump occurs.
This might be more realistic as the jump could be caused by some mechanic performing some partial maintenance and a mechanic might be needed to completely repair the asset.

\subsubsection{Jumps not according to a Poisson process}
The time in between the jumps could also have another distribution than the exponential distribution.
This would, however, make the problem significantly more complicated as the memorylessness simplifies the problem slightly.