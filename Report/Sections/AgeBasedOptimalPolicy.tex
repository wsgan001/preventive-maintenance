\section{Analysis of the optimal policy}\label{section:AgeBasedOptimalPolicy}
Instead of finding the optimal control limit $\mu^*$ by solving the Bellman equations, we can also minimize $V(0^+,\mu)$ by looking for critical points of the expected total discounted cost.
Although we could use \eqref{eq:AgeBasedPolicyTDC} for the total discounted cost, we will use a slightly different formula to simplify the analysis.
Instead of using $V(0^+,\mu)$ on the right hand side of \eqref{eq:AgeBasedPolicyTDC}, we will use $V(0^+,\mu^*)$.
This corresponds to minimizing the expected total discounted cost for a machine using control limit $\mu$ in the first run and optimal control limit $\mu^*$ afterwards.
Obviously, this would be minimized by $\mu=\mu^*$.
We will look at critical points of
\begin{equation}\label{eq:AgeBasedHatTDC}
\hat{V}(0^+,\mu)=aF(\mu)\mathbb{E}[e^{-\beta Q_0}|Q_0\leq \mu]+(c+V(0^+,\mu^*))\mathbb{E}[e^{-\beta(Q_0\wedge\mu)}].
\end{equation}
Note that 
\[
\begin{split}
&F(\mu)\mathbb{E}[e^{-\beta Q_0}|Q_0\leq \mu]=\int\limits_0^\mu f(x)e^{-\beta x}dx\\
&\Rightarrow \frac{d}{d\mu}\left[F(\mu)\mathbb{E}[e^{-\beta Q_0}|Q_0\leq \mu]\right]=f(\mu)e^{-\beta \mu}.
\end{split}
\]
Furthermore,
\[
\begin{split}
&\mathbb{E}[e^{-\beta(Q_0\wedge\mu)}]=\int\limits_0^\mu f(x)e^{-\beta x}dx+\bar{F}(\mu)e^{-\beta\mu}\\
&\Rightarrow \frac{d}{d\mu}\mathbb{E}[e^{-\beta(Q_0\wedge\mu)}]=-\beta\bar{F}(\mu)e^{-\beta\mu}.
\end{split}
\]
So that taking the derivative of \eqref{eq:AgeBasedHatTDC} yields
\[\frac{d}{d\mu}\hat{V}(0^+,\mu)=af(\mu)e^{-\beta\mu}-(c+V(0^+,\mu^*))\beta\bar{F}(\mu)e^{-\beta\mu}.\]
We are interested in the zeroes of this derivative:
\[
\begin{split}
&af(\mu)e^{-\beta\mu}-(c+V(0^+,\mu^*))\beta\bar{F}(\mu)e^{-\beta\mu}=0\\
&\Rightarrow h(\mu)=\frac{f(\mu)}{\bar{F}(\mu)}=\beta\frac{c+V(0^+,\mu^*)}{a}
\end{split}
.\]
Note that the right hand side of this equation is a constant and the left hand side is increasing by assumption.
Hence, there is at most one $\mu$ that satisfies the above equation.
From the above bounds it can also be seen that $\hat{V}(0^+,\mu)$ is decreasing when $h(\mu)$ is smaller than this constant and increasing when it is larger.
We will now establish that if there is one $\mu$ that satisfies the equation above, it is also the global minimum of $\hat{V}(0^+,\mu)$:
\begin{lemma}
	If there is a $\hat{\mu}$ that satisfies
	\begin{equation}\label{eq:AgeBasedHazardBound}
	h(\hat{\mu})=\beta\frac{c+V(0^+,\mu^*)}{a},
	\end{equation}
	then this $\hat{\mu}$ is the optimal control limit.
	\begin{proof}
		From the previous derivation, it follows that this $\hat{\mu}$ is a stationary point of $\hat{V}(0^+,\mu)$.
		Since $h$ is increasing by assumption, we know that
		\begin{itemize}
			\item $\mu<\hat{\mu}\Rightarrow h(\mu)<h(\hat{\mu})$ so that $\frac{d}{d\mu}\hat{V}(0^+,\mu)<0$.
			\item $\mu>\hat{\mu}\Rightarrow h(\mu)>h(\hat{\mu})$ so that $\frac{d}{d\mu}\hat{V}(0^+,\mu)>0$.
		\end{itemize}
		Which establishes that $\hat{\mu}$ is the global minimum of $\hat{V}(0^+,\mu)$.
		Concluding the control limit $\hat{\mu}$ that satisfies \eqref{eq:AgeBasedHazardBound} equals the optimal control limit $\mu^*$.
	\end{proof}
\end{lemma}
\begin{corollary}
	From \eqref{eq:AgeBasedHazardBound} and the fact that $\hat{V}(0^+,\mu)$ is decreasing at $\mu=0$, it follows that
	\[
	h(0)<\beta\frac{c+V(0^+,\mu^*)}{a}.
	\]
\end{corollary}
\begin{corollary}
If for all $\mu>0$
\[
h(\mu)<\beta\frac{c+V(0^+,\mu^*)}{a},
\]
then $\hat{V}(0^+,\mu)$ is strictly decreasing and has an asymptotic minimum, concluding $\mu^*=\infty$.
Note that this also implies that for decreasing hazard rates $\mu^*=\infty$.
\end{corollary}

\begin{remark}
	Using the Bellman equations, it can also be proven without the assumption of an increasing hazard rate that if there is an optimal control limit $\mu^*$, $\mu^*$ must satisfy \eqref{eq:AgeBasedHazardBound} and the hazard rate must be increasing at $\mu^*$.
	For a proof of this, we refer to appendix \ref{AppendixAgeBasedControlLimit}.
\end{remark}

	Returning to remark \ref{remark:AgeBasedWeirdODE}:
	The differential equation \eqref{eq:AgeBasedBellmanODE} seemed counterintuitive as the total discounted cost would be decreasing for high hazards.
	
	Consider the cost of one run of the machine, we replace the repair cost by $c^*=c+V(0^+,\mu^*)$ so that the expected discounted cost of the first repair equals the expected total discounted cost of the original problem.
	$V(x,\mu^*)$ now corresponds to the cost of this altered problem, but starting with a machine of age $x$.
	If $V(x,\mu^*)$ were to be decreasing in the neighborhood of some $x$, this would mean that for that $x$, the problem would have a lower expected optimal cost if we started with a slightly older machine.
	This seems to conflict with the assumption that $h$ is increasing so that the machine deteriorates over time.
	However, we can prove that $V(x,\mu^*)$ is increasing for $x<\mu^*$:

\begin{theorem}\label{theorem:TdcNonDecreasing}
	The expected total discounted cost \eqref{eq:AgeBasedPolicyRemainingTDC} is increasing for $x< \mu^*$, i.e.
	\[
	\frac{d}{dx}V(x,\mu^*)> 0.
	\]
	\begin{proof}
		We will prove that $\frac{d}{dx}V(x,\mu^*)\leq0$ for some $x$ implies $\frac{d^2}{dx^2}V(x,\mu^*)<0$ so that $V(x,\mu^*)$ remains decreasing and will eventually be negative, contradicting the fact that all costs are positive.
		
		Let $x'<\mu^*$, it holds that
		\[
		V(x',\mu^*)\leq V(\mu^*,\mu^*)=c+V(0^+,\mu^*).
		\]
		Now we will prove that if $\frac{d}{dx}V(x',\mu^*)\leq 0$, this implies that $\frac{d^2}{dx^2}V(x',\mu^*)<0$.
		We take the derivative of \eqref{eq:AgeBasedBellmanODE}:
		\[
		\frac{d^2}{dx^2}V(x,\mu^*)=-h'(x)[c+a+V(0^+,\mu^*)-V(x,\mu^*)]+(\beta+h(x))\frac{d}{dx}V(x,\mu^*).
		\]
		We know that for our $x'$, $\frac{d}{dx}V(x,\mu^*)\leq 0$ and $V(x,\mu^*)\leq c+V(0^+,\mu^*)$.
		Furthermore, by assumption $h'(x')>0$.
		Concluding
		\[
		\frac{d^2}{dx^2}V(x',\mu^*)<0.
		\]
		This implies that if $\frac{d}{dx}V(x',\mu^*)=\leq0$ for some $x'$, $V(x,\mu^*)$ is strictly concave down after $x'$ so that for all $x>x'$
		\[
		V(x,\mu^*)< c+V(0^+,\mu^*).
		\]
		Hence no control limit will ever be and $V(x,\mu^*)$ will eventually become negative.
		This contradicts the fact that all costs are positive so that it is proven that for increasing hazard rates, $V(x,\mu^*)$ is nondecreasing for $x\leq\mu^*$.
	\end{proof}
\end{theorem}

\begin{corollary}
	The remaining discounted cost $V(x,\mu^*)$ has the following lower bound:
	\[
	V(x,\mu^*)\geq\frac{h(x)}{\beta+ h(x)}(c+a+ V(0^+,\mu^*)).
	\]
	\begin{proof}
		By theorem \ref{theorem:TdcNonDecreasing}, $\frac{d}{dx}V(x,\mu^*)\geq 0$ holds.
		The lower bound then follows from the differential equation of $V(x,\mu^*)$ \ref{eq:AgeBasedBellmanODE}.
	\end{proof}
\end{corollary}