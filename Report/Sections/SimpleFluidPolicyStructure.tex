\section{Structure of optimal policy}
In this section, we will establish that for the simple fluid model with jumps, the optimal policy is a stationary policy to repair whenever the buffer is empty and the used initial fluid $L_0(t)$ exceeds a certain control limit.
By this, we mean that there exists some optimal control limit $\mu^*>0$ and repair should be chosen whenever the buffer $L_c(t)=0$ and $L_0(t)\geq\mu^*$.

\subsection{Stationary policy}
%From the Bellman equations \eqref{eq:SimpleFluidBellman}, it can be seen that when $x_k\neq x_{BREAK}$, repair is chosen when
%\[ c+\alpha_\delta V_\delta(x_{NEW},\mu^*) <\alpha_\delta \mathbb{E}[V_\delta(S(x_k),\mu^*)]. \]
%Where the left hand side is a constant and the right hand side only depends on $S(x_k)$, which only depends on $L_0$ and $L_c$ and not on $k$ by theorem \ref{theorem:SimpleCurrentLevel}.
%Hence, we have established that the optimal policy must again be a stationary policy.
Any discretization of $L_0$ and $L_c$ results in a countable state space and the cost function is again independent of the decision epoch.
Hence, by section 6.2.4 of \cite{Puterman2008}, any discrete approximation of the continuous-time MDP has a stationary optimal policy.

\subsection{Empty buffer}
Repairing when $L_c>0$ cannot be optimal since it is certain that the machine will not break in the next $L_0$ time units and waiting more would decrease the cost.
This proves that in the optimal policy $L_c$ must be zero whenever repair is chosen.

\subsection{Control limit}
The optimal policy is a control limit policy by the same argument as for the age-based problem (see section \ref{section:AgeBasedControlLimit}). 
%For states $x\neq x_{BREAK}$, repair is chosen whenever
%\begin{equation}\label{eq:SimpleFluidRepairCondition}
%c+\alpha_\delta V_\delta(x_{NEW},\mu^*) <\alpha_\delta \mathbb{E}[V_\delta(S(x_k),\mu^*)].
%\end{equation}
%
%Now we can distinguish two cases:
%\begin{enumerate}
%	\item There is no value $l_0$ so that $x=(l_0,0)$ satisfies \eqref{eq:AgeBasedRepairCondition} and preventive repair is never the optimal choice.
%	In this case we set control limit $\mu^*=\infty$.
%	\item There are values $\mu$ so that $x=(\mu,0)$ satisfy \eqref{eq:AgeBasedRepairCondition}.
%	We can now simply set the control limit $\mu^*$ to the smallest such values.
%	What happens for larger values than this $\mu^*$ is not relevant as these will never be reached.
%\end{enumerate}

Hence, we have established that the optimal policy is a control limit on the used initial fluid $L_0$.