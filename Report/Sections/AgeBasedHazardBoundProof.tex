\chapter{Proof of properties of optimal age-based control limits}\label{AppendixAgeBasedControlLimit}
Without the assumption of increasing hazard rates, it can also be proven that if an optimal control limit $\mu^*$ exists, $\mu^*$ must satisfy \eqref{eq:AgeBasedHazardBound} and the hazard rate must be increasing at $\mu^*$.
This can be proven using the Bellman equations.
For this, we will briefly return to discretized time:
If the control limit equals $\mu^*$, then one time interval earlier, $c+V_\delta(\delta,\mu^*)\geq V_\delta(\mu^*-\delta,\mu^*)$ holds since else the control limit would be smaller than $\mu^*$.
Using \eqref{eq:gatheredDelta}, we get

\[\begin{split}
&c+V_\delta(0^+,\mu^*)\\
\geq& V_\delta(\mu^*-\delta,\mu^*)\\
=&\delta h(\mu^*)(c+a+ V_\delta(0^+,\mu^*))+(1-\delta\beta-\delta h(\mu^*)) V_\delta(\mu^*,\mu^*)+o(\delta^2).
\end{split}
\]
Since we repair at age $\mu^*$, $V_\delta(\mu^*,\mu^*)=V_\delta(0^+,\mu^*)+c$ and we can write
$$
c+V_\delta(0^+,\mu^*)\geq \delta h(\mu^*)(c+a+ V_\delta(0^+))+(1-\delta\beta-\delta h(\mu^*)) (c+V_\delta(0^+))+o(\delta^2).
$$
Which simplifies to
$$
0\geq ah(\mu^*)-\beta (c+V_\delta(0^+,\mu^*))+o(\delta^2)
$$
and can be rewritten as
$$
h(\mu^*)\leq \beta\frac{c+V_\delta(0^+,\mu^*)}{a} +o(\delta^2)\rightarrow\beta\frac{c+V_\delta(0^+,\mu^*)}{a}.
$$
If instead of looking a decision stage before the control limit, we now look at the decision stage where the control limit $\mu^*$ is reached, the Bellman equations yield
$$
c+V_\delta(0^+,\mu^*)\leq V_\delta(\mu^*-\delta,\mu^*)=\delta h(\mu^*)(c+a+ V_\delta(0^+,\mu^*))+(1-\delta\beta-\delta h(\mu^*)) V_\delta(\mu^*+\delta)+o(\delta^2)
$$
And using the same steps, we get
$$
h(\mu^*)\geq \beta\frac{c+V_\delta(0^+,\mu^*)}{a} +o(\delta^2)\rightarrow\beta\frac{c+V_\delta(0^+,\mu^*)}{a}
$$
such that the result is proven when $\delta$ approaches zero.
From the above, it also follows that the hazard rate is increasing at the control limit.
This can be summarized in the following theorem:

\begin{theorem}
	Whenever the optimal policy is to repair when the age reaches control limit $\mu^*<\infty$, it holds that the hazard rate is increasing at $\mu^*$ and
	\[h(\mu^*)=\beta\frac{c+V(0^+,\mu^*)}{a}.\]
\end{theorem}

\begin{corollary}
	If the hazard rate of the lifetime $Q_0$ of the asset is monotonously decreasing, preventive repair will never be the optimal choice.
\end{corollary}