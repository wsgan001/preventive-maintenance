\section{Literature}
\begin{itemize}
\item \citep{Asmussen1996} Discusses the Phase-Type distribution and methods to estimate the parameters.
\item \citep{Bertsekas1995} Book on dynamic programming and optimal control.
\item \citep{Bertsekas2012} Introduces a new policy iteration-like
algorithm for finding the optimal state costs or Q-factors and proves convergence properties.
\item \citep{Birge2011}
\item \citep{Buchholz2014} Concerns the Phase-Type distribution. Shows properties and proposes methods to estimate the parameters. Mentions the difficulties: Many parameters ($n^2$ for $n$ states), of which, most are redundant but in most cases there is no canonical representation for each class ($2n-1$ independent parameters).
\item \citep{Chen2014} Uses BIDE to mine closed frequent sequential patterns.
\item \citep{Fournier-Viger2017} A Survey of Sequential Pattern Mining. Describes general concepts, problem variants and the existing algorithms.
\item \citep{Gribaudo2007} Intoduces Reward Model, Fluid Model (first and second order, reflecting and absorbing boundary) and Fluid Stochastic Petri Nets. Gives differential equations for transient behavior and proposes methods to solve/compute them.
\item\citep{Hajiaghayi2014} Estimates transition probabilities for countably infinite state spaces.
\item \citep{Horton1998} Concerns Fluid Stochastic Petri Nets, summarizes the theory and provides examples.
\item\citep{Inamura2006} Concerns estimating CTMC's with discretely observed data.
\item \citep{Kalosi2016} Considers three states (perfect, satisfactory and failed) and maintenance at scheduled (fixed interval) and unscheduled (exponential interval) times and finds the policy that minimizes the cost.
\item \citep{Lee2010} Proposes a method of modelling insurance losses via mixtures of Erlang distributions. Also explains why Phase-type distributions are not practical for this situation.
\item \citep{Okamura2006} Models software faults with intervals of phase type distribution.
\item\citep{Tzenova2005} Introduces jumps that can occur at transition times, the size of the jumps are according to some distribution. The paper gives the resulting differential equations and methods to solve them.
\item\citep{Wang2004} Introduces the BIDE algorithm for finding closed frequent sequential patterns.
\item\citep{Serfozo1979} Proves that for every CTMC decision problem (where decisions are made at the beginning of the sojourn time) there is an equivalent DTMC decision problem.
\item\citep{Liu2011} (Section "Markov Chains and Spectral Clustering") Regards spectral clustering, shows that paritioning the state to minimize the cut is equivalent to minimizing to maximize the internal cohesion of the clusters. Clusters can be formed by grouping on the entries of the eigenvectors corresponding to the subdominant eigenvalues. Clusters can be made by grouping on multiple eigenvectors or by iteratively partitioning a cluster on the subdominant eigenvector.
\item\citep{Dongen2000} Introduces the MCL algorithm for clustering graphs. Focusses mainly on undirected graphs and there are not many characteristics of the resulting clusters proven.

\end{itemize}
