\section{Estimating fluid rates and jump quantities}
Estimating the fluid rates and jump quantities is more difficult as we do not observe the fluid level at each time, but only the time at which the fluid level reaches zero (i.e. when the machine breaks).
In this section we will first compute the log likelihood of rate and jump parameters given trace data and discuss maximizing this likelihood.
Then we will propose an alternative method to estimate the parameters.

\subsection{Likelihood}
Suppose we have observed a run of the machine and have seen that it started in state $s_{i_1}$, stayed there for a period time of length $\tau_1$.
Suppose also that after this time, a transition occurred to $s_{i_2}$ and the machine stayed there for a time $\tau_2$ and so forth.
Hence we have observations in the following form
$$
\sigma=\left[(i_1,\tau_1),...,(i_L,\tau_L)\right].
$$
We assume that no preventive maintenance had been done so that after the last observation in the trace, the machine failed.
We also assume that the initial distribution is known and has probability density function $f$.

For a given MMFM model $M$ with rates $r_i$ and jump quantities $J_{ij}$, this would mean that initially the fluid level was
\begin{equation}\label{eq:initialLevelDefinition}
q_0(M,\sigma)=\tau_1r_{i_1}+\sum\limits_{l=2}^{L}\tau_lr_{i_l}-J_{i_{l-1}i_l}
\end{equation}
So that the likelihood of this trace would be
$$
L(M,\sigma)=f(q_0(M,\sigma))\left[\prod\limits_{l=1}^{L-1}\lambda_{i_li_{l+1}}e^{-\lambda_{i_l}\tau_l}\right]e^{-\lambda_{i_L}\tau_L}
$$
using $\lambda_i=\sum_j\lambda_{ij}$.
If we have a set of traces $\Sigma=[\sigma_1,...,\sigma_K]$ with \[
\sigma^{(k)}=\left[(i_1^{(k)},\tau_1^{(k)}),...,(i_{L^{(k)}}^{(k)},\tau_{L^{(k)}}^{(k)})\right],
\]
then the log-likelihood would be
\begin{equation}\label{eq:MmfmLikelihood}
\begin{split}
L(M,\Sigma)&=\sum\limits_{k=1}^K\log L(M,\sigma_k)\\
&=\sum\limits_{k=1}^K\log f(q_0(M,\sigma_k))+\log\left(\left[\prod\limits_{l=1}^{L-1}\lambda_{i_ki_{k+1}}e^{-\lambda_{i_k}\tau_k}\right]e^{-\lambda_{i_L}\tau_L}\right).
\end{split}
\end{equation}

\subsection{Maximizing likelihood}
To maximize the log-likelihood \eqref{eq:MmfmLikelihood}, we take partial derivatives to the fluid rates and jump quantities.
Let us first define some quantities:
Let $\tau(i,\sigma)$ be the total time the process was in state $s_i$ for trace $\sigma$, i.e.
$$
\tau(i,\sigma)=\sum\limits_{k|i_k=i}\tau_k.
$$
Furthermore, let $\#(i,j,\sigma)$ be the number of times a transition from $s_i$ to $s_j$ occurred in trace $\sigma$.
We will now introduce two lemmas:
\begin{lemma}
	The derivative of the initial level $q_0(M,\sigma)$ to the fluid rate $r_i$ is given by
	\[
	\frac{d}{dr_i}q_0(M,\sigma)=\tau(i,\sigma).
	\]
	\begin{proof}
		The proof is straightforward:
		\[
		\frac{d}{dr_i}q_0(M,\sigma)=\frac{d}{dr_i}\left[\tau_1r_{i_1}+\sum\limits_{l=2}^{L}\tau_lr_{i_l}-J_{i_{l-1}i_l}\right]=\sum\limits_{i_l=i}\tau_l=\tau(i,\sigma).
		\]
	\end{proof}
\end{lemma}
And similarly, for the jump quantity $J_{ij}$:
\begin{lemma}
	The derivative of the initial level  $q_0(M,\sigma)$ to the jump quantity $J_{ij}$ is given by
	\[
	\frac{d}{dJ_{ij}}q_0(M,\sigma)=\#(i,j,\sigma).
	\]
	\begin{proof}
		Again:
		\[
		\frac{d}{dJ_{ij}}q_0(M,\sigma)=\frac{d}{dJ_{ij}}\left[\tau_1r_{i_1}+\sum\limits_{l=2}^{L}\tau_lr_{i_l}-J_{i_{l-1}i_l}\right]=-\#(i,j,\sigma).
		\]
	\end{proof}
\end{lemma}

Before we take the derivative of \eqref{eq:MmfmLikelihood} to $r_i$, we note that only $\log f(q_0(M,\sigma))$ depends on $r_i$ so that the other term vanishes.
Hence:
\[
\frac{d}{dr_i}\log L(M,\Sigma)
=\frac{d}{dr_i}\sum_k\log f(q_0(M,\sigma^{(k)}))=\sum_k\frac{f'(q_0(M,\sigma^{(k)}))}{f(q_0(M,\sigma^{(k)}))}\tau(i,\sigma^{(k)}).
\]
Similarly, for the jump quantities $J_{ij}$, we get
$$
\frac{d}{dJ_{ij}}\log L(M,\Sigma)=\frac{d}{dJ_{ij}}\sum_k\log f(q_0(M,\sigma^{(k)}))=-\sum_k\frac{f'(q_0(M,\sigma^{(k)}))}{f(q_0(M,\sigma^{(k)}))}\#(i,j,\sigma^{(k)}).
$$
The maximum likelihood estimators $\hat r_i$ and $\hat J_{ij}$ are then a solution to the set of equations 
\[
\frac{d}{dr_i}\log L(M,\Sigma)=0,
\]
and
\[
\frac{d}{dJ_{ij}}\log L(M,\Sigma)=0,
\]
for all $i,j\in\{1,...,N\}$.

\begin{remark}
	Note that this maximum likelihood estimator for the fluid rates and jump quantities does not depend on the transition rates of the CTMC.
\end{remark}

\begin{remark}
	It may be difficult to find a solution to these equations.
	Alternatively, we could also find estimates by numerically maximizing the likelihood \eqref{eq:MmfmLikelihood}. 
\end{remark}

\subsection{Minimizing variance}
We will now propose an alternative method to estimate the fluid rates and jump quantities.
The machine is likely produced by a manufacturer that strives for a constant quality of the produced goods (i.e. wants to maintain continuity).
Hence, we could expect that the initial fluid level (which corresponds to the initial fitness of the machine), has a low variance.
%Therefore, we will assume that the initial fluid level has a low variance.
Given trace data, we will therefore try to find MMFM parameters that minimize the variance of the initial fluid level.
From \eqref{eq:initialLevelDefinition} we can find the initial fluid levels for given parameters.
We then still need to fix an average initial fluid level $\bar q$.
Note that it does not matter which value we choose for $\bar q$ as \eqref{eq:initialLevelDefinition} is linear so that multiplying $\bar q$ by a constant will merely result in the parameters being multiplied by the same constant.
We then compute the variance for given parameters and trace data by squaring the difference between the initial level and the average $\bar q$.
Hence, we will minimize the following goal function:
\begin{equation}\label{eq:MinVarianceGoalFunction}
G(M,\Sigma)=\frac1K\sum_k \left(\bar q - q_0\left(M,\sigma^{(k)}\right)\right)^2.
\end{equation}

\subsection{Results}
Although we haven't been able to analyze the method of minimizing variance, we have implemented it in Matlab.
We have tested it with simulated trace data and compared the resulting parameters with the original parameters.
The performance of the method depends a lot on the variance of the initial distribution.
For distributions with large variance, it often occurs that the method manages to minimize the goal function \eqref{eq:MinVarianceGoalFunction} below the actual variance of the distribution.
This results in incorrect parameters.
However, the accuracy seems to improve for smaller variances.