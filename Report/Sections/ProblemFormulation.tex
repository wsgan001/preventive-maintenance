\section{Problem Formulation}
We consider a machine that is subject to deterioration over time.
Calendar time is discretized in steps of size $\delta$, i.e. the $k$'th decision stage is at time $t_k=k\delta$.
We refer to the interval $(k\delta,(k+1)\delta]$ as the $k$'th time interval.
At stage $k\in\mathbb{N}\cup\{0\}$, we denote the state of the machine by $x_k\in X=\mathbb{N}\cup\{0\}$ and initially $x_0=1$.
When $x_k=x$, this means that after the $k$'th time interval, the machine has age $x\delta$.
We now introduce some random disturbances $\omega_k:=\omega_k(x_k)$ at decision stage $k$ which only depend on $x_k$.
\begin{equation}
\omega_k(x_k):=\begin{cases}
1,&\text{if the machine lives longer than the}\ k\text{'th decision time given}\\
&\text{that it has been alive for}\ x_k\ \text{decision times} \\
0,&\text{else}.
\end{cases}
\end{equation}
At times step $k$, we shall choose an action $u_k$ from the action set $U(x_k)$.
Where
\begin{equation}
U(x_k):=\begin{cases}
\{a_W,a_R\},&\text{if}\ x_k>0 \\
\{a_R\},&\text{if}\ x_k=0.
\end{cases}
\end{equation}
These actions are
\begin{itemize}
\item $a_R$:
Repair (or replace) the machine.
\item $a_W$:
Wait and do not repair the machine.
\end{itemize}
The state evolves in the following way
\begin{equation}
x_{k+1}=f(x_k,u_k,\omega_k):=\begin{cases}
1,&\text{if}\ u_k=a_R \\
0,&\text{if}\ u_k=a_W\ \text{and}\ \omega_k=0 \\
x_k+1,&\text{if}\ u_k=a_W\ \text{and}\ \omega_k=1.
\end{cases}
\end{equation}
For convenience, we define the random variable $S(x_k):=f(x_k,a_W,\omega_k)$.
Note that in the above definition, repairing the machine takes exactly one time interval.
When the machine is in state $x_k$ and the action $u_k$ is chosen, the following cost is incurred
\begin{equation}
g(x_k,a_k):=\begin{cases}
c+a,&\text{if}\ x_k=0 \\
c,&\text{if}\ x_k>0\ \text{and}\ u_k=a_R \\
0,&\text{else}.
\end{cases}
\end{equation}
Furthermore, a discount $\alpha_\delta$ is introduced such that costs $n$ decision stages in the future are discounted by $\alpha_\delta^n$.
We consider the total discounted cost
\begin{equation}
V(x_k)=\sum\limits_{m=k}^\infty \alpha_\delta^mg(x_m,u_k)
\end{equation}
We want to find a stationary policy $\mu:X\rightarrow \{a_W,a_R\}$ that chooses the action $u_k=\mu(x_k)$ that minimizes the expected total discounted cost $V_\mu(x_k)$ for each state.
$V_\mu(x_k)$ is given by
$$
V_\mu(x_k)=g(x_k,\mu(x_k))+\alpha_\delta \mathbb{E}[V_\mu(f(x_k,\mu(x_k),\omega_k))]
$$
The Bellman equations for the optimal cost $V^*$ read
\begin{equation}
V^*(x_k)=\begin{cases}
\min\{c+\alpha_\delta V^*(1),\alpha_\delta \mathbb{E}[V^*(S(x_k))]\},&\text{if}\ x_k>0 \\
c+a+\alpha_\delta V^*(1),&\text{else.}
\end{cases}
\end{equation}
$\mu$ is optimal if $V_\mu(x)=V^*(x)$ for all $x$.

\subsection{Alternative models}
\subsubsection{Instantaneous repairs}
If we would want to make repairs take zero time, we would have to change the definition of $x_{k+1}$ to 
\begin{equation}
x_{k+1}=f_2(x_k,u_k,\omega_k):=\begin{cases}
0,&\text{if}\ \omega_k=0 \\
2,&\text{if}\ u_k=a_R\ \text{and}\ \omega_k=1\\
x_k+1,&\text{if}\ u_k=a_W\ \text{and}\ \omega_k=1.
\end{cases}
\end{equation}
This would however, introduce the possibility of having to correctively repair the machine twice in a row.

\subsubsection{Stochastic inter-decision times}
Another possibility would be to have positive stochastic i.i.d. inter-decision times $\Delta_k$ and to use a continuous discount such that costs at time $t$ are discounted by $e^{-\beta t}$ for $\beta>0$.
The state space could then be modeled as $X=\mathbb{R}^+\cup\{0,x_f\}$ where $x_f$ denotes that the machine is broken.
The state evolution would be as follows

\begin{equation}
x_{k+1}=f(x_k,u_k,\omega_k,\Delta_k):=\begin{cases}
\Delta_k,&\text{if}\ u_k=a_R \\
x_f,&\text{if}\ u_k=a_W\ \text{and}\ \omega_k=0 \\
x_k+\Delta_k,&\text{if}\ u_k=a_W\ \text{and}\ \omega_k=1.
\end{cases}
\end{equation}
The Bellman equations should then be changed to
\begin{equation}
V^*(x_k)=\begin{cases}
\min\{c+\mathbb{E}[e^{-\beta \Delta} V^*(\Delta)],\mathbb{E}[e^{-\beta \Delta} V^*(f(x_k,a_W,\omega_k,\Delta))],&\text{if}\ x_k\neq x_f \\
c+a+\mathbb{E}[e^{-\beta \Delta} V^*(\Delta)],&\text{else.}
\end{cases}
\end{equation}
Where repair would again take one inter-decision time. Where $\Delta$ is of the same family of i.i.d. random variables as the $\Delta_i$'s.

\subsubsection{Time to live in state information}
The time-to-live could also be included in the state information.
We would then denote the cost when the machine still has a time $q$ to live by $V(q)$ and in each time step, this time-to-live would decrease by $\delta$ until the time-to-live is negative, when the machine must be repaired.
The total discounted cost would then be $\mathbb{E}[V(Q)]$ and the Bellman equations would be:
$$
V(q)=\min\{
c+\alpha_{\delta} \mathbb{E}[V(Q)],
\alpha_{\delta} V(q-\delta)
\}
$$
for $q>0$ and
$$
V(q)=c + a + \alpha_{\delta} \mathbb{E}[V(Q)],
$$
else.
These equations are easy to solve although the resulting policy is not very useful in a realistic setting:
If $0<q\leq\delta$, then 
\begin{equation}\begin{split}
V(q)&=\min\{
c+\alpha_{\delta} \mathbb{E}[V(Q)],\alpha_{\delta} V(q-\delta)\}\\
&=\min\{
c+\alpha_{\delta} \mathbb{E}[V(Q)],\alpha_{\delta}(c+a+\alpha_{\delta} \mathbb{E}[V(Q)])\}\\
&=c+\alpha_{\delta} \mathbb{E}[V(Q)]
\end{split}
\end{equation}
(Assuming $a>\frac{1-\alpha_\delta}{\alpha_\delta}(c+\alpha_\delta\mathbb{E}[V(Q)])$, note that for sufficiently small discount factor $\alpha_\delta$, this does not hold.).
And by induction, we can prove that for $(k-1)\delta<q\leq k\delta$ ($k>0$)
\begin{equation}\begin{split}
V(q)&=\min\{
c+\alpha_{\delta} \mathbb{E}[V(Q)],\alpha_{\delta} V(q-\delta)\}\\
&=\min\{
c+\alpha_{\delta} \mathbb{E}[V(Q)],\alpha_{\delta} \alpha_{\delta}^{k-2}(c+\alpha_\delta\mathbb{E}[V(Q)])\}\\
&=\alpha_{\delta}^{k-1}(c+\alpha_\delta\mathbb{E}[V(Q)]).
\end{split}
\end{equation}
As can be seen, the cost does not depend on $a$ as correctively repairing never occurs. This is because in this formulation, the time-to-live is known when making the decision (when choosing the minimum). Such that we will repair at the last opportunity before the machine breaks down.
If, for instance, we choose $alpha_\delta=e^{-\beta \delta}$ for $\beta>0$, we have for $(k-1)\delta<q\leq k\delta$
$$
V(q)=e^{-\beta(k-1)\delta}(c+\alpha_\delta\mathbb{E}[V(Q)]).
$$
If we let $\delta$ approach zero, the cost approaches 
$$
V(q)=e^{-\beta q}(c+\alpha_\delta\mathbb{E}[V(Q)]).
$$
As can be seen, this is the cost of repairing the machine preventively at exactly the instant that it will break, which is indeed optimal but can unfortunately not be realized as the time-to-live cannot be directly observed.