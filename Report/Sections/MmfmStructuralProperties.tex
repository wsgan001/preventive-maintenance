\section{Structural properties}\label{section:MmfmStructuralProperties}
In this section, the effect of changing the parameters to the expected total discounted cost and the control limits are investigated.
These structural properties are mostly similar to the simple fluid problem.
The main difference is that there are multiple control limits for the various CTMC-states and that the control limit in a certain state is also influenced by the costs in other states.

\begin{remark}
	Referring back to the equation for the optimal control limit $\mu_i^*$ for a CTMS-state $s_i$ \eqref{eq:MmfmHazardBoundsShort}: if some change of the parameters would cause an increase in the expected remaining cost for some state $s_j$ that neighbors $s_i$ in the CTMC defined by the generator $\Lambda^D$ (i.e. if $\Lambda^D_{ij}>0$), then $\Lambda^D_{ij}V(j,\mu_i^*,0,\pi^*)$ would increase so that the hazard at which repair is chosen must decrease.
	This results in a lower control limit.
\end{remark}

Furthermore, there are also different fluid rates for different CTMC-states:
\begin{remark}
	An increase in the fluid rate $r_i$ for some state $s_i$ increases the hazard in that state.
	This results in a lower control limit, as one would expect as a higher fluid rate corresponds to the asset deteriorating quicker in that state.
\end{remark}

Again, appendix \ref{AppendixComputationsTable} contains computed values of the optimal control limit and the corresponding expected total discounted cost.