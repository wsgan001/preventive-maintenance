\section{Computation of total discounted cost}
In this section, the Bellman equations will be used to find the expected total discounted cost of the optimal cost.
From the discrete Bellman equations \eqref{eq:AgeBasedBellman}, differential equations will be derived.
The following Bellman equations will be considered
\[V_\delta^*(x)=\begin{cases}
\min\{c+\alpha_\delta V^*_\delta(1),\alpha_\delta \mathbb{E}[V^*_\delta(S(x))]\},&\text{if}\ x>0 \\
c+a+\alpha_\delta V^*_\delta(1),&\text{else.}
\end{cases}\]
We assume that for $x$
$$
c+\alpha_\delta V^*_\delta(0^+)>\alpha_\delta \mathbb{E}[V^*_\delta(S(x)\delta)],
$$
i.e. the optimal control limit $\mu^*>\delta x$.
Now, we can write $V^*_\delta(x)$ in the following way
\[V^*_\delta(x)=\alpha_\delta \bar F(x;x-\delta) (c+a+\alpha_\delta V^*_\delta(0^+))
+\alpha_\delta \bar F(x;x-\delta)V^*_\delta(x+\delta)
\]
We are now going to let $\delta$ approach zero.
\begin{equation}
\begin{split}
\lim\limits_{\delta\rightarrow 0} V_\delta^*(x) &=
\lim\limits_{\delta\rightarrow 0}(1-\beta\delta+o(\delta^2)) (\delta h(x)+o(\delta^2))(c+a+(1-\beta\delta+o(\delta^2)) V^*_\delta(0^+))\\
&+(1-\beta\delta+o(\delta^2)) (1-\delta h(x)+o(\delta^2))V_\delta^*(x+\delta).
\end{split}
\end{equation}
Gathering the terms of $o(\delta^2)$, we get
\begin{equation}\label{eq:gatheredDelta}
\begin{split}
\lim\limits_{\delta\rightarrow 0} V_\delta^*(x) =
\lim\limits_{\delta\rightarrow 0}\delta h(x)(c+a+ V_\delta^*(0^+))+(1-\delta\beta-\delta h(x)) V_\delta^*(x+\delta)+o(\delta^2).
\end{split}
\end{equation}
By moving one $V_\delta^*(x+\delta)$ to the left and dividing by $-\delta$, we get
\begin{equation}\label{eq:AgeBasedBellmanODE}
\begin{split}
\frac{d}{dx}V^*(x)&=\lim\limits_{\delta\rightarrow 0} \frac{V_\delta^*(x+\delta)-V_\delta^*(x)}{\delta} \\
&=\lim\limits_{\delta\rightarrow 0} -h(x)(c+a+ V_\delta^*(0^+))+(\beta+ h(x)) V_\delta^*(x+\delta)+o(\delta)\\
&=-h(x)(c+a+ V^*(0^+))+(\beta+ h(x)) V^*(x).
\end{split}
\end{equation}
\begin{remark}\label{remark:AgeBasedWeirdODE}
	The differential equation \eqref{eq:AgeBasedBellmanODE} seems counterintuitive as for high hazard rates, the expected total discounted cost would be decreasing for increasing age of the asset.
	We will return to this later on.
\end{remark}
We will now try to solve this O.D.E.
We use the method of the integrating factor.
Our integrating factor will be
$$
e^{\int\limits_0^x (-\beta - h(q))dq}=e^{-\beta x -H(x)}.
$$
Where $H(x)$ is the cumulative hazard function.
We get
$$
V^*(x)=e^{\beta x +H(x)} [C + \int\limits_0^x e^{-\beta q -H(q)}(-h(q)(c+a+V^*(0^+)))dq]
$$
$$
=\frac{e^{\beta x}}{\bar F(x)} [C - (c+a+V^*(0^+))\int\limits_0^x e^{-\beta q}f(q)dq].
$$
Using the identities 
\[e^{H(x)}=(e^{-H(x)})^{-1}=\frac{1}{\bar F(x)},\]
and
\[h(x)e^{-H(x)}=f(x).\]
$C$ is an integrating constant and since $\lim\limits_{x\rightarrow 0}V^*(x)=V^*(0^+)$ should hold, we find $C=V^*(0^+)$.
We can rewrite the expression to
$$
V^*(x)=\frac{e^{\beta x}}{\bar F(x)} [V^*(0^+)  - (c+a+V^*(0^+))F(x)\mathbb{E}[e^{-\beta Q_0}|Q_0<x]].
$$
Which results in the following theorem:
\begin{theorem}
	When the asset has age $x$, the expected remaining total discounted cost equals
	\begin{equation}\label{eq:AgeBasedSolvedBellman}
	\begin{split}
	V^*(x)=\min\{&c+V^*(0^+),\\
	&\frac{ e^{\beta x}}{\bar F(x)} [V^*(0^+) - (c+a+V(0^+))F(x)\mathbb{E}[e^{-\beta Q_0}|Q_0<x]]
	\}
	\end{split}
	\end{equation}
	and preventive maintenance is chosen if and only if $V^*(x)\geq c+V^*(0^+)$.
\end{theorem}
Unfortunately, the value of $V^*(0^+)$ depends on the optimal policy and it is also difficult to solve $V^*(x)=c+V(0^+)$ analytically for $x$.
Control limit $\mu^*$ is then chosen as the smallest positive $x$ that satisfies $V(x)=c+V(0^+)$ if such $x$ exist and $\mu^*=\infty$ else.
The policy that we just derived, schedules preventive maintenance at time $\mu^*$ if the asset has not already failed by then.
We denote the total discounted cost of this policy by $V(0^+,\mu^*)$.

For any (possibly sub-optimal) control limit $\mu$, we can derive the expected remaining total discounted cost.
The length of one run of the asset is the minimum of its lifetime $Q_0\sim F$ and the chosen control limit $\mu$.
At the end of each run, at least $c$ is paid.
If the run ends because the asset broke (i.e. $Q_0<\mu$), an additional cost of $a$ is paid.
Hence, we get the following expression for $V(0^+,\mu)$
\begin{equation}\label{eq:AgeBasedPolicyTDC}
V(0^+,\mu)=aF(\mu)\mathbb{E}[e^{-\beta Q_0}|Q_0\leq \mu]+(c+V(0^+,\mu))\mathbb{E}[e^{-\beta(Q_0\wedge\mu)}].
\end{equation}
Where $A\wedge B$ denotes the minimum af $A$ and $B$.
We get a similar expression for $V(x,\mu)$:

\begin{theorem}
	Given control limit $\mu$ and an asset of age $x$, the expected remaining total cost equals 
	\begin{equation}\label{eq:AgeBasedPolicyRemainingTDC}
	V(x,\mu)=aF(\mu;x)\mathbb{E}[e^{-\beta Q_0}|x<Q_0\leq \mu]+(c+V(0^+,\mu))\mathbb{E}[e^{-\beta(Q_0\wedge\mu)}|Q_0>x].
	\end{equation}
\end{theorem}

\begin{remark}
	By taking the derivative of \eqref{eq:AgeBasedPolicyRemainingTDC}, one can see that this also adheres to the differential equation \eqref{eq:AgeBasedBellmanODE} that resulted from the Bellman equations.
\end{remark}

\begin{remark}
	In theorem \ref{theorem:TdcNonDecreasing}, we will prove that for increasing hazard rates, $V(x,\mu)$ is increasing for $x<\mu$.
\end{remark}

\begin{remark}
	\eqref{eq:AgeBasedPolicyTDC} can also be rewritten to get an explicit expression for $V(0^+,\mu)$:
	\begin{equation}\label{eq:AgeBasedOptimalTDC}
	V(0^+,\mu)=\frac{aF(\mu)\mathbb{E}[e^{-\beta Q_0}|Q_0\leq \mu]+c\mathbb{E}[e^{-\beta(Q_0\wedge\mu)}]}{1-\mathbb{E}[e^{-\beta (Q_0\wedge\mu)}]}.
	\end{equation}
\end{remark}

\begin{remark}
	If $\mu^*=\infty$, then the expected total discounted cost equals
	\[V(0^+,\infty)=\mathbb{E}[e^{-\beta Q_0}](c+a+V(0^+,\infty)).\]
	Which can be rewritten to
	\[V(0^+,\infty)=\frac{\tilde{F}(-\beta)}{1-\tilde{F}(-\beta)}(c+a),\]
	where $\tilde{F}$ is the moment generating function of $Q_0$.
\end{remark}

\begin{example}
	Let $Q_0\sim\text{Exp}(\lambda)$.
	Because of the memoryless property, we would expect $V(x,\mu^*)$ to be constant.
	Filling this in into \eqref{eq:AgeBasedBellmanODE}, we get
	
	\[
	\begin{split}
	0&=\frac{d}{dx}V(x,\mu^*)\\
	&=-\lambda(c+a+ V(0^+,\mu^*))+(\beta+ \lambda) V(x,\mu^*)\\
	&=-\lambda(c+a+ V(0^+,\mu^*))+(\beta+ \lambda) V(0^+,\mu^*).
	\end{split}
	\]
	Where the last equality holds as $V(x,\mu^*)$ is constant.
	This can be rewritten to
	$$
	V(0^+,\mu^*)=\frac{\lambda}{\beta}(c+a)
	$$
	which equals exactly the total discounted cost for control limit $\mu^*=\infty$.
\end{example}