\section{Structure of optimal policy}
In this section, we will establish that for the age-based maintenance problem, the optimal policy is a stationary control limit policy.
This means that repair is chosen if and only if the age has exceeded a certain threshold $\mu$ (the control limit).
If no repair is chosen, then we set $\mu=\infty$.

\subsection{Stationary policy}\label{section:AgeBasedStationaryPolicy}
Referring back to the Bellman equations of the age-based problem \eqref{eq:AgeBasedBellman}, you can see that when $x_k>0$, repair is chosen whenever
\[ c+\alpha_\delta V^*_\delta(1) <\alpha_\delta \mathbb{E}[V^*_\delta(S(x_k))]. \]
Where the left hand side is a constant and the right hand side only depends on $S(x_k)$, which only depends on the age of the machine by assumption.
Hence, we have established that the optimal policy only depends on the age $x_k$ so that it must be a stationary policy.

\subsection{Control limit}
Using similar reasoning, for states $x>0$, repair is chosen whenever
\begin{equation}\label{eq:AgeBasedRepairCondition}
 c+\alpha_\delta V^*_\delta(1) <\alpha_\delta \mathbb{E}[V^*_\delta(S(x))]. 
 \end{equation}
Now we can distinguish two cases:
\begin{enumerate}
	\item There is no age $x$ that satisfies \eqref{eq:AgeBasedRepairCondition} and preventive repair is never the optimal choice.
	In this case we set control limit $\mu=\infty$.
	\item There are ages $x_1,x_2,...$ that satisfy \eqref{eq:AgeBasedRepairCondition}.
	The control limit will now simply be the smallest such age.
	What happens for ages greater than this $\mu$ is not relevant as these will never be reached.
\end{enumerate}
Hence, we have established that the optimal policy must be of control limit type where the machine is repaired whenever its age exceeds some threshold $\mu$.