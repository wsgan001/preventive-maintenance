\section{Computation of total discounted cost}
In this section, the expected total discounted cost when using a control limit policy $\pi=[\mu_1,...,\mu_N]$ is calculated.

Similar to \eqref{eq:SimpleFluidDiscountDefinition}, but keeping in mind that there are multiple CTMC-states, we define
\[
D_{i}^t(q):=\mathbb{E}[e^{-\beta T_t(q)}|S(t)=i],
\]
as the expected discount over the time until the fluid level $Q(t)$ is decreased by $q$, given that the CTMC was in $s_i$ at time $t$.
In this definition, we disregard failures and policies (i.e. we briefly assume that $Q_0=\infty$ and all control limits $\mu_i=\infty$).
Furthermore, we define
\[
D_{ij}^t(q):=\mathbb{E}[e^{-\beta T_t(q)}\mathds{1}\{S(t+T_t(q))=j\}|S(t)=i].
\]
Note that these expectations do not depend on calendar time but only on the state of the background CTMC when the fluid is at level $Q(t)$.
Hence, we will omit $t$ in these notations.
Note that
\[
D_{i}(q)=\sum\limits_{j}D_{ij}(q).
\]
When we take the policy $\pi=[\mu_1,...,\mu_N]$ into account, states cannot be reached via paths that go over a control limit.
We define similar quantities:
\[
\begin{split}
&D_{ij}^t(q,\pi,l)\\
&:=\mathbb{E}\left[e^{-\beta T_t(q)}\mathds{1}\left\{\begin{split}
&S(t+T_t(q)=j),\\
&\forall \tau\in[t,t+T_t(q)):
\begin{split}
&L_c(\tau)>0\\&\text{ or } L_0(\tau)<\mu_{S(\tau)}
\end{split}
\end{split}\right\}\middle| S(t)=i,L_0(t)=l\right],
\end{split}
\]
and
\[
D_{i}^t(q,\pi,l):=\sum\limits_{j:\mu_j\leq l+q}D_{ij}^t(q,\pi,l).
\]
Where these are similar to the earlier defined $D_{ij}$ but without the asset being repaired before fluid level $Q(t)-q$ is reached.
These expectations also do not depend on $t$.
Hence, $t$ will again be omitted in the notation.
If preventive repair is chosen, then it is chosen when $L_0$ equals some random variable $R$.
We define the following quantity
\[
\begin{split}
\Gamma_i^t(q,\pi,l):=\lim\limits_{\delta\rightarrow 0}\frac1\delta&\mathbb{P}(q\leq R<q+\delta|S(t)=i,L_0(t)=l)\\
&\times\mathbb{E}[e^{-\beta T_t(R)}|q\leq R<q+\delta,S(t)=i,L_0(t)=l].
\end{split}
\]
The interpretation of this quantity can be difficult, it could be seen as the density of $R$ multiplied by the expected discount factor given that repair is chosen at this level of $L_0$.
Note that $\Gamma_i^t(q,\pi,l)$ also does not depend on $t$ so that $t$ will again be omitted in the notation.
\begin{remark}
	These defined quantities should be interpreted as neither probabilities nor discount factors but more as a combination of these two: The expected discount factor given some event, multiplied by the probability (or density) of this event.
	We will refer to $D_{ij}(q),D_{ij}(q,\pi,l)$ and $D_{i}(q,\pi,l)$ as 'discounted probabilities' and $\Gamma_i(q,\pi,l)$ as a 'discounted density'.
\end{remark}
These discount quantities will be computed in the next section.
We will now derive the expected total discounted cost of a policy $\pi$.
Each run of the asset can end in two ways:
\begin{enumerate}
	\item The asset breaks,
	\item or preventive maintenance is chosen.
\end{enumerate}
We will split the total discounted cost in terms corresponding to these two scenarios.

\subsubsection{The asset breaks}
When the asset breaks, it means that $L_0(t)$ has reached $Q_0$ without encountering a control limit.
Hence, the repair costs are discounted at $D_1(Q_0,\pi,0)$.
The expected value of this, is given by
\[
\mathbb{E}[D_{1}(Q_0,\pi,0)]=\int\limits_0^\infty f(q)D_{1}(q,\pi,0)dq.
\]
We get the following term in the expression of the total discounted cost
\[
\mathbb{E}[D_{1}(Q_0,\pi,0)](c+a+V(x_{NEW},\pi)).
\]

\subsection{Preventive maintenance is chosen}
When preventive maintenance is chosen, it can either be chosen in a state $x=(i,l_0,0)\in X$ where
\begin{enumerate}
	\item $l_0=\mu_i$, preventive maintenance is chosen 'at the control limit';
	\item or $l_0>\mu_i$, preventive maintenance is chosen 'after the control limit'.
\end{enumerate}
We will again split the total discounted cost in terms corresponding to these two scenarios.

\subsubsection{Preventive maintenance is chosen at the control limit}
In this case, repair is chosen in a state $s_i$ with $l_0=\mu_i$.
For this to happen, it must be the case that $Q_0>\mu_i$, with probability $\bar{F}(\mu_i)$.
Preventive maintenance has a cost $c$, discounted at $D_{i}(\mu_i,\pi,0)$.
Hence, we get the following term:
\[
\sum\limits_i\bar{F}(\mu_i)D_{i}(\mu_i,\pi,0)(c+V(x_{NEW},\pi)).
\]

\subsubsection{Preventive maintenance is chosen after the control limit}
In this case, repair is chosen in some CTMC-state $s_i$ and $l_0>\mu_i$.
For this to be able to happen, $Q_0>l_0$ must hold with probability $\bar{F}(l_0)$.
This event would have cost $c$ and discounted density $\Gamma_0(l_0,\pi,0)$.
Hence, this results in the following term:
\[
\left[\int\limits_1^\infty \bar{F}(q)\Gamma_0(q,\pi,0)dq\right](c+V(x_{NEW},\pi)).
\]
Concluding:
\begin{theorem}
	The expected total discounted cost of a policy $\pi$ is given by
	\begin{equation}\label{eq:MmfmTDC}
	\begin{split}
	V(x_{NEW},\pi)=&\mathbb{E}[D_{1}(Q_0,\pi,0)]\left(c+a+V(x_{NEW},\pi)\right)\\
	&+\left[\int\limits_0^\infty \bar{F}(q)\Gamma_1(q,\pi,0)dq+\sum\limits_i\bar{F}(\mu_i)D_{i}(\mu_i,\pi,0)\right]\left(c+V(x_{NEW},\pi)\right).
	\end{split}
	\end{equation}
\end{theorem}