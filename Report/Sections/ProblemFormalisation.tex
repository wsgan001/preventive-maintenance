\section{Problem Formalisation}
The problem of scheduling preventive maintenance in order to minimize cost can be modeled as a dynamical programming problem. 

The set of possible actions will be $U(x_t)=U=\{"Repair", "No Repair"\}$. The random disturbances and the state space will follow from the stochastic model that will be chosen. We assume that preventive maintence has a certain cost $c("Repair")=c_p$ and corrective maintenace has a cost $c_c>c_p$.

We assume that the problem can be modeled as a renewal process such that after each repair (corrective or preventive), the process starts over.

We want to minimize the expected cost per time unit.

Since the machine will not be immediately repaired when it is decided that this is necessary, a certain time lag could be applicable. For simplicity, we will omit this aspect for now.

It can also be modelled as an optimal stopping problem.
Perfect state information.
% Correlated disturbances
% Limited lookahead
