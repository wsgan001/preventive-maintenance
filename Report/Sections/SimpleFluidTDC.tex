\section{Computation of total discounted cost}
In this section the expected total discounted cost is calculated corresponding to following a control limit policy with control limit $\mu$.
As in the age-based maintenance problem, we can use the Bellman equations to find differential equations to which the total discounted cost adheres.
But for this problem, we will opt for the simpler approach of directly calculating the total discounted costs corresponding to the policies.

%When in a run of the asset $Q_0\leq\mu$, we know that we will have to correctively repair after this run.
%Else, we will have to preventively repair.
The challenge lies in calculating the discount over these costs.
In the age-based problem, this was simple as there were no jumps and the discount after $q$ fluid (age) was depleted was simply $e^{-\beta q}$.
In this problem, we need the distribution of the time it takes until $q$ initial fluid is depleted.

\subsection{Time until control limit}
Let $T_t(q)$ be the random variable denoting the time until the fluid level is $q$ lower than it was at time $t$, i.e.
$$
T_t(q)=\min\{\tau\geq0|Q(t+\tau)\leq Q(t)-q\}.
$$
Note that, using this definition, $T_0(\mu)$ equals the time until the control limit is reached ($L_0(t)=\mu$) and $T_0(Q_0)$ equals the time until the asset fails.
\begin{lemma}\label{lemma:TimeUntilFluidLemma}
	For any $A,B>0$:
	$A\leq\mu\Leftrightarrow T_t(A)\leq T_t(B)$
	\begin{proof}
		$\Rightarrow$: 
		\[\begin{split}
		A\leq\mu&\Rightarrow Q(t)-B\leq Q(t)-A\\
		&\Rightarrow (Q(t+\tau)\leq Q(t)-B\Rightarrow Q(t+\tau)\leq Q(t)-A)\\
		&\Rightarrow T_t(A)\leq T_t(B)\\
		\end{split}\]
		$\Leftarrow$: We will prove that $A>B\Rightarrow T_t(A)> T_t(B)$:\\
		We know that
		$$
		A>B\Rightarrow Q(t)-B > Q(t)-A.
		$$
		Since $Q(t)$ is piecewise continuous and does not decrease at the discontinuities, we know that 
		$$
		Q(t+T_t(B))=Q(t)-\mu>Q(t)-A\Rightarrow T_t(A)> T_t(B).
		$$
	\end{proof}
\end{lemma}
To find the distribution of $T_t(q)$, we will condition on the number of jumps.
Let $N_t(q)$ be the random variable denoting the number of jumps that occur in the interval $(t,t+T_t(q)]$.
We will now compute its distribution.
The probability that zero jumps occur equals the probability that the exponentially distributed time interval is larger than $q$:
$$
\mathbb{P}(N_t(q)=0)=e^{-\lambda q}.
$$
The probability that exactly one jump occurs equals the probability that exactly one Poisson event happens in the interval $(t,t+q]$ while none happen in $(t+q,t+q+J]$. Resulting in
$$
\mathbb{P}(N_t(q)=1)=\lambda q e^{-\lambda q} e^{-\lambda J}=\lambda q e^{-\lambda (q+J)}.
$$

For each $k\geq0$, by conditioning on the time until the first jump, we get the following recursion
\begin{equation}\label{eq:SimpleFluidJumpsRecursion}
\begin{split}
\mathbb{P}(N_t(q)=k+1)&=\int\limits_0^{q}\lambda e^{-\lambda x}\mathbb{P}(N_t(q-x+J)=k)dx\\
&=\int\limits_0^{q}\lambda e^{-\lambda (q-y)}\mathbb{P}(N_t(y+J)=k)dy,
\end{split}
\end{equation}
since after this first jump, the fluid level equals $q-x+J$ and $k$ jumps should occur. The second equality follows after the substition $y=q-x$.
The solution of this recursion, is given by the following lemma.
\begin{lemma}
	The probability that $k$ jumps occur before the fluid is decreased by $q$ equals
	
	\[\mathbb{P}(N_t(q)=k)=\lambda q\frac{(\lambda (q+kJ))^{k-1}}{k!}e^{-\lambda(q+kJ)}.\]
	
	\begin{proof}
		This can be seen by substituting this expression into \eqref{eq:SimpleFluidJumpsRecursion} and setting $k=1$ to see that it also satisfies the calculated expression for $\mathbb{P}(N_t(q)=1)$.
	\end{proof}
\end{lemma}

Now we can define the quantity $D(q)$ as the expected discount over the time until $q$ initial fluid is used in the following way
\begin{equation}\label{eq:SimpleFluidDiscountDefinition}
D(q):=\mathbb{E}[e^{-\beta T_t(q)}]=\sum\limits_{k=0}^\infty e^{-\beta(q+kJ)}\mathbb{P}(N_t(q)=k).
\end{equation}
Note that this $D$ does not depend on $t$ because the time intervals in between jumps are memoryless.
This quantity has the following properties:
\begin{lemma}\label{lemma:SimpleFluidDiscountLogLinear}
	\[D(A)D(B)=D(A+B).\]
	\begin{proof}
		\begin{equation}
		\begin{split}
		D(A)D(B)&=\left[\sum\limits_{k=0}^\infty e^{-\beta(A+kJ)}\mathbb{P}(N_t(A)=k)\right]\left[\sum\limits_{m=0}^\infty e^{-\beta(B+mJ)}\mathbb{P}(N_t(B)=m)\right]\\
		&=\sum\limits_{n=0}^\infty e^{-\beta(A+B+nJ)}\sum\limits_{k=0}^n \mathbb{P}(N_t(A)=k)\mathbb{P}(N_t(B)=n-k)\\
		&=\sum\limits_{n=0}^\infty e^{-\beta(A+B+nJ)} \mathbb{P}(N_t(A+B)=n).\\
		\end{split}
		\end{equation}
		The last step holds since on the second last line, the second sum equals the probability that $n$ jumps occur before a fluid quantity $A+B$ is drained, conditioned on the number of jumps that occur before a quantity $A$ is drained.
	\end{proof}
\end{lemma}

\begin{lemma}\label{lemma:SimpleFluidDiscountODE}
	$$\frac{d}{dA}D(A)=-(\beta+\lambda)D(A)+\lambda D(A+J)$$
	\begin{proof}
		This can be seen from taking the derivative of \eqref{eq:SimpleFluidDiscountDefinition}.
	\end{proof}
\end{lemma}
Using these lemmas we can find a simpler expression for $D(q)$ than \eqref{eq:SimpleFluidDiscountDefinition}:
\begin{theorem}\label{theorem:SimpleFluidDiscountExpression}
	The expected discount over the time until the fluid has decreased by $q$, is given by
	\[
	D(A)=e^{-(\beta+\lambda(1-D(J)))A}.
	\]
	\begin{proof}
		Using lemma \ref{lemma:SimpleFluidDiscountLogLinear}, we can rewrite the derivative of $D(A)$ given by \ref{lemma:SimpleFluidDiscountODE} to
		\[
		\frac{d}{dA}D(A)=-(\beta+\lambda)D(A)+\lambda D(A)D(J)=-(\beta+\lambda(1-D(J)))D(A).
		\]
		Which leads to solution
		\[
		D(A)=Ce^{-(\beta+\lambda(1-D(J)))A}.
		\]
		For some integration constant $C$. Since $D(0)=1$, we know that $C=1$, which completes the proof.
	\end{proof}
\end{theorem}
\begin{corollary}
	The value $D(J)$ is now implicitly given by
	\[
	D(J)=e^{-(\beta+\lambda(1-D(J)))J}.
	\]
	This quantity can be approximated by a method of successive approximation.
	For the parameters that were used in appendix \ref{AppendixComputationsTable}, this quantity converged within ten iterations up to five decimals.
\end{corollary}
\begin{remark}\label{remark:SimpleFluidDiscountEquivalence}
	From theorem \ref{theorem:SimpleFluidDiscountExpression}, it can be seen that this expected discount factor of the simple fluid model is actually the same as the discount factor for the age-based model with adjusted discount exponent $\beta^*=\beta+\lambda(1-D(J))$.
	That is,
	\[
	D(q)=e^{-\beta^*q}.
	\]
\end{remark}

Now we will derive the expected total discounted cost:
If $T_0(Q_0)\leq T_0(\mu)$, the asset will break in the first run and expected value at which the corrective repair cost is discounted is $\mathbb{E}[D(Q_0)|T_0(Q_0)\leq T_0(\mu)]$.
Note that $\mathbb{P}(T_0(Q_0)\leq T_0(\mu))=\mathbb{P}(Q_0\leq\mu)$ by lemma \ref{lemma:TimeUntilFluidLemma}.
If $T_0(Q_0)> T_0(\mu)$, the expected value at which the preventive repair cost is discounted is $D(\mu)$.
This results in the following expression for the expected total discounted cost:
\begin{theorem}
	The expected total discounted cost of an asset with control limit $\mu$, is given by
	\begin{equation}\label{eq:SimpleFluidImplicitTDC}
	V(x_{NEW},\mu)=F(\mu)\mathbb{E}[D(Q_0)|Q_0\leq \mu]a+\mathbb{E}[D(Q_0\wedge\mu)](c+V(x_{NEW},\mu)).
	\end{equation}
\end{theorem}

\begin{corollary}
	Alternatively \eqref{eq:SimpleFluidImplicitTDC} can be written as
	\begin{equation}\label{eq:SimpleFluidExplicitTDC}
	V(x_{NEW},\mu)=\frac{F(\mu)\mathbb{E}[D(Q_0)|Q_0\leq \mu]a+\mathbb{E}[D(Q_0\wedge\mu)]c}{1-\mathbb{E}[D(Q_0\wedge\mu)]}.
	\end{equation}
\end{corollary}

\begin{remark}\label{remark:SimpleFluidTDCEquivalence}
	Using remark \ref{remark:SimpleFluidDiscountEquivalence}, we can see that \eqref{eq:SimpleFluidImplicitTDC} is again the same as for the age-based maintenance problem \eqref{eq:AgeBasedHatTDC} for the adjusted discount factor $\beta^*=\beta+\lambda(1-D(J))$.
	%Hence, the same methods can be used to find the optimal policy and the corresponding total discounted cost.
\end{remark}