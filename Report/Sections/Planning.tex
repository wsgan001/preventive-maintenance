\section{Planning}
I am planning to focus my research on finding a preventive maintenance policy based on fluid models. This can be divided into two parts:
\begin{itemize}
\item Estimating the parameters of the fluid model. Possible simplifications to start with are fluid models without jumps or fluid models with jumps but with all rates equal to $-1$.
\item Basing a maintenance policy on this fluid model.
\end{itemize}

\subsection{Tasks}
Below are a few things that might be useful to do in order to find preventive maintenance policies for the data. I have ordered them in the order that seems most useful and feasible to me.
\begin{enumerate}
\item Plot density of time-to-live for each state and gather averages and other statistics.
\item Find (candidate) jump states. This could for instance be done by comparing the average time-to-live of a state with the time-to-live from each incoming state, if it is higher than all incoming states, it is a candidate jump state.
\item Find (distribution of) jump quantities. This could be done based on the difference between subsequent time-to-live distributions.
\item Improve clustering algorithm such that it minimizes cut values.
\item Prove the optimality of the CTMC policy that resulted from the value iteration. I hope to find some argument based on the fact that all repair states are in the same cluster.
\item Find rates for each state. A while ago, I came up with some estimator that could be used, but it requires some distribution of the fluid quantity.
\item Find method for deriving policies for fluid models without jumps. I've managed to reduce the cost equations for when no preventive maintenance is scheduled to a nonhomogeneous matrix differential equation which should be feasible to solve. Currently, the main obstacle seems to be finding initial conditions for this differential equation.
\item Find method for deriving policies for fluid models with jumps but with rates equal to $-1$.
\item Try combining the above two methods to find a method to derive policies for fluid models with jumps and different rates.
\end{enumerate}

\subsection{Timeline}
I will try to do the first two tasks in week 32 and 33 (as I don't have much time in these weeks). In week 34, I will have more time and will work on task 3, 4 and 5. In week 35 I will work on task 6 and 7. After that, the academic year has started again and we can meet and plan the rest of the project. \\
\\
This planning is of course a draft, it will likely be changed when obstacles are met.