\section{Computation of discounted probabilities}
In this section, we will show how to compute the discounted probabilities $D_{ij}(q,\pi,l)$ and $D_{i}(q,\pi,l)$ and the discounted density $\Gamma_i(q,\pi,l)$.
We will do this, by first deriving $D_{ij}(q)$.

\begin{remark}
Note that
\[
D_i^t(q)=\mathbb{E}[e^{-\beta T_t(q)}|S(t)=i]
\]	
equals the value of the moment generating function of $T_t(q)$ (conditioned on $S(t)=i$) evaluated at $-\beta$.
Furthermore, when all jump quantities are 0 (i.e. $J_{ij}=0$ for all $i,j$), $T_t(q)$ corresponds to the busy period of the fluid flow process considered in \cite{Asmussen1994}. 
\cite{Asmussen1994} derives the moment generating function of this busy period, which also corresponds to our results.
\end{remark}

\subsection{Disregarding failures and policies}
We repeat the definition of $D_{ij}(q)$:
\[
D_{ij}^t(q):=\mathbb{E}[e^{-\beta T_t(q)}\mathds{1}\{S(t+T_t(q))=j\}|S(t)=i].
\]
We will now prove a few properties regarding $D_{ij}(q)$:
\begin{lemma}\label{lemma:MmfmDiscountsExponentLinear}
	\[
	D_{ij}(A+B)=\sum\limits_k D_{ik}(A)D_{kj}(B)
	\]
	\begin{proof}
		At the time when the fluid level has decreased by $A$, the process must be in some CTMC-state $s_k$.
		Furthermore, the time until the fluid is decreased by $A$ is independent of the time until the fluid is decreased by $A$ because of the Markov property.
	\end{proof}
\end{lemma}
\begin{lemma}\label{lemma:MmfmDiscountsInfinitisimal}
	For small $\delta$, $D_{ij}(\delta r_i)$ is given by
	\[
	D_{ij}(\delta r_i)=(1-\delta\lambda_i-\delta\beta)\mathds{1}\{i=j\}+\sum\limits_{k\neq i}\delta\lambda_{ik}D_{kj}(J_{ik})+o(\delta^2).
	\]
	\begin{proof}
		In a time period of length $\delta$, either a transition occurs to some state $s_k$ ($k\neq i$) or no transition occurs.
		These have probabilities $\delta\lambda_{ik}+o(\delta^2)$ and $1-\delta\lambda_i+o(\delta^2)$ respectively.
		When a transition from $s_i$ to $s_k$ occurs, the fluid level increases by $J_{ik}$.
		Over this time interval, the discount factor is $1-\delta\beta$.
		Furthermore, $D_{ij}(0)=\mathds{1}\{i=j\}$.
		Putting these together results in
		\[
		\begin{split}
		D_{ij}(\delta r_i)&=(1-\delta\lambda_i)(1-\delta\beta)\mathds{1}\{i=j\}+\sum\limits_{k\neq i}\delta(1-\delta\beta)\lambda_{ik}D_{kj}(J_{ik})+o(\delta^2)\\
		&=(1-\delta\lambda_i-\delta\beta)\mathds{1}\{i=j\}+\sum\limits_{k\neq i}\delta\lambda_{ik}D_{kj}(J_{ik})+o(\delta^2).
		\end{split}
		\]
	\end{proof}
\end{lemma}
\begin{lemma}
	$D_{ij}$ adheres to the following differential equation:
	\begin{equation}\label{eq:MmfmDiscountDifferential}
	r_i\frac{d}{dq}D_{ij}(q)=\sum\limits_m\left[\sum\limits_{k\neq i}\lambda_{ik}D_{km}(J_{ik})\right] D_{mj}(q)-(\lambda_i+\beta)D_{ij}(q)
	\end{equation}
	\begin{proof}
		First we write
		\[
		\begin{split}
		D_{ij}(q+\delta r_i)&=\sum_mD_{im}(\delta r_i)D_{mj}(q)\\
		&=\sum\limits_m\left[(1-\delta\lambda_i-\delta\beta)\mathds{1}\{i=j\}+\sum\limits_{k\neq i}\delta\lambda_{ik}D_{kj}(J_{ik})+o(\delta^2)\right]D_{mj}(q)\\
		&=(1-\delta\lambda_i-\delta\beta)D_{ij}(q)+\sum\limits_m\left[\sum\limits_{k\neq i}\delta\lambda_{ij}D_{kj}(J_{ik})\right]D_{mj}(q)+o(\delta^2)\\
		&=D_{ij}(q)+\delta\left( \sum\limits_m\left[\sum\limits_{k\neq i}\lambda_{ik}D_{km}(J_{ik})\right] D_{mj}(q)-(\lambda_i+\beta)D_{ij}(q) \right)+o(\delta^2).\\
		\end{split}
		\]
		If we then subtract $D_{ij}(q)$ from both sides, divide by $\delta$ and let $\delta\rightarrow 0$, we get
		\[
		\begin{split}
		\lim\limits_{\delta\rightarrow 0}\frac{D_{ij}(q+\delta r_i)-D_{ij}(q)}{\delta}&=r_i\frac{d}{dq}D_{ij}(q)\\
		&=\sum\limits_m\left[\sum\limits_{k\neq i}\lambda_{ik}D_{km}(J_{ik})\right] D_{mj}(q)-(\lambda_i+\beta)D_{ij}(q).
		\end{split}
		\]
	\end{proof}
\end{lemma}
Hence, the derivative of $D_{ij}$ is a linear combination of $D_{kj}$.
This suggests defining the following matrix:
\begin{equation}\label{eq:MmfmDiscountGenerator}
\begin{split}
\Lambda^D_{im}:=\begin{cases}
\sum\limits_{k\neq i}\frac{\lambda_{ik}}{r_i}D_{km}(J_{ik})-\frac{(\lambda_i+\beta)}{r_i}&\text{ if }i=m\\
\sum\limits_{k\neq i}\frac{\lambda_{ik}}{r_i}D_{km}(J_{ik})&\text{ else.}
\end{cases}
\end{split}
\end{equation}
Furthermore, if we let $D(q)$ be the matrix with entries $D_{ij}(q)$, we can solve the differential equation \eqref{eq:MmfmDiscountDifferential} in the following way:
\begin{theorem}
	For $\Lambda^D$ as defined above, the solution to differential equation \eqref{eq:MmfmDiscountDifferential} is given by
	\begin{equation}\label{eq:MmfmDiscountMatrixSolution}
	D(q)=e^{\Lambda^D q}.
	\end{equation}
	So that the discounted probability of going from state $i$ to $j$ while $Q(t)$ decreases by $q$, is given by
	\begin{equation}\label{eq:MmfmDiscountEntrySolution}
	D_{ij}(q)=\left(e^{\Lambda^D q}\right)_{ij}.
	\end{equation}
	\begin{proof}
		The differential equation \eqref{eq:MmfmDiscountDifferential} can be rewritten to
		\[
			\frac{d}{dq}D_{ij}(q)=\sum\limits_m\Lambda^D_{im}D_{mj}(q).
		\]
		So that for the matrix $D(q)$, we have the following matrix differential equation
		\[
		\frac{d}{dq}D(q)=\Lambda^D D(q),
		\]
		of which \eqref{eq:MmfmDiscountMatrixSolution} is a solution.
		This combined with $D(0)=I$ completes the proof.
	\end{proof}
\end{theorem}

\begin{remark}
	To compute \eqref{eq:MmfmDiscountMatrixSolution}, we still need the constants $D_{km}(J_{ik})$.
	These $N^3$ values can be estimated using a method of successive approximation where iteratively these values $D_{km}(J_{ik})$ are calculated using \eqref{eq:MmfmDiscountMatrixSolution}.
	For the problem parameters that we used, ten iterations were enough to make these values converge for up to five decimals.
\end{remark}

\begin{remark}\label{remark:MmfmStochasticShortestPath}
	As $L_0(t)$ increases continuously when $L_c(t)=0$ and is constant when $L_c(t)>0$, we know that in each run of the asset for each value $l_0\geq0$, there exists a $t$ so that $L_0(t)=l_0$.
	If we omit all time intervals where $L_c(t)>0$, we can view the asset as a CTMC over $L_0$.
	That is, for the first run of the asset, we consider a CTMC $S^*(l_0)$ where
	\[
	S^*(l_0):=S(\min\{\tau|L_c(\tau)=0\text{ and } L_0(\tau)=l_0\}).
	\]
	When we would start in a state $s_i$ and pay a certain cost when $X(t)=(i,l_0,0)$, the expected value at which this cost would be discounted, would correspond to \eqref{eq:MmfmDiscountEntrySolution}.
	Transition probabilities of $S^*$ are given by generator $e^{\Lambda^D}$ for $\beta=0$.
	Viewing the process as this adjusted CTMC helps understanding the problem.
\end{remark}

\subsection{Taking policies into account}
When we take policies into account, the following complication arises in computing $D_{ij}(q, \pi,l)$:
In the path from $s_i$ to $s_j$, no state $s_k$ must be visited when $L_0(t)>\mu_k$.
This is summarized by the following lemma.
\begin{lemma}
	Similar to lemma \ref{lemma:MmfmDiscountsExponentLinear}, we have
	\[
	D_{ij}(A+B,\pi,l)=\sum\limits_{\mu_k>l+A} D_{ik}(A,\pi,l)D_{kj}(B,\pi,l+A)
	\]
	\begin{proof}
		The reasoning is the same as for lemma \ref{lemma:MmfmDiscountsExponentLinear}, but with the addition that we also need to keep into account that $L_0$ has been increased by $A$, this explains the $l+A$ on the right hand side.
	\end{proof}
\end{lemma}

\begin{lemma}
	For small $\delta$, we have
	\[
	D_{ij}(r_i\delta,\pi,l)=(1-\delta\lambda_i)\mathds{1}\{i=j\}+\sum\limits_{\mu_k>l+r_i\delta}\delta\lambda_{ij}D_{kj}(J_{ik})+o(\delta^2).
	\]
	\begin{proof}
		The reasoning is the same as in lemma \ref{lemma:MmfmDiscountsInfinitisimal}, but now we know that for $\mu_k\leq l+\delta r_i$, we have that $D_{kj}(J_{ik},\pi,l+\delta r_i)=0$.
	\end{proof}
\end{lemma}
Which suggests that we should replace the generator $\Lambda^D$ by a $\Lambda^D(l_0)$ dependent of the amount of used fluid $l_0$:
\begin{equation}
\begin{split}
\Lambda^D_{im}(l_0,\pi):=\begin{cases}
0&\text{ if }\mu_i<l_0\\
\Lambda^D_{im}&\text{ else.}
\end{cases}
\end{split}
\end{equation}
$D_{ij}(q, \pi,l)$ can now be calculated in the following straightforward way:
\begin{theorem}
	The discounted probabilities $D_{ij}(q, \pi,l)$ are given by
	\[
	D_{ij}(q, \pi,l)=\left(e^{\int_l^{l+q}\Lambda^D(x,\pi)dx}\right)_{ij}.
	\]
\end{theorem}
Now we will calculate the discounted density $\Gamma_i(q,\pi,l)$:
$\Gamma_i^t(q,\pi,l)$ corresponds to repairing when the fluid level $Q(t)$ reaches $Q(t)-q$.
The discounted probability of reaching fluid level $Q(t)-q$ in state $s_j$ equals $D_{ij}(q, \pi,l)$.
When the process reaches this state, the asset can be repaired by transitioning to a CTMC-state $s_k$ where the control limit has already been exceeded.
However, the presence of jumps complicates this:
If the transition from $s_j$ to $s_k$ has a fluid jump, then repair won't be chosen immediately.
This problem is solved by using transition rates $\Lambda^D_kj$ instead of $\lambda_{kj}$ since, referring back to remark \ref{remark:MmfmStochasticShortestPath}, are not interested in time intervals where the buffer $L_c$ is nonempty.
Concluding:

\begin{theorem}
	The discounted density corresponding to repairing when $L_0(t)=l+q$ given that initially the process is in CTMC-state $s_i$ with drained initial fluid $l$ is given by
	\[
	\Gamma_i(q,\pi,l)=\sum\limits_{\mu_j>l+q}D_{ij}(q,\pi,l)\sum\limits_{\mu_k<l+q}\Lambda_{jk}^D.
	\]
\end{theorem}


%\begin{remark}
%	Referring back to remark \ref{remark:MmfmStochasticShortestPath}, if we view the process as this adjusted CTMC, the problem simplifies to a stochastic shortest path problem where in each state $s_i$, there is a transition to a terminating state with rate
%	\[
%	-\sum\limits_j \Lambda^D_{ij},
%	\]
%	and a terminating cost of 0.
%\end{remark}