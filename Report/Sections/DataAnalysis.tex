\chapter{Data analysis}\label{chapter:DataAnalysis}
In this chapter, the data from the Philips machine will be investigated.
First, the data will be described and visualized, then we will attempt to fit the lifetimes of the machine to various lifetime distributions.
\section{Data description}
The data from the Philips machine contains information about which operation the machine was performing at each time.
The data is anonimized so that for each operation, no name or description is given, but only an identifier.
\subsection{Data format}
Each run of the machine is represented by a trace.
A trace is a sequence of events.
These events are either the start or the end of an operation.
Each event has a timestamp and an integer representing the identifier of the operation.
The breakdown of the machine is represented by the end of a trace.
The lifetime of the machine is then the length of the time interval between the start of the first event and the end of the last event.

\subsection{Cleaning}
Before the data could be used, it first needed to be cleaned.
Because of the transitions between summer and winter time, a few events ended before they started.
We resolved this by simply ignoring the traces for which there was such an event.
Furthermore, there were some other events with a time length of $-1$ seconds, these traces were also ignored.

\subsection{Visualization}
We will now visualize the distribution of the lengths of the runs of the machine.
In figure \ref{figure:distribution}, the empirical cumulative distribution function and the probability density function are plotted.
In these plots it is visible that the distribution has its mode around five days and has a part with a less steep downward slope around 6.5 days.
This part on the right hand side of the mode could be caused by intermediate repairs.
\begin{figure}[H]
	\setlength\fwidth{0.4\textwidth}
	% This file was created by matlab2tikz.
%
%The latest updates can be retrieved from
%  http://www.mathworks.com/matlabcentral/fileexchange/22022-matlab2tikz-matlab2tikz
%where you can also make suggestions and rate matlab2tikz.
%
\definecolor{mycolor1}{rgb}{0.00000,0.44700,0.74100}%
%
\begin{tikzpicture}

\begin{axis}[%
width=0.951\fwidth,
height=0.75\fwidth,
at={(0\fwidth,0\fwidth)},
scale only axis,
xmin=3,
xmax=9,
xlabel style={font=\color{white!15!black}},
xlabel={Lifetime (d)},
ymin=0,
ymax=1,
ylabel style={font=\color{white!15!black}},
ylabel={Probability},
axis background/.style={fill=white}
]
\addplot [color=mycolor1, forget plot]
  table[row sep=crcr]{%
3.65539351851852	0\\
3.65539351851852	0.00431034482758619\\
3.88612268518519	0.00862068965517238\\
4.04607638888889	0.0129310344827586\\
4.05780092592593	0.0172413793103448\\
4.05827546296296	0.0215517241379309\\
4.06336805555556	0.0258620689655171\\
4.16209490740741	0.0301724137931033\\
4.16583333333333	0.0344827586206895\\
4.16824074074074	0.0387931034482757\\
4.17811342592593	0.0431034482758619\\
4.23090277777778	0.0474137931034481\\
4.23425925925926	0.0517241379310343\\
4.34474537037037	0.0560344827586204\\
4.34844907407407	0.0603448275862066\\
4.35103009259259	0.0646551724137928\\
4.39923611111111	0.068965517241379\\
4.40034722222222	0.0732758620689652\\
4.40402777777778	0.0775862068965514\\
4.45054398148148	0.0818965517241376\\
4.45975694444444	0.0862068965517238\\
4.46435185185185	0.0905172413793101\\
4.46532407407407	0.0948275862068964\\
4.51805555555556	0.0991379310344825\\
4.51966435185185	0.103448275862069\\
4.52009259259259	0.107758620689655\\
4.52246527777778	0.112068965517241\\
4.52336805555556	0.116379310344827\\
4.52395833333333	0.120689655172413\\
4.57143518518519	0.125\\
4.57194444444444	0.129310344827586\\
4.57393518518519	0.133620689655172\\
4.5762962962963	0.137931034482758\\
4.57699074074074	0.142241379310344\\
4.58032407407407	0.146551724137931\\
4.58097222222222	0.150862068965517\\
4.58162037037037	0.155172413793103\\
4.61822916666667	0.159482758620689\\
4.62917824074074	0.163793103448275\\
4.62978009259259	0.168103448275862\\
4.63076388888889	0.172413793103448\\
4.63206018518519	0.176724137931034\\
4.63226851851852	0.18103448275862\\
4.6349537037037	0.185344827586206\\
4.63815972222222	0.189655172413793\\
4.68565972222222	0.193965517241379\\
4.6862962962963	0.198275862068965\\
4.6946412037037	0.202586206896551\\
4.69556712962963	0.206896551724137\\
4.74951388888889	0.211206896551724\\
4.75083333333333	0.21551724137931\\
4.75155092592593	0.219827586206896\\
4.80350694444444	0.224137931034482\\
4.80724537037037	0.228448275862068\\
4.81015046296296	0.232758620689655\\
4.81168981481482	0.237068965517241\\
4.8565162037037	0.241379310344827\\
4.85902777777778	0.245689655172413\\
4.86289351851852	0.249999999999999\\
4.86568287037037	0.254310344827586\\
4.86581018518519	0.258620689655172\\
4.86608796296296	0.262931034482758\\
4.86652777777778	0.267241379310344\\
4.8666087962963	0.271551724137931\\
4.86809027777778	0.275862068965517\\
4.86835648148148	0.280172413793103\\
4.86905092592593	0.284482758620689\\
4.869375	0.288793103448275\\
4.91210648148148	0.293103448275862\\
4.9234837962963	0.297413793103448\\
4.92386574074074	0.301724137931034\\
4.92396990740741	0.30603448275862\\
4.9243287037037	0.310344827586206\\
4.9250462962963	0.314655172413793\\
4.92568287037037	0.318965517241379\\
4.92667824074074	0.323275862068965\\
4.97186342592593	0.327586206896551\\
4.98336805555556	0.331896551724137\\
4.98383101851852	0.336206896551724\\
5.02824074074074	0.34051724137931\\
5.03604166666667	0.344827586206896\\
5.03638888888889	0.349137931034482\\
5.0368287037037	0.353448275862068\\
5.03813657407407	0.357758620689654\\
5.03834490740741	0.362068965517241\\
5.03850694444444	0.366379310344827\\
5.03966435185185	0.370689655172413\\
5.03972222222222	0.374999999999999\\
5.03991898148148	0.379310344827585\\
5.04091435185185	0.383620689655172\\
5.04208333333333	0.387931034482758\\
5.04222222222222	0.392241379310344\\
5.08837962962963	0.39655172413793\\
5.09479166666667	0.400862068965516\\
5.09842592592593	0.405172413793103\\
5.0990162037037	0.409482758620689\\
5.09925925925926	0.413793103448275\\
5.14765046296296	0.418103448275861\\
5.15236111111111	0.422413793103447\\
5.15459490740741	0.426724137931033\\
5.1550462962963	0.43103448275862\\
5.15631944444444	0.435344827586206\\
5.20888888888889	0.439655172413792\\
5.21149305555556	0.443965517241378\\
5.21186342592593	0.448275862068965\\
5.21314814814815	0.452586206896551\\
5.21327546296296	0.456896551724137\\
5.21475694444444	0.461206896551723\\
5.26159722222222	0.465517241379309\\
5.27122685185185	0.469827586206895\\
5.27228009259259	0.474137931034482\\
5.27280092592593	0.478448275862068\\
5.31628472222222	0.482758620689654\\
5.32148148148148	0.48706896551724\\
5.32450231481482	0.491379310344826\\
5.32523148148148	0.495689655172413\\
5.3290625	0.499999999999999\\
5.32938657407407	0.504310344827585\\
5.32986111111111	0.508620689655171\\
5.33034722222222	0.512931034482757\\
5.37268518518519	0.517241379310344\\
5.38502314814815	0.52155172413793\\
5.38523148148148	0.525862068965516\\
5.38607638888889	0.530172413793102\\
5.38643518518519	0.534482758620688\\
5.38741898148148	0.538793103448275\\
5.38778935185185	0.543103448275861\\
5.42561342592593	0.547413793103447\\
5.43828703703704	0.551724137931033\\
5.45168981481481	0.55603448275862\\
5.49454861111111	0.560344827586206\\
5.49575231481481	0.564655172413792\\
5.498125	0.568965517241378\\
5.55136574074074	0.573275862068964\\
5.55638888888889	0.577586206896551\\
5.55946759259259	0.581896551724137\\
5.56048611111111	0.586206896551723\\
5.60539351851852	0.590517241379309\\
5.61043981481482	0.594827586206895\\
5.61315972222222	0.599137931034482\\
5.61460648148148	0.603448275862068\\
5.61685185185185	0.607758620689654\\
5.61778935185185	0.61206896551724\\
5.63293981481482	0.616379310344827\\
5.67181712962963	0.620689655172413\\
5.67417824074074	0.624999999999999\\
5.67565972222222	0.629310344827585\\
5.67583333333333	0.633620689655171\\
5.72508101851852	0.637931034482758\\
5.72584490740741	0.642241379310344\\
5.72984953703704	0.64655172413793\\
5.73196759259259	0.650862068965516\\
5.73237268518518	0.655172413793103\\
5.7333912037037	0.659482758620689\\
5.78729166666667	0.663793103448275\\
5.84063657407407	0.668103448275861\\
5.84469907407407	0.672413793103447\\
5.84494212962963	0.676724137931034\\
5.84501157407407	0.68103448275862\\
5.84572916666667	0.685344827586206\\
5.84731481481481	0.689655172413792\\
5.84744212962963	0.693965517241378\\
5.84780092592593	0.698275862068965\\
5.84868055555556	0.702586206896551\\
5.90346064814815	0.706896551724137\\
5.90456018518519	0.711206896551723\\
5.90581018518519	0.715517241379309\\
5.95017361111111	0.719827586206896\\
5.95128472222222	0.724137931034482\\
5.95733796296296	0.728448275862068\\
5.95859953703704	0.732758620689654\\
5.99944444444444	0.73706896551724\\
6.01927083333333	0.741379310344827\\
6.02068287037037	0.745689655172413\\
6.07572916666667	0.749999999999999\\
6.07767361111111	0.754310344827585\\
6.12369212962963	0.758620689655172\\
6.13309027777778	0.762931034482758\\
6.13498842592593	0.767241379310344\\
6.13619212962963	0.77155172413793\\
6.17320601851852	0.775862068965516\\
6.18494212962963	0.780172413793103\\
6.24876157407407	0.784482758620689\\
6.25163194444444	0.788793103448275\\
6.25172453703704	0.793103448275861\\
6.3034837962963	0.797413793103448\\
6.30511574074074	0.801724137931034\\
6.30893518518519	0.80603448275862\\
6.30917824074074	0.810344827586206\\
6.36299768518519	0.814655172413792\\
6.36516203703704	0.818965517241379\\
6.36681712962963	0.823275862068965\\
6.37883101851852	0.827586206896551\\
6.42065972222222	0.831896551724137\\
6.42096064814815	0.836206896551724\\
6.47592592592593	0.84051724137931\\
6.5343287037037	0.844827586206896\\
6.5375462962963	0.849137931034482\\
6.53884259259259	0.853448275862068\\
6.59601851851852	0.857758620689655\\
6.64996527777778	0.862068965517241\\
6.65070601851852	0.866379310344827\\
6.65076388888889	0.870689655172413\\
6.69883101851852	0.875\\
6.70283564814815	0.879310344827586\\
6.70502314814815	0.883620689655172\\
6.70605324074074	0.887931034482758\\
6.71027777777778	0.892241379310344\\
6.71064814814815	0.896551724137931\\
6.71068287037037	0.900862068965517\\
6.71171296296296	0.905172413793103\\
6.71193287037037	0.909482758620689\\
6.73209490740741	0.913793103448275\\
6.75421296296296	0.918103448275862\\
6.76596064814815	0.922413793103448\\
6.81811342592593	0.926724137931034\\
6.88480324074074	0.93103448275862\\
6.94032407407407	0.935344827586207\\
6.94290509259259	0.939655172413793\\
6.98989583333333	0.943965517241379\\
7.05606481481481	0.948275862068965\\
7.15710648148148	0.952586206896551\\
7.16491898148148	0.956896551724138\\
7.28027777777778	0.961206896551724\\
7.28799768518518	0.96551724137931\\
7.28805555555556	0.969827586206896\\
7.4034375	0.974137931034483\\
7.45590277777778	0.978448275862069\\
7.69079861111111	0.982758620689655\\
7.74900462962963	0.987068965517241\\
8.03364583333333	0.991379310344828\\
8.08883101851852	0.995689655172414\\
8.84060185185185	1\\
};
\end{axis}
\end{tikzpicture}%
	\setlength\fwidth{0.4\textwidth}
	% This file was created by matlab2tikz.
%
%The latest updates can be retrieved from
%  http://www.mathworks.com/matlabcentral/fileexchange/22022-matlab2tikz-matlab2tikz
%where you can also make suggestions and rate matlab2tikz.
%
\definecolor{mycolor1}{rgb}{0.00000,0.44700,0.74100}%
%
\begin{tikzpicture}

\begin{axis}[%
width=0.951\fwidth,
height=0.75\fwidth,
at={(0\fwidth,0\fwidth)},
scale only axis,
xmin=3,
xmax=9,
xlabel style={font=\color{white!15!black}},
xlabel={Lifetime (d)},
ymin=0,
ymax=0.5,
ylabel style={font=\color{white!15!black}},
ylabel={Probability density},
axis background/.style={fill=white}
]
\addplot [color=mycolor1, forget plot]
  table[row sep=crcr]{%
3.65539351851852	0.0311060613415889\\
3.65539351851852	0.0311060613415889\\
3.88612268518519	0.0746253089773507\\
4.04607638888889	0.123071799241282\\
4.05780092592593	0.127260318602443\\
4.05827546296296	0.127431461243989\\
4.06336805555556	0.129276898782796\\
4.16209490740741	0.168146163176176\\
4.16583333333333	0.169734343907335\\
4.16824074074074	0.170759418377405\\
4.17811342592593	0.174982656334577\\
4.23090277777778	0.198382132389908\\
4.23425925925926	0.199912541349046\\
4.34474537037037	0.252402378945583\\
4.34844907407407	0.254211073003693\\
4.35103009259259	0.255472496776924\\
4.39923611111111	0.279123202489396\\
4.40034722222222	0.279668462445415\\
4.40402777777778	0.281474209298978\\
4.45054398148148	0.304175146339895\\
4.45975694444444	0.308631239966174\\
4.46435185185185	0.310843709669417\\
4.46532407407407	0.311311134444933\\
4.51805555555556	0.336210692488908\\
4.51966435185185	0.336953134075069\\
4.52009259259259	0.337150565141363\\
4.52246527777778	0.338242920655672\\
4.52336805555556	0.338657866443379\\
4.52395833333333	0.33892897192696\\
4.57143518518519	0.360142241614023\\
4.57194444444444	0.360362785263918\\
4.57393518518519	0.361223353467588\\
4.5762962962963	0.36224078682367\\
4.57699074074074	0.362539357159603\\
4.58032407407407	0.363968180905486\\
4.58097222222222	0.36424517235094\\
4.58162037037037	0.364521889805077\\
4.61822916666667	0.37968063473534\\
4.62917824074074	0.384039203306384\\
4.62978009259259	0.384275265977709\\
4.63076388888889	0.384660519508577\\
4.63206018518519	0.385166972627279\\
4.63226851851852	0.385248241904796\\
4.6349537037037	0.386292600308746\\
4.63815972222222	0.387531910989408\\
4.68565972222222	0.404871800094251\\
4.6862962962963	0.405090451855561\\
4.6946412037037	0.407921643805695\\
4.69556712962963	0.408231729408636\\
4.74951388888889	0.424844467361337\\
4.75083333333333	0.425213435099512\\
4.75155092592593	0.425413331542827\\
4.80350694444444	0.438414238138172\\
4.80724537037037	0.439234718987131\\
4.81015046296296	0.439861458694081\\
4.81168981481482	0.440189701985337\\
4.8565162037037	0.448567522944882\\
4.85902777777778	0.448968572866692\\
4.86289351851852	0.449571613617826\\
4.86568287037037	0.449996009578901\\
4.86581018518519	0.450015165572552\\
4.86608796296296	0.450056895372192\\
4.86652777777778	0.450122784969028\\
4.8666087962963	0.450134898113357\\
4.86809027777778	0.450355056247529\\
4.86835648148148	0.450394346698338\\
4.86905092592593	0.450496457452213\\
4.869375	0.450543918105183\\
4.91210648148148	0.455739643129222\\
4.9234837962963	0.456771204537073\\
4.92386574074074	0.456803160732379\\
4.92396990740741	0.456811847081443\\
4.9243287037037	0.456841671685796\\
4.9250462962963	0.45690087904012\\
4.92568287037037	0.456952908724559\\
4.92667824074074	0.457033335688471\\
4.97186342592593	0.459507417443618\\
4.98336805555556	0.459774481144679\\
4.98383101851852	0.459782212039674\\
5.02824074074074	0.459469506883455\\
5.03604166666667	0.459207876861199\\
5.03638888888889	0.45919470071338\\
5.0368287037037	0.459177839698741\\
5.03813657407407	0.459126571373713\\
5.03834490740741	0.459118248935525\\
5.03850694444444	0.459111746362447\\
5.03966435185185	0.459064547899184\\
5.03972222222222	0.459062153401258\\
5.03991898148148	0.459053987503064\\
5.04091435185185	0.45901209543166\\
5.04208333333333	0.458961657305564\\
5.04222222222222	0.458955575831265\\
5.08837962962963	0.455935004692026\\
5.09479166666667	0.455363516501769\\
5.09842592592593	0.455024089911181\\
5.0990162037037	0.454967912461828\\
5.09925925925926	0.454944695887363\\
5.14765046296296	0.449397553350645\\
5.15236111111111	0.448765269795688\\
5.15459490740741	0.44846015201853\\
5.1550462962963	0.448398087517995\\
5.15631944444444	0.448222298152631\\
5.20888888888889	0.440100784291771\\
5.21149305555556	0.439658562703536\\
5.21186342592593	0.439595396311849\\
5.21314814814815	0.439375766414825\\
5.21327546296296	0.439353957369841\\
5.21475694444444	0.439099600736633\\
5.26159722222222	0.430566129820649\\
5.27122685185185	0.428707767851718\\
5.27228009259259	0.428502943070104\\
5.27280092592593	0.428401530144731\\
5.31628472222222	0.419681539505363\\
5.32148148148148	0.418612978213793\\
5.32450231481482	0.417989156139148\\
5.32523148148148	0.41783834921183\\
5.3290625	0.417044604331096\\
5.32938657407407	0.41697735373287\\
5.32986111111111	0.41687885051349\\
5.33034722222222	0.416777909123169\\
5.37268518518519	0.407877823186896\\
5.38502314814815	0.405256123744893\\
5.38523148148148	0.405211817188985\\
5.38607638888889	0.405032112784854\\
5.38643518518519	0.404955792023614\\
5.38741898148148	0.40474461334853\\
5.38778935185185	0.404665822294029\\
5.42561342592593	0.39660651876914\\
5.43828703703704	0.393915792319455\\
5.45168981481481	0.391071603188613\\
5.49454861111111	0.382058207357328\\
5.49575231481481	0.381806687525811\\
5.498125	0.381311221377449\\
5.55136574074074	0.370314503350382\\
5.55638888888889	0.369288266429788\\
5.55946759259259	0.368660229595307\\
5.56048611111111	0.368452614718545\\
5.60539351851852	0.359367260943744\\
5.61043981481482	0.35835466385146\\
5.61315972222222	0.357809431401849\\
5.61460648148148	0.357519565306394\\
5.61685185185185	0.35706989590712\\
5.61778935185185	0.356882218991164\\
5.63293981481482	0.353849716757515\\
5.67181712962963	0.346118508407335\\
5.67417824074074	0.345648109441022\\
5.67565972222222	0.345354180232946\\
5.67583333333333	0.345319736963845\\
5.72508101851852	0.33555056870579\\
5.72584490740741	0.33539896294851\\
5.72984953703704	0.334604028450754\\
5.73196759259259	0.33418347667351\\
5.73237268518518	0.334103034219292\\
5.7333912037037	0.333900765746892\\
5.78729166666667	0.323156903394869\\
5.84063657407407	0.312447894171271\\
5.84469907407407	0.311631133166836\\
5.84494212962963	0.311582258483435\\
5.84501157407407	0.311568294117013\\
5.84572916666667	0.311423991259582\\
5.84731481481481	0.311105100765924\\
5.84744212962963	0.31107949481046\\
5.84780092592593	0.311007331317359\\
5.84868055555556	0.310830406661559\\
5.90346064814815	0.299802129527457\\
5.90456018518519	0.299581380799908\\
5.90581018518519	0.299330464076039\\
5.95017361111111	0.290466596293458\\
5.95128472222222	0.29024626058022\\
5.95733796296296	0.289047601921\\
5.95859953703704	0.288798165750233\\
5.99944444444444	0.280806289857521\\
6.01927083333333	0.277006981640479\\
6.02068287037037	0.276738705901988\\
6.07572916666667	0.266570062085861\\
6.07767361111111	0.266222677158962\\
6.12369212962963	0.25825983008255\\
6.13309027777778	0.256707050641493\\
6.13498842592593	0.256394560169394\\
6.13619212962963	0.256198126939575\\
6.17320601851852	0.250352953324108\\
6.18494212962963	0.248589826283529\\
6.24876157407407	0.239723892180693\\
6.25163194444444	0.239352118852701\\
6.25172453703704	0.2393402276219\\
6.3034837962963	0.233050427279844\\
6.30511574074074	0.232864116000349\\
6.30893518518519	0.232430538301331\\
6.30917824074074	0.232403062995238\\
6.36299768518519	0.226617445316686\\
6.36516203703704	0.226396034748598\\
6.36681712962963	0.226227185658025\\
6.37883101851852	0.225003885183884\\
6.42065972222222	0.220894727144977\\
6.42096064814815	0.22086557571531\\
6.47592592592593	0.215529433667882\\
6.5343287037037	0.209577949937548\\
6.5375462962963	0.209233526838778\\
6.53884259259259	0.209092274835756\\
6.59601851851852	0.202516105094586\\
6.64996527777778	0.19539186592084\\
6.65070601851852	0.195287086321773\\
6.65076388888889	0.195278891889695\\
6.69883101851852	0.188035951460756\\
6.70283564814815	0.187393364968611\\
6.70502314814815	0.187039790829892\\
6.70605324074074	0.186872666029378\\
6.71027777777778	0.186183075356191\\
6.71064814814815	0.186122297266695\\
6.71068287037037	0.186116596672719\\
6.71171296296296	0.185947272643141\\
6.71193287037037	0.185911073109487\\
6.73209490740741	0.182510800128473\\
6.75421296296296	0.178616530074655\\
6.76596064814815	0.176478360351458\\
6.81811342592593	0.166453773871518\\
6.88480324074074	0.152653308857269\\
6.94032407407407	0.140711601082539\\
6.94290509259259	0.140153255561151\\
6.98989583333333	0.130011523070882\\
7.05606481481481	0.116117647440739\\
7.15710648148148	0.0966285800115478\\
7.16491898148148	0.095234596354422\\
7.28027777777778	0.0766719943892029\\
7.28799768518518	0.0755646585877256\\
7.28805555555556	0.0755564189315758\\
7.4034375	0.0608310644793743\\
7.45590277777778	0.0551738275332293\\
7.69079861111111	0.0362658725559196\\
7.74900462962963	0.0329266889235524\\
8.03364583333333	0.0207032976089452\\
8.08883101851852	0.0186753166915754\\
8.84060185185185	0.00608852609872549\\
};
\end{axis}
\end{tikzpicture}%
	\caption{The empirical cumulative distribution function and the probability density function of the lifetime of the machine.}
	\label{figure:distribution} 
\end{figure}

For survival analysis, the hazard rate of the lifetime is important.
The observed hazard rate over time is plotted in figure \ref{figure:hazard}.
As you can see, the hazard rate is increasing for lifetimes shorter than 6.5 days.
For lifetimes larger than 6.5 days, the hazard rate seems to jump up and down a lot.
This is likely because these large lifetimes did not occur frequently enough in the dataset to smoothen out the hazard rate.
Hence, we can safely assume that the lifetime has an increasing hazard rate.
\begin{figure}[H]
	\centering
	\setlength\fwidth{0.5\textwidth}
% This file was created by matlab2tikz.
%
%The latest updates can be retrieved from
%  http://www.mathworks.com/matlabcentral/fileexchange/22022-matlab2tikz-matlab2tikz
%where you can also make suggestions and rate matlab2tikz.
%
\definecolor{mycolor1}{rgb}{0.00000,0.44700,0.74100}%
%
\begin{tikzpicture}

\begin{axis}[%
width=0.951\fwidth,
height=0.75\fwidth,
at={(0\fwidth,0\fwidth)},
scale only axis,
unbounded coords=jump,
xmin=3.5,
xmax=8.5,
xlabel style={font=\color{white!15!black}},
xlabel={Lifetime (d)},
ymin=0,
ymax=4.5,
ylabel style={font=\color{white!15!black}},
ylabel={Hazard rate},
axis background/.style={fill=white}
]
\addplot [color=mycolor1, forget plot]
  table[row sep=crcr]{%
3.65539351851852	0.0311060613415889\\
3.65539351851852	0.031240719615795\\
3.88612268518519	0.0752742247075886\\
4.04607638888889	0.124684093554486\\
4.05780092592593	0.129492955770907\\
4.05827546296296	0.130238321623813\\
4.06336805555556	0.132709028838976\\
4.16209490740741	0.173377377141657\\
4.16583333333333	0.175796284761168\\
4.16824074074074	0.177651054096672\\
4.17811342592593	0.18286475797127\\
4.23090277777778	0.208256356174021\\
4.23425925925926	0.210816861786267\\
4.34474537037037	0.267385168563358\\
4.34844907407407	0.270536554756224\\
4.35103009259259	0.273131885955053\\
4.39923611111111	0.299798995266389\\
4.40034722222222	0.301781782731797\\
4.40402777777778	0.305149610081135\\
4.45054398148148	0.331308140614345\\
4.45975694444444	0.337747394679964\\
4.46435185185185	0.341780761342676\\
4.46532407407407	0.343924681862973\\
4.51805555555556	0.373209955298692\\
4.51966435185185	0.375832341852962\\
4.52009259259259	0.377869232428967\\
4.52246527777778	0.380933774719009\\
4.52336805555556	0.38326158543836\\
4.52395833333333	0.385448634740464\\
4.57143518518519	0.411591133273169\\
4.57194444444444	0.413882010798163\\
4.57393518518519	0.416934417932739\\
4.5762962962963	0.420199312715457\\
4.57699074074074	0.422658949050391\\
4.58032407407407	0.42646776752562\\
4.58097222222222	0.428958781651868\\
4.58162037037037	0.431474889973356\\
4.61822916666667	0.451722601326148\\
4.62917824074074	0.459263377149903\\
4.62978009259259	0.461926744594966\\
4.63076388888889	0.46479812773953\\
4.63206018518519	0.46784679397659\\
4.63226851851852	0.470408379589013\\
4.6349537037037	0.474179276569466\\
4.63815972222222	0.478230868880545\\
4.68565972222222	0.50230084289768\\
4.6862962962963	0.505274111991882\\
4.6946412037037	0.511555791150925\\
4.69556712962963	0.514726963167411\\
4.74951388888889	0.538600636217651\\
4.75083333333333	0.542030312874102\\
4.75155092592593	0.545281176342187\\
4.80350694444444	0.565067240266977\\
4.80724537037037	0.569287457011253\\
4.81015046296296	0.573302575376555\\
4.81168981481482	0.576971812771741\\
4.8565162037037	0.591293552972798\\
4.85902777777778	0.595204050886129\\
4.86289351851852	0.599428818157101\\
4.86568287037037	0.603462856776329\\
4.86581018518519	0.606997200074605\\
4.86608796296296	0.610603507171628\\
4.86652777777778	0.61428521242832\\
4.8666087962963	0.617936664865673\\
4.86809027777778	0.621918887198969\\
4.86835648148148	0.62569753553302\\
4.86905092592593	0.629609506800682\\
4.869375	0.633492054547893\\
4.91210648148148	0.64470486101207\\
4.9234837962963	0.650128340200006\\
4.92386574074074	0.65418724253032\\
4.92396990740741	0.658263034303693\\
4.9243287037037	0.662420423944404\\
4.9250462962963	0.666672980737785\\
4.92568287037037	0.670968828000618\\
4.92667824074074	0.675361362291244\\
4.97186342592593	0.683370005428969\\
4.98336805555556	0.688178578229455\\
4.98383101851852	0.692658916838989\\
5.02824074074074	0.696711932006284\\
5.03604166666667	0.700896233103934\\
5.03638888888889	0.705517685864266\\
5.0368287037037	0.710195058734052\\
5.03813657407407	0.714881641333566\\
5.03834490740741	0.71969887670974\\
5.03850694444444	0.724584524871344\\
5.03966435185185	0.729472432278155\\
5.03972222222222	0.734499445442012\\
5.03991898148148	0.739586979866047\\
5.04091435185185	0.744690952028987\\
5.04208333333333	0.749852848555568\\
5.04222222222222	0.755160947467045\\
5.08837962962963	0.755549436346785\\
5.09479166666667	0.760031193010146\\
5.09842592592593	0.764968035212999\\
5.0990162037037	0.770456610884263\\
5.09925925925926	0.776082128278441\\
5.14765046296296	0.772298017609996\\
5.15236111111111	0.776966735765668\\
5.15459490740741	0.782276355400743\\
5.1550462962963	0.78809360836496\\
5.15631944444444	0.793798268484048\\
5.20888888888889	0.78541063042839\\
5.21149305555556	0.790703771683877\\
5.21186342592593	0.796766655815225\\
5.21314814814815	0.802639195340466\\
5.21327546296296	0.808969191347642\\
5.21475694444444	0.81496885896719\\
5.26159722222222	0.805575339664438\\
5.27122685185185	0.808619529606491\\
5.27228009259259	0.814858055674293\\
5.27280092592593	0.821397975153531\\
5.31628472222222	0.811384309710366\\
5.32148148148148	0.816119419710922\\
5.32450231481482	0.821809188341374\\
5.32523148148148	0.828534162539694\\
5.3290625	0.834089208662191\\
5.32938657407407	0.841206487530658\\
5.32986111111111	0.848385029115171\\
5.33034722222222	0.855685618730752\\
5.37268518518519	0.844889776601425\\
5.38502314814815	0.847021808187522\\
5.38523148148148	0.854628559889492\\
5.38607638888889	0.862086698771431\\
5.38643518518519	0.86990503471739\\
5.38741898148148	0.877577105578119\\
5.38778935185185	0.885683686530327\\
5.42561342592593	0.876311546232764\\
5.43828703703704	0.87873522902032\\
5.45168981481481	0.880860310094738\\
5.49454861111111	0.868995138302941\\
5.49575231481481	0.877021302039485\\
5.498125	0.884642033595679\\
5.55136574074074	0.867807725023116\\
5.55638888888889	0.874233447058272\\
5.55946759259259	0.881744054289805\\
5.56048611111111	0.890427152236483\\
5.60539351851852	0.877612679357351\\
5.61043981481482	0.884449808654666\\
5.61315972222222	0.892599871884181\\
5.61460648148148	0.901571077729166\\
5.61685185185185	0.910332042312656\\
5.61778935185185	0.919963053399442\\
5.63293981481482	0.922394767278013\\
5.67181712962963	0.912494249437516\\
5.67417824074074	0.921728291842722\\
5.67565972222222	0.931653137372595\\
5.67583333333333	0.942519752654258\\
5.72508101851852	0.926758713568369\\
5.72584490740741	0.937500715711494\\
5.72984953703704	0.946684568299691\\
5.73196759259259	0.957167488743879\\
5.73237268518518	0.968898799235943\\
5.7333912037037	0.980569337383274\\
5.78729166666667	0.961184635738582\\
5.84063657407407	0.941401447373176\\
5.84469907407407	0.951295038088234\\
5.84494212962963	0.963827786242089\\
5.84501157407407	0.976808705880363\\
5.84572916666667	0.989731040715382\\
5.84731481481481	1.00244976913464\\
5.84744212962963	1.01648510980319\\
5.84780092592593	1.03076715522324\\
5.84868055555556	1.04511093254321\\
5.90346064814815	1.02285432427014\\
5.90456018518519	1.03735642306833\\
5.90581018518519	1.05219193432789\\
5.95017361111111	1.03674231292434\\
5.95128472222222	1.0521426946033\\
5.95733796296296	1.06442926421701\\
5.95859953703704	1.08066410409764\\
5.99944444444444	1.06798457781877\\
6.01927083333333	1.07109366234318\\
6.02068287037037	1.0881928774451\\
6.07572916666667	1.06628024834344\\
6.07767361111111	1.08357300176981\\
6.12369212962963	1.06993358177056\\
6.13309027777778	1.08283701361502\\
6.13498842592593	1.10154699924628\\
6.13619212962963	1.12147104622606\\
6.17320601851852	1.11695933021525\\
6.18494212962963	1.13083999407409\\
6.24876157407407	1.11231885971841\\
6.25163194444444	1.13325901171074\\
6.25172453703704	1.15681110017252\\
6.3034837962963	1.15037657721114\\
6.30511574074074	1.17444510678436\\
6.30893518518519	1.19830855302019\\
6.30917824074074	1.22539796852034\\
6.36299768518519	1.22268017008072\\
6.36516203703704	1.25056857289701\\
6.36681712962963	1.28011480665028\\
6.37883101851852	1.30502253406652\\
6.42065972222222	1.31404042814447\\
6.42096064814815	1.34844246226189\\
6.47592592592593	1.3514278002959\\
6.5343287037037	1.35061345515308\\
6.5375462962963	1.38691937790275\\
6.53884259259259	1.42674728711457\\
6.59601851851852	1.42374958733163\\
6.64996527777778	1.41659102792609\\
6.65070601851852	1.46150335569843\\
6.65076388888889	1.51015676394697\\
6.69883101851852	1.50428761168604\\
6.70283564814815	1.55268788116849\\
6.70502314814815	1.60715672120499\\
6.70605324074074	1.66747917380059\\
6.71027777777778	1.72777893930545\\
6.71064814814815	1.79918220691138\\
6.71068287037037	1.87735001861177\\
6.71171296296296	1.96089851150948\\
6.71193287037037	2.05387471244765\\
6.73209490740741	2.11712528149027\\
6.75421296296296	2.18100184091157\\
6.76596064814815	2.27460997786323\\
6.81811342592593	2.27160444342306\\
6.88480324074074	2.21347297843039\\
6.94032407407407	2.1763394300766\\
6.94290509259259	2.32253966358478\\
6.98989583333333	2.32020564249573\\
7.05606481481481	2.24494118385428\\
7.15710648148148	2.03798459660718\\
7.16491898148148	2.20944263542258\\
7.28027777777778	1.97643363314389\\
7.28799768518518	2.19137509904403\\
7.28805555555556	2.50415559887508\\
7.4034375	2.35213449320246\\
7.45590277777778	2.56006559754183\\
7.69079861111111	2.10342060824333\\
7.74900462962963	2.54633061008805\\
8.03364583333333	2.40158252263762\\
8.08883101851852	4.3326734724455\\
8.84060185185185	inf\\
};
\end{axis}
\end{tikzpicture}%
\caption{The hazard rate of the lifetime of the machine.}
\label{figure:hazard}
\end{figure}

\section{Fitting lifetime distributions}
To be able to predict the remaining time until a failure, it is helpful to know how the lifetime of the machine is distributed.
In this section we will attempt to fit the lifetime to a distribution.

\cite{Lai2006} mentions a few common lifetime distributions.
We tried to fit these distributions over the observed lifetimes of the machine (using maximum likelihood estimation).
The Gamma distribution and the log-normal distribution fitted the data best, although still not very well.
The probability densities of these distributions are plotted over the density of the observed lifetimes in figure \ref{figure:fits}.
As you can see, these estimations are still not very accurate as they do not include the blob on the right side of the mode.

\begin{figure}[H]\label{figure:fits}
	\centering
	\setlength\fwidth{0.7\textwidth}
	% This file was created by matlab2tikz.
%
%The latest updates can be retrieved from
%  http://www.mathworks.com/matlabcentral/fileexchange/22022-matlab2tikz-matlab2tikz
%where you can also make suggestions and rate matlab2tikz.
%
\definecolor{mycolor1}{rgb}{0.00000,0.44700,0.74100}%
\definecolor{mycolor2}{rgb}{0.85000,0.32500,0.09800}%
\definecolor{mycolor3}{rgb}{0.92900,0.69400,0.12500}%
%
\begin{tikzpicture}

\begin{axis}[%
width=0.951\fwidth,
height=0.75\fwidth,
at={(0\fwidth,0\fwidth)},
scale only axis,
xmin=3,
xmax=9,
xlabel style={font=\color{white!15!black}},
xlabel={Lifetime (d)},
ymin=0,
ymax=0.5,
ylabel style={font=\color{white!15!black}},
ylabel={Probability density},
axis background/.style={fill=white},
legend style={legend cell align=left, align=left, draw=white!15!black}
]
\addplot [color=mycolor1]
  table[row sep=crcr]{%
3.65539351851852	0.0311060613415889\\
3.65539351851852	0.0311060613415889\\
3.88612268518519	0.0746253089773507\\
4.04607638888889	0.123071799241282\\
4.05780092592593	0.127260318602443\\
4.05827546296296	0.127431461243989\\
4.06336805555556	0.129276898782796\\
4.16209490740741	0.168146163176176\\
4.16583333333333	0.169734343907335\\
4.16824074074074	0.170759418377405\\
4.17811342592593	0.174982656334577\\
4.23090277777778	0.198382132389908\\
4.23425925925926	0.199912541349046\\
4.34474537037037	0.252402378945583\\
4.34844907407407	0.254211073003693\\
4.35103009259259	0.255472496776924\\
4.39923611111111	0.279123202489396\\
4.40034722222222	0.279668462445415\\
4.40402777777778	0.281474209298978\\
4.45054398148148	0.304175146339895\\
4.45975694444444	0.308631239966174\\
4.46435185185185	0.310843709669417\\
4.46532407407407	0.311311134444933\\
4.51805555555556	0.336210692488908\\
4.51966435185185	0.336953134075069\\
4.52009259259259	0.337150565141363\\
4.52246527777778	0.338242920655672\\
4.52336805555556	0.338657866443379\\
4.52395833333333	0.33892897192696\\
4.57143518518519	0.360142241614023\\
4.57194444444444	0.360362785263918\\
4.57393518518519	0.361223353467588\\
4.5762962962963	0.36224078682367\\
4.57699074074074	0.362539357159603\\
4.58032407407407	0.363968180905486\\
4.58097222222222	0.36424517235094\\
4.58162037037037	0.364521889805077\\
4.61822916666667	0.37968063473534\\
4.62917824074074	0.384039203306384\\
4.62978009259259	0.384275265977709\\
4.63076388888889	0.384660519508577\\
4.63206018518519	0.385166972627279\\
4.63226851851852	0.385248241904796\\
4.6349537037037	0.386292600308746\\
4.63815972222222	0.387531910989408\\
4.68565972222222	0.404871800094251\\
4.6862962962963	0.405090451855561\\
4.6946412037037	0.407921643805695\\
4.69556712962963	0.408231729408636\\
4.74951388888889	0.424844467361337\\
4.75083333333333	0.425213435099512\\
4.75155092592593	0.425413331542827\\
4.80350694444444	0.438414238138172\\
4.80724537037037	0.439234718987131\\
4.81015046296296	0.439861458694081\\
4.81168981481482	0.440189701985337\\
4.8565162037037	0.448567522944882\\
4.85902777777778	0.448968572866692\\
4.86289351851852	0.449571613617826\\
4.86568287037037	0.449996009578901\\
4.86581018518519	0.450015165572552\\
4.86608796296296	0.450056895372192\\
4.86652777777778	0.450122784969028\\
4.8666087962963	0.450134898113357\\
4.86809027777778	0.450355056247529\\
4.86835648148148	0.450394346698338\\
4.86905092592593	0.450496457452213\\
4.869375	0.450543918105183\\
4.91210648148148	0.455739643129222\\
4.9234837962963	0.456771204537073\\
4.92386574074074	0.456803160732379\\
4.92396990740741	0.456811847081443\\
4.9243287037037	0.456841671685796\\
4.9250462962963	0.45690087904012\\
4.92568287037037	0.456952908724559\\
4.92667824074074	0.457033335688471\\
4.97186342592593	0.459507417443618\\
4.98336805555556	0.459774481144679\\
4.98383101851852	0.459782212039674\\
5.02824074074074	0.459469506883455\\
5.03604166666667	0.459207876861199\\
5.03638888888889	0.45919470071338\\
5.0368287037037	0.459177839698741\\
5.03813657407407	0.459126571373713\\
5.03834490740741	0.459118248935525\\
5.03850694444444	0.459111746362447\\
5.03966435185185	0.459064547899184\\
5.03972222222222	0.459062153401258\\
5.03991898148148	0.459053987503064\\
5.04091435185185	0.45901209543166\\
5.04208333333333	0.458961657305564\\
5.04222222222222	0.458955575831265\\
5.08837962962963	0.455935004692026\\
5.09479166666667	0.455363516501769\\
5.09842592592593	0.455024089911181\\
5.0990162037037	0.454967912461828\\
5.09925925925926	0.454944695887363\\
5.14765046296296	0.449397553350645\\
5.15236111111111	0.448765269795688\\
5.15459490740741	0.44846015201853\\
5.1550462962963	0.448398087517995\\
5.15631944444444	0.448222298152631\\
5.20888888888889	0.440100784291771\\
5.21149305555556	0.439658562703536\\
5.21186342592593	0.439595396311849\\
5.21314814814815	0.439375766414825\\
5.21327546296296	0.439353957369841\\
5.21475694444444	0.439099600736633\\
5.26159722222222	0.430566129820649\\
5.27122685185185	0.428707767851718\\
5.27228009259259	0.428502943070104\\
5.27280092592593	0.428401530144731\\
5.31628472222222	0.419681539505363\\
5.32148148148148	0.418612978213793\\
5.32450231481482	0.417989156139148\\
5.32523148148148	0.41783834921183\\
5.3290625	0.417044604331096\\
5.32938657407407	0.41697735373287\\
5.32986111111111	0.41687885051349\\
5.33034722222222	0.416777909123169\\
5.37268518518519	0.407877823186896\\
5.38502314814815	0.405256123744893\\
5.38523148148148	0.405211817188985\\
5.38607638888889	0.405032112784854\\
5.38643518518519	0.404955792023614\\
5.38741898148148	0.40474461334853\\
5.38778935185185	0.404665822294029\\
5.42561342592593	0.39660651876914\\
5.43828703703704	0.393915792319455\\
5.45168981481481	0.391071603188613\\
5.49454861111111	0.382058207357328\\
5.49575231481481	0.381806687525811\\
5.498125	0.381311221377449\\
5.55136574074074	0.370314503350382\\
5.55638888888889	0.369288266429788\\
5.55946759259259	0.368660229595307\\
5.56048611111111	0.368452614718545\\
5.60539351851852	0.359367260943744\\
5.61043981481482	0.35835466385146\\
5.61315972222222	0.357809431401849\\
5.61460648148148	0.357519565306394\\
5.61685185185185	0.35706989590712\\
5.61778935185185	0.356882218991164\\
5.63293981481482	0.353849716757515\\
5.67181712962963	0.346118508407335\\
5.67417824074074	0.345648109441022\\
5.67565972222222	0.345354180232946\\
5.67583333333333	0.345319736963845\\
5.72508101851852	0.33555056870579\\
5.72584490740741	0.33539896294851\\
5.72984953703704	0.334604028450754\\
5.73196759259259	0.33418347667351\\
5.73237268518518	0.334103034219292\\
5.7333912037037	0.333900765746892\\
5.78729166666667	0.323156903394869\\
5.84063657407407	0.312447894171271\\
5.84469907407407	0.311631133166836\\
5.84494212962963	0.311582258483435\\
5.84501157407407	0.311568294117013\\
5.84572916666667	0.311423991259582\\
5.84731481481481	0.311105100765924\\
5.84744212962963	0.31107949481046\\
5.84780092592593	0.311007331317359\\
5.84868055555556	0.310830406661559\\
5.90346064814815	0.299802129527457\\
5.90456018518519	0.299581380799908\\
5.90581018518519	0.299330464076039\\
5.95017361111111	0.290466596293458\\
5.95128472222222	0.29024626058022\\
5.95733796296296	0.289047601921\\
5.95859953703704	0.288798165750233\\
5.99944444444444	0.280806289857521\\
6.01927083333333	0.277006981640479\\
6.02068287037037	0.276738705901988\\
6.07572916666667	0.266570062085861\\
6.07767361111111	0.266222677158962\\
6.12369212962963	0.25825983008255\\
6.13309027777778	0.256707050641493\\
6.13498842592593	0.256394560169394\\
6.13619212962963	0.256198126939575\\
6.17320601851852	0.250352953324108\\
6.18494212962963	0.248589826283529\\
6.24876157407407	0.239723892180693\\
6.25163194444444	0.239352118852701\\
6.25172453703704	0.2393402276219\\
6.3034837962963	0.233050427279844\\
6.30511574074074	0.232864116000349\\
6.30893518518519	0.232430538301331\\
6.30917824074074	0.232403062995238\\
6.36299768518519	0.226617445316686\\
6.36516203703704	0.226396034748598\\
6.36681712962963	0.226227185658025\\
6.37883101851852	0.225003885183884\\
6.42065972222222	0.220894727144977\\
6.42096064814815	0.22086557571531\\
6.47592592592593	0.215529433667882\\
6.5343287037037	0.209577949937548\\
6.5375462962963	0.209233526838778\\
6.53884259259259	0.209092274835756\\
6.59601851851852	0.202516105094586\\
6.64996527777778	0.19539186592084\\
6.65070601851852	0.195287086321773\\
6.65076388888889	0.195278891889695\\
6.69883101851852	0.188035951460756\\
6.70283564814815	0.187393364968611\\
6.70502314814815	0.187039790829892\\
6.70605324074074	0.186872666029378\\
6.71027777777778	0.186183075356191\\
6.71064814814815	0.186122297266695\\
6.71068287037037	0.186116596672719\\
6.71171296296296	0.185947272643141\\
6.71193287037037	0.185911073109487\\
6.73209490740741	0.182510800128473\\
6.75421296296296	0.178616530074655\\
6.76596064814815	0.176478360351458\\
6.81811342592593	0.166453773871518\\
6.88480324074074	0.152653308857269\\
6.94032407407407	0.140711601082539\\
6.94290509259259	0.140153255561151\\
6.98989583333333	0.130011523070882\\
7.05606481481481	0.116117647440739\\
7.15710648148148	0.0966285800115478\\
7.16491898148148	0.095234596354422\\
7.28027777777778	0.0766719943892029\\
7.28799768518518	0.0755646585877256\\
7.28805555555556	0.0755564189315758\\
7.4034375	0.0608310644793743\\
7.45590277777778	0.0551738275332293\\
7.69079861111111	0.0362658725559196\\
7.74900462962963	0.0329266889235524\\
8.03364583333333	0.0207032976089452\\
8.08883101851852	0.0186753166915754\\
8.84060185185185	0.00608852609872549\\
};
\addlegendentry{Observed}

\addplot [color=mycolor2]
  table[row sep=crcr]{%
3.65539351851852	0.0391965073696616\\
3.65539351851852	0.0391965073696616\\
3.88612268518519	0.0782714425008287\\
4.04607638888889	0.116703631748607\\
4.05780092592593	0.119882649173509\\
4.05827546296296	0.120012317661526\\
4.06336805555556	0.121408758707586\\
4.16209490740741	0.150171059820832\\
4.16583333333333	0.151320119383177\\
4.16824074074074	0.152062247519286\\
4.17811342592593	0.155123325984739\\
4.23090277777778	0.171949231709942\\
4.23425925925926	0.173043766137226\\
4.34474537037037	0.210451255921093\\
4.34844907407407	0.211744455730498\\
4.35103009259259	0.212646869938005\\
4.39923611111111	0.229662908761296\\
4.40034722222222	0.230058252682247\\
4.40402777777778	0.231368678910887\\
4.45054398148148	0.248021566788717\\
4.45975694444444	0.251334387949898\\
4.46435185185185	0.252987734353704\\
4.46532407407407	0.25333763954495\\
4.51805555555556	0.272325869822675\\
4.51966435185185	0.272904573025265\\
4.52009259259259	0.273058601007824\\
4.52246527777778	0.2739118769187\\
4.52336805555556	0.274236481420118\\
4.52395833333333	0.274448705300261\\
4.57143518518519	0.291450940362877\\
4.57194444444444	0.291632361467382\\
4.57393518518519	0.292341309957968\\
4.5762962962963	0.293181644062963\\
4.57699074074074	0.293428693348531\\
4.58032407407407	0.294613830844281\\
4.58097222222222	0.294844137489713\\
4.58162037037037	0.295074398839929\\
4.61822916666667	0.307996281907833\\
4.62917824074074	0.311824044017845\\
4.62978009259259	0.312033889858613\\
4.63076388888889	0.31237677764191\\
4.63206018518519	0.312828335753722\\
4.63226851851852	0.312900881246795\\
4.6349537037037	0.313835252585219\\
4.63815972222222	0.314949233894136\\
4.68565972222222	0.331221211620492\\
4.6862962962963	0.331436013490677\\
4.6946412037037	0.334243053111763\\
4.69556712962963	0.33455348747899\\
4.74951388888889	0.352250423989753\\
4.75083333333333	0.352672884459663\\
4.75155092592593	0.352902420363506\\
4.80350694444444	0.369076576225681\\
4.80724537037037	0.370204405660797\\
4.81015046296296	0.371077297871569\\
4.81168981481482	0.371538564522719\\
4.8565162037037	0.384570091932575\\
4.85902777777778	0.385276348313898\\
4.86289351851852	0.386358239378235\\
4.86568287037037	0.387134974485335\\
4.86581018518519	0.387170348496339\\
4.86608796296296	0.387247504273229\\
4.86652777777778	0.387369600557244\\
4.8666087962963	0.38739208301078\\
4.86809027777778	0.387802698090007\\
4.86835648148148	0.387876381340508\\
4.86905092592593	0.38806845609253\\
4.869375	0.388158020442473\\
4.91210648148148	0.399562247194678\\
4.9234837962963	0.402457982949802\\
4.92386574074074	0.402554133335406\\
4.92396990740741	0.402580344116426\\
4.9243287037037	0.402670586142284\\
4.9250462962963	0.402850886145214\\
4.92568287037037	0.403010624088795\\
4.92667824074074	0.403260008175751\\
4.97186342592593	0.41406941063572\\
4.98336805555556	0.416656891674948\\
4.98383101851852	0.416759577328118\\
5.02824074074074	0.426079576784174\\
5.03604166666667	0.427606053211011\\
5.03638888888889	0.427673213377648\\
5.0368287037037	0.427758186726089\\
5.03813657407407	0.428010234957295\\
5.03834490740741	0.428050296327786\\
5.03850694444444	0.428081438460618\\
5.03966435185185	0.428303456811925\\
5.03972222222222	0.428314538126971\\
5.03991898148148	0.428352200626944\\
5.04091435185185	0.428542397527174\\
5.04208333333333	0.428765061896776\\
5.04222222222222	0.428791466313086\\
5.08837962962963	0.436960385793963\\
5.09479166666667	0.4379980859819\\
5.09842592592593	0.438575564049152\\
5.0990162037037	0.438668627133959\\
5.09925925925926	0.438706887778474\\
5.14765046296296	0.445626114821438\\
5.15236111111111	0.446224660171915\\
5.15459490740741	0.4465037941149\\
5.1550462962963	0.446559831987159\\
5.15631944444444	0.446717221537875\\
5.20888888888889	0.452351260388676\\
5.21149305555556	0.452586187546755\\
5.21186342592593	0.452619259037844\\
5.21314814814815	0.452733318631911\\
5.21327546296296	0.452744566269484\\
5.21475694444444	0.45287471105012\\
5.26159722222222	0.456288104501826\\
5.27122685185185	0.456821036766911\\
5.27228009259259	0.456875831417636\\
5.27280092592593	0.456902673058496\\
5.31628472222222	0.458550136536001\\
5.32148148148148	0.458668730738072\\
5.32450231481482	0.458730010600388\\
5.32523148148148	0.458743959362839\\
5.3290625	0.458811863600415\\
5.32938657407407	0.458817193099953\\
5.32986111111111	0.458824880355624\\
5.33034722222222	0.458832611376051\\
5.37268518518519	0.458950418363649\\
5.38502314814815	0.458779445378623\\
5.38523148148148	0.45877576903849\\
5.38607638888889	0.458760591262025\\
5.38643518518519	0.458754015783501\\
5.38741898148148	0.458735588436396\\
5.38778935185185	0.458728500070289\\
5.42561342592593	0.457572630938337\\
5.43828703703704	0.456995858263498\\
5.45168981481481	0.456283845450251\\
5.49454861111111	0.453314929113886\\
5.49575231481481	0.453216561637211\\
5.498125	0.453020299457958\\
5.55136574074074	0.447806411219346\\
5.55638888888889	0.447236110801849\\
5.55946759259259	0.44688004294799\\
5.56048611111111	0.446761157674393\\
5.60539351851852	0.440992240145313\\
5.61043981481482	0.44028095455561\\
5.61315972222222	0.4398924220541\\
5.61460648148148	0.439684290802025\\
5.61685185185185	0.439359262497907\\
5.61778935185185	0.439222833444995\\
5.63293981481482	0.436959674048755\\
5.67181712962963	0.430662851105381\\
5.67417824074074	0.430258419587384\\
5.67565972222222	0.430003402268444\\
5.67583333333333	0.429973454157642\\
5.72508101851852	0.420958272934047\\
5.72584490740741	0.420810527617039\\
5.72984953703704	0.420032167446849\\
5.73196759259259	0.419617911610337\\
5.73237268518518	0.419538479920638\\
5.7333912037037	0.419338479479844\\
5.78729166666667	0.408191861519303\\
5.84063657407407	0.396158968579786\\
5.84469907407407	0.395205439567846\\
5.84494212962963	0.395148232655932\\
5.84501157407407	0.395131884557525\\
5.84572916666667	0.394962869282455\\
5.84731481481481	0.394588852110608\\
5.84744212962963	0.39455878889067\\
5.84780092592593	0.394474039215156\\
5.84868055555556	0.3942661032915\\
5.90346064814815	0.380886135598317\\
5.90456018518519	0.380609399181243\\
5.90581018518519	0.380294426759409\\
5.95017361111111	0.368876569870847\\
5.95128472222222	0.368584964131086\\
5.95733796296296	0.366991798201514\\
5.95859953703704	0.366658810422459\\
5.99944444444444	0.355712307277214\\
6.01927083333333	0.35029334587633\\
6.02068287037037	0.349905035661813\\
6.07572916666667	0.334553813478987\\
6.07767361111111	0.334004806048521\\
6.12369212962963	0.320907183230165\\
6.13309027777778	0.318211498399996\\
6.13498842592593	0.317666347793967\\
6.13619212962963	0.317320526103771\\
6.17320601851852	0.306649356017734\\
6.18494212962963	0.303254238822469\\
6.24876157407407	0.284757709623229\\
6.25163194444444	0.283926120294938\\
6.25172453703704	0.283899297282112\\
6.3034837962963	0.268944436813564\\
6.30511574074074	0.268474662351379\\
6.30893518518519	0.267375724950591\\
6.30917824074074	0.267305818471352\\
6.36299768518519	0.251918616098513\\
6.36516203703704	0.251304273247139\\
6.36681712962963	0.250834746204252\\
6.37883101851852	0.24743366441529\\
6.42065972222222	0.235699717935024\\
6.42096064814815	0.235615961467833\\
6.47592592592593	0.220497814059692\\
6.5343287037037	0.204878010294223\\
6.5375462962963	0.204032194637212\\
6.53884259259259	0.203691889961786\\
6.59601851851852	0.188952938998539\\
6.64996527777778	0.175563455136432\\
6.65070601851852	0.175383286802903\\
6.65076388888889	0.175369215445825\\
6.69883101851852	0.16390015878561\\
6.70283564814815	0.162964672917672\\
6.70502314814815	0.162454993026365\\
6.70605324074074	0.162215309269121\\
6.71027777777778	0.161234512381039\\
6.71064814814815	0.161148691760255\\
6.71068287037037	0.161140647459404\\
6.71171296296296	0.160902107703659\\
6.71193287037037	0.16085121051987\\
6.73209490740741	0.156225369730294\\
6.75421296296296	0.151244103734519\\
6.76596064814815	0.148638533211955\\
6.81811342592593	0.137412541070876\\
6.88480324074074	0.123880971624953\\
6.94032407407407	0.113327640370464\\
6.94290509259259	0.112852794049051\\
6.98989583333333	0.104451341897768\\
7.05606481481481	0.0933976934492163\\
7.15710648148148	0.0782266349880034\\
7.16491898148148	0.0771371772237238\\
7.28027777777778	0.062381501392703\\
7.28799768518518	0.0614801921294619\\
7.28805555555556	0.06147347515762\\
7.4034375	0.0492066942709017\\
7.45590277777778	0.04433347944635\\
7.69079861111111	0.0271676623442419\\
7.74900462962963	0.0239290029596777\\
8.03364583333333	0.0124775661227122\\
8.08883101851852	0.0109357506846589\\
8.84060185185185	0.00153406025285768\\
};
\addlegendentry{Gamma}

\addplot [color=mycolor3]
  table[row sep=crcr]{%
3.65539351851852	0.0315930067177022\\
3.65539351851852	0.0315930067177022\\
3.88612268518519	0.0716129149621451\\
4.04607638888889	0.113274256104093\\
4.05780092592593	0.116773960469671\\
4.05827546296296	0.116916833565613\\
4.06336805555556	0.118456073307555\\
4.16209490740741	0.150349167678444\\
4.16583333333333	0.15162890504352\\
4.16824074074074	0.152455592737911\\
4.17811342592593	0.155866704882163\\
4.23090277777778	0.174643282117539\\
4.23425925925926	0.175865750052902\\
4.34474537037037	0.2176340871865\\
4.34844907407407	0.219075683358871\\
4.35103009259259	0.220081485095925\\
4.39923611111111	0.239016513768934\\
4.40034722222222	0.239455648494986\\
4.40402777777778	0.240910933410999\\
4.45054398148148	0.259361107514755\\
4.45975694444444	0.263020704675936\\
4.46435185185185	0.264845640080319\\
4.46532407407407	0.265231730253706\\
4.51805555555556	0.286109730931025\\
4.51966435185185	0.286743567257506\\
4.52009259259259	0.286912243141898\\
4.52246527777778	0.287846462802398\\
4.52336805555556	0.288201769803509\\
4.52395833333333	0.288434039714928\\
4.57143518518519	0.306968444443221\\
4.57194444444444	0.307165380895864\\
4.57393518518519	0.307934780612511\\
4.5762962962963	0.308846398005794\\
4.57699074074074	0.309114326813703\\
4.58032407407407	0.310399136803965\\
4.58097222222222	0.310648718072355\\
4.58162037037037	0.310898219303362\\
4.61822916666667	0.324848025518907\\
4.62917824074074	0.328959662349098\\
4.62978009259259	0.32918478345194\\
4.63076388888889	0.329552565824075\\
4.63206018518519	0.330036784547605\\
4.63226851851852	0.330114564011337\\
4.6349537037037	0.331116020924039\\
4.63815972222222	0.332309185157695\\
4.68565972222222	0.349632818056558\\
4.6862962962963	0.349860112178181\\
4.6946412037037	0.352826857433508\\
4.69556712962963	0.353154545087497\\
4.74951388888889	0.371690451850331\\
4.75083333333333	0.37212923615202\\
4.75155092592593	0.372367564025102\\
4.80350694444444	0.389014974638047\\
4.80724537037037	0.390164266877546\\
4.81015046296296	0.391052657984112\\
4.81168981481482	0.391521718574341\\
4.8565162037037	0.404652449900448\\
4.85902777777778	0.405356937720316\\
4.86289351851852	0.406434586561296\\
4.86568287037037	0.407207116370009\\
4.86581018518519	0.407242275553872\\
4.86608796296296	0.407318955683399\\
4.86652777777778	0.407440279412761\\
4.8666087962963	0.407462616953699\\
4.86809027777778	0.407870439593447\\
4.86835648148148	0.407943592418013\\
4.86905092592593	0.408134242360136\\
4.869375	0.408223121469713\\
4.91210648148148	0.419424118279232\\
4.9234837962963	0.422228242878986\\
4.92386574074074	0.42232104678629\\
4.92396990740741	0.422346341843599\\
4.9243287037037	0.422433419713792\\
4.9250462962963	0.422607344935392\\
4.92568287037037	0.422761375989528\\
4.92667824074074	0.423001738940077\\
4.97186342592593	0.433277751048753\\
4.98336805555556	0.435691399583365\\
4.98383101851852	0.43578677351607\\
5.02824074074074	0.444294373870171\\
5.03604166666667	0.445656041051024\\
5.03638888888889	0.445715716299448\\
5.0368287037037	0.445791190590365\\
5.03813657407407	0.446014871851283\\
5.03834490740741	0.446050398056992\\
5.03850694444444	0.446078009701621\\
5.03966435185185	0.446274730507957\\
5.03972222222222	0.446284543275121\\
5.03991898148148	0.446317890099663\\
5.04091435185185	0.446486192927966\\
5.04208333333333	0.446683012846598\\
5.04222222222222	0.44670633719696\\
5.08837962962963	0.45374403761591\\
5.09479166666667	0.454608247248305\\
5.09842592592593	0.455085698832781\\
5.0990162037037	0.455162400887773\\
5.09925925925926	0.455193915357302\\
5.14765046296296	0.460666862051421\\
5.15236111111111	0.461114212700323\\
5.15459490740741	0.461321041489841\\
5.1550462962963	0.461362421335323\\
5.15631944444444	0.461478382726298\\
5.20888888888889	0.465299213220806\\
5.21149305555556	0.465439486529844\\
5.21186342592593	0.465459061987131\\
5.21314814814815	0.465526241731325\\
5.21327546296296	0.46553283809714\\
5.21475694444444	0.465608786125849\\
5.26159722222222	0.467244781045351\\
5.27122685185185	0.467398520978826\\
5.27228009259259	0.467411588392468\\
5.27280092592593	0.46741777770565\\
5.31628472222222	0.467303374220536\\
5.32148148148148	0.467207060184445\\
5.32450231481482	0.467143053865539\\
5.32523148148148	0.467126722518213\\
5.3290625	0.467035295965608\\
5.32938657407407	0.467027129160482\\
5.32986111111111	0.467015048944359\\
5.33034722222222	0.46700252419693\\
5.37268518518519	0.465336166100842\\
5.38502314814815	0.464639682011586\\
5.38523148148148	0.464627117634754\\
5.38607638888889	0.464575889584477\\
5.38643518518519	0.464554003005702\\
5.38741898148148	0.464493587390402\\
5.38778935185185	0.464470689356899\\
5.42561342592593	0.461696330742013\\
5.43828703703704	0.460577080883142\\
5.45168981481481	0.459292341217369\\
5.49454861111111	0.454506229927172\\
5.49575231481481	0.454357275394334\\
5.498125	0.454061387450971\\
5.55136574074074	0.4466501086658\\
5.55638888888889	0.445876974971006\\
5.55946759259259	0.445397032113494\\
5.56048611111111	0.445237241048465\\
5.60539351851852	0.437705708169151\\
5.61043981481482	0.436801813973101\\
5.61315972222222	0.436309962310975\\
5.61460648148148	0.436047017160842\\
5.61685185185185	0.43563711620259\\
5.61778935185185	0.435465322525208\\
5.63293981481482	0.432636667177446\\
5.67181712962963	0.424944035254519\\
5.67417824074074	0.424457553413042\\
5.67565972222222	0.42415121917295\\
5.67583333333333	0.424115265750828\\
5.72508101851852	0.413470679615348\\
5.72584490740741	0.413298867704501\\
5.72984953703704	0.412394966030645\\
5.73196759259259	0.411914737841747\\
5.73237268518518	0.411822722257872\\
5.7333912037037	0.411591130317538\\
5.78729166666667	0.398873643663306\\
5.84063657407407	0.385490587900312\\
5.84469907407407	0.384442888532236\\
5.84494212962963	0.384380086871445\\
5.84501157407407	0.384362141090619\\
5.84572916666667	0.384176637703862\\
5.84731481481481	0.383766324753694\\
5.84744212962963	0.383733355496616\\
5.84780092592593	0.383640422658648\\
5.84868055555556	0.383412465887422\\
5.90346064814815	0.368901703745352\\
5.90456018518519	0.368604623717474\\
5.90581018518519	0.368266636820734\\
5.95017361111111	0.356109889528385\\
5.95128472222222	0.355801721899461\\
5.95733796296296	0.354119992939339\\
5.95859953703704	0.353768901492562\\
5.99944444444444	0.342301922290282\\
6.01927083333333	0.336675837630438\\
6.02068287037037	0.336273898610468\\
6.07572916666667	0.320505614598195\\
6.07767361111111	0.319945882378816\\
6.12369212962963	0.306671554788074\\
6.13309027777778	0.303957572999751\\
6.13498842592593	0.303409441699248\\
6.13619212962963	0.303061853564405\\
6.17320601851852	0.292382356206059\\
6.18494212962963	0.289002732414914\\
6.24876157407407	0.270733777353154\\
6.25163194444444	0.269917834873022\\
6.25172453703704	0.269891524001991\\
6.3034837962963	0.255292465639274\\
6.30511574074074	0.254836072215917\\
6.30893518518519	0.253768944610322\\
6.30917824074074	0.253701085629096\\
6.36299768518519	0.238832223555173\\
6.36516203703704	0.238241288256079\\
6.36681712962963	0.237789787109129\\
6.37883101851852	0.234522768827404\\
6.42065972222222	0.223297078608101\\
6.42096064814815	0.223217197671132\\
6.47592592592593	0.208853197457142\\
6.5343287037037	0.194119970184463\\
6.5375462962963	0.193325082699535\\
6.53884259259259	0.193005350048258\\
6.59601851851852	0.179200773704579\\
6.64996527777778	0.166728803843125\\
6.65070601851852	0.166561395659138\\
6.65076388888889	0.166548321330087\\
6.69883101851852	0.155912622211993\\
6.70283564814815	0.155046863718688\\
6.70502314814815	0.154575279810585\\
6.70605324074074	0.154353537004973\\
6.71027777777778	0.153446326231519\\
6.71064814814815	0.153366957470355\\
6.71068287037037	0.15335951803429\\
6.71171296296296	0.153138922807022\\
6.71193287037037	0.153091856524117\\
6.73209490740741	0.148817189067223\\
6.75421296296296	0.144220459238812\\
6.76596064814815	0.141818530142612\\
6.81811342592593	0.13148782673092\\
6.88480324074074	0.119068081265702\\
6.94032407407407	0.109399998797508\\
6.94290509259259	0.10896524888857\\
6.98989583333333	0.101275581279868\\
7.05606481481481	0.0911603774448656\\
7.15710648148148	0.0772619765531253\\
7.16491898148148	0.076262400138364\\
7.28027777777778	0.0626875131361991\\
7.28799768518518	0.0618554279662462\\
7.28805555555556	0.0618492253067717\\
7.4034375	0.0504740060039697\\
7.45590277777778	0.0459208161302344\\
7.69079861111111	0.0296173610549714\\
7.74900462962963	0.0264704284323471\\
8.03364583333333	0.0149926115586325\\
8.08883101851852	0.013381680942153\\
8.84060185185185	0.00258710649522333\\
};
\addlegendentry{Log-Normal}

\end{axis}
\end{tikzpicture}%
	\caption{The probability densities over the density of the observed lifetimes.}
\end{figure}
\subsection{Phase-type}
As the class of Phase-Type distributions is dense in the space of positive continuous distributions \cite{Ocinneide1999}, a Phase-Type distribution could also be used to model the lifetimes.
However, Phase-Type distributions have a few disadvantages:
the number of parameters grows quadratically with the amount of states and most of these parameters are redundant.
Furthermore, convergence of the EM-algorithm (to estimate the parameters) is slow and can get stuck in saddle points and local maxima \cite{Asmussen1996}.
Because of this, we will not use a Phase-type distribution to model the lifetime of the machine.

\section{Transition times}
In order to model the transitions between the events as a Markov chain, we need to find out whether the transition times (i.e. the length of the time intervals between the start and end of an event) are exponentially distributed.
It turns out that the exponential distribution fits the transition times reasonably well.
This justifies modeling the transition times as exponential.
An example of an event of which the time length fits an exponential well is given by \ref{figure:transitionFitGood}.
An example of an event of which the time length does not fit an exponential distribution is given by \ref{figure:transitionFitBad}.

\begin{figure}[H]\label{figure:transitionFitGood}
\centering
\setlength\fwidth{0.5\textwidth}
% This file was created by matlab2tikz.
%
%The latest updates can be retrieved from
%  http://www.mathworks.com/matlabcentral/fileexchange/22022-matlab2tikz-matlab2tikz
%where you can also make suggestions and rate matlab2tikz.
%
\definecolor{mycolor1}{rgb}{0.00000,0.44700,0.74100}%
\definecolor{mycolor2}{rgb}{0.85000,0.32500,0.09800}%
%
\begin{tikzpicture}

\begin{axis}[%
width=0.951\fwidth,
height=0.75\fwidth,
at={(0\fwidth,0\fwidth)},
scale only axis,
xmin=0,
xmax=8000,
xlabel style={font=\color{white!15!black}},
xlabel={Transition time (s)},
ymin=0,
ymax=0.003,
ylabel style={font=\color{white!15!black}},
ylabel={Probability density},
axis background/.style={fill=white},
legend style={legend cell align=left, align=left, draw=white!15!black}
]
\addplot [color=mycolor1]
  table[row sep=crcr]{%
0.0524647823054307	0.000168720750067794\\
71.1379424924512	0.00234352862109625\\
142.223420202597	0.0018875382323276\\
213.308897912743	0.00153762541093055\\
284.394375622888	0.00124312037357338\\
355.479853333034	0.00101458997062068\\
426.56533104318	0.000824763018476088\\
497.650808753326	0.000663802326618968\\
568.736286463471	0.00053470388915322\\
639.821764173617	0.000435337947947336\\
710.907241883763	0.000358898922185091\\
781.992719593909	0.000298514198205878\\
853.078197304054	0.000249294551717048\\
924.1636750142	0.000208317961194143\\
995.249152724346	0.000173892549115089\\
1066.33463043449	0.000144941318298182\\
1137.42010814464	0.000120668931029797\\
1208.50558585478	0.000100402413300655\\
1279.59106356493	8.35409201207265e-05\\
1350.67654127507	6.95512448498407e-05\\
1421.76201898522	5.79614889971335e-05\\
1492.84749669537	4.8367880821619e-05\\
1563.93297440551	4.04304971582347e-05\\
1635.01845211566	3.38594924160877e-05\\
1706.1039298258	2.84153498651808e-05\\
1777.18940753595	2.39001654439114e-05\\
1848.2748852461	2.01505417310171e-05\\
1919.36036295624	1.70337738842155e-05\\
1990.44584066639	1.44390891314175e-05\\
2061.53131837653	1.22764808765788e-05\\
2132.61679608668	1.04703844385699e-05\\
2203.70227379682	8.96116927949649e-06\\
2274.78775150697	7.69629065341224e-06\\
2345.87322921712	6.6338726997167e-06\\
2416.95870692726	5.73978505700662e-06\\
2488.04418463741	4.98541278507439e-06\\
2559.12966234755	4.34640869247088e-06\\
2630.2151400577	3.80413044070322e-06\\
2701.30061776784	3.34183837195048e-06\\
2772.38609547799	2.94588676477087e-06\\
2843.47157318814	2.60602093498567e-06\\
2914.55705089828	2.31208513321993e-06\\
2985.64252860843	2.05701770750836e-06\\
3056.72800631857	1.8344505479959e-06\\
3127.81348402872	1.63983547141636e-06\\
3198.89896173886	1.46891598666636e-06\\
3269.98443944901	1.31757828092523e-06\\
3341.06991715916	1.18335953496068e-06\\
3412.1553948693	1.06392288939478e-06\\
3483.24087257945	9.57194357181979e-07\\
3554.32635028959	8.61512642166951e-07\\
3625.41182799974	7.75771338971165e-07\\
3696.49730570989	6.98691045806068e-07\\
3767.58278342003	6.28815342783803e-07\\
3838.66826113018	5.66128299116094e-07\\
3909.75373884032	5.0929676615934e-07\\
3980.83921655047	4.58149985919918e-07\\
4051.92469426061	4.11712633667751e-07\\
4123.01017197076	3.6982647577408e-07\\
4194.09564968091	3.31958602779233e-07\\
4265.18112739105	2.97727622489496e-07\\
4336.2666051012	2.66751170356841e-07\\
4407.35208281134	2.38748769873207e-07\\
4478.43756052149	2.13544540625344e-07\\
4549.52303823163	1.90746609663693e-07\\
4620.60851594178	1.70320697825183e-07\\
4691.69399365193	1.51950995820596e-07\\
4762.77947136207	1.35493037815564e-07\\
4833.86494907222	1.20659447839365e-07\\
4904.95042678236	1.07457755058982e-07\\
4976.03590449251	9.56194510184757e-08\\
5047.12138220265	8.4967264364918e-08\\
5118.2068599128	7.54901630442656e-08\\
5189.29233762295	6.69711489646621e-08\\
5260.37781533309	5.94166982294639e-08\\
5331.46329304324	5.26770938298371e-08\\
5402.54877075338	4.66300902766071e-08\\
5473.63424846353	4.12895277723282e-08\\
5544.71972617367	3.65378602447837e-08\\
5615.80520388382	3.22747491303988e-08\\
5686.89068159397	2.85304106467529e-08\\
5757.97615930411	2.52067929679982e-08\\
5829.06163701426	2.22153633850427e-08\\
5900.1471147244	1.96106027929002e-08\\
5971.23259243455	1.72639178114042e-08\\
6042.3180701447	1.51894750768295e-08\\
6113.40354785484	1.3355968821806e-08\\
6184.48902556499	1.17752820977408e-08\\
6255.57450327513	1.03775118573326e-08\\
6326.65998098528	9.14223983515619e-09\\
6397.74545869542	8.05119611346466e-09\\
6468.83093640557	7.08805575266619e-09\\
6539.91641411572	6.20153348951684e-09\\
6611.00189182586	5.45809240738225e-09\\
6682.08736953601	4.8024522839825e-09\\
6753.17284724615	4.22449636321371e-09\\
6824.2583249563	3.67883967071346e-09\\
6895.34380266644	3.23598714884239e-09\\
6966.42928037659	2.84583056759748e-09\\
7037.51475808674	2.50222260705933e-09\\
};
\addlegendentry{Observed}

\addplot [color=mycolor2]
  table[row sep=crcr]{%
0.0524647823054307	0.00282263910246041\\
71.1379424924512	0.00230941427034239\\
142.223420202597	0.00188950626646251\\
213.308897912743	0.00154594780886661\\
284.394375622888	0.00126485668248875\\
355.479853333034	0.00103487479852852\\
426.56533104318	0.000846709246554478\\
497.650808753326	0.000692756794561265\\
568.736286463471	0.000566796664101295\\
639.821764173617	0.000463739166412384\\
710.907241883763	0.000379420042645874\\
781.992719593909	0.000310432198071835\\
853.078197304054	0.000253988031121632\\
924.1636750142	0.00020780679437806\\
995.249152724346	0.000170022436092688\\
1066.33463043449	0.000139108198369592\\
1137.42010814464	0.000113814925243657\\
1208.50558585478	9.31205878592623e-05\\
1279.59106356493	7.61889872061207e-05\\
1350.67654127507	6.23359657078996e-05\\
1421.76201898522	5.10017623705104e-05\\
1492.84749669537	4.17283944406492e-05\\
1563.93297440551	3.41411516320701e-05\\
1635.01845211566	2.79334551541847e-05\\
1706.1039298258	2.28544697396177e-05\\
1777.18940753595	1.86989681081702e-05\\
1848.2748852461	1.52990383191544e-05\\
1919.36036295624	1.25172989299172e-05\\
1990.44584066639	1.02413478045048e-05\\
2061.53131837653	8.37922026469719e-06\\
2132.61679608668	6.85567306028107e-06\\
2203.70227379682	5.60914400442273e-06\\
2274.78775150697	4.58926442169947e-06\\
2345.87322921712	3.7548238939257e-06\\
2416.95870692726	3.07210506497127e-06\\
2488.04418463741	2.51352121879537e-06\\
2559.12966234755	2.05650157911955e-06\\
2630.2151400577	1.68257928888624e-06\\
2701.30061776784	1.37664521736035e-06\\
2772.38609547799	1.12633744335198e-06\\
2843.47157318814	9.21541745322893e-07\\
2914.55705089828	7.5398291460988e-07\\
2985.64252860843	6.16890377900808e-07\\
3056.72800631857	5.04724617723606e-07\\
3127.81348402872	4.12953336382241e-07\\
3198.89896173886	3.37868318764292e-07\\
3269.98443944901	2.76435593969736e-07\\
3341.06991715916	2.26172841220758e-07\\
3412.1553948693	1.85049086375869e-07\\
3483.24087257945	1.51402636071235e-07\\
3554.32635028959	1.23873933442495e-07\\
3625.41182799974	1.01350622318729e-07\\
3696.49730570989	8.29226000897283e-08\\
3767.58278342003	6.78452430614264e-08\\
3838.66826113018	5.55093183412397e-08\\
3909.75373884032	4.54163664786243e-08\\
3980.83921655047	3.71585601437354e-08\\
4051.92469426061	3.04022249909726e-08\\
4123.01017197076	2.48743568326219e-08\\
4194.09564968091	2.03515903201271e-08\\
4265.18112739105	1.66511733889376e-08\\
4336.2666051012	1.36235827700531e-08\\
4407.35208281134	1.11464821821984e-08\\
4478.43756052149	9.11977907244596e-09\\
4549.52303823163	7.46158016230909e-09\\
4620.60851594178	6.10488237448411e-09\\
4691.69399365193	4.99486543005299e-09\\
4762.77947136207	4.08667671773885e-09\\
4833.86494907222	3.34361892811427e-09\\
4904.95042678236	2.73566722024191e-09\\
4976.03590449251	2.23825600369084e-09\\
5047.12138220265	1.8312863132582e-09\\
5118.2068599128	1.49831366724663e-09\\
5189.29233762295	1.22588359297234e-09\\
5260.37781533309	1.0029879699892e-09\\
5331.46329304324	8.20620223412813e-10\\
5402.54877075338	6.7141139397848e-10\\
5473.63424846353	5.49332379464591e-10\\
5544.71972617367	4.49450315908553e-10\\
5615.80520388382	3.67729254676709e-10\\
5686.89068159397	3.00867081318511e-10\\
5757.97615930411	2.4616208656203e-10\\
5829.06163701426	2.01403797966262e-10\\
5900.1471147244	1.64783661049337e-10\\
5971.23259243455	1.34821960772415e-10\\
6042.3180701447	1.10308030485355e-10\\
6113.40354785484	9.02513323485759e-11\\
6184.48902556499	7.38414325308304e-11\\
6255.57450327513	6.04152538950437e-11\\
6326.65998098528	4.94302829468898e-11\\
6397.74545869542	4.04426484154857e-11\\
6468.83093640557	3.30891856843299e-11\\
6539.91641411572	2.7072762347404e-11\\
6611.00189182586	2.21502719381248e-11\\
6682.08736953601	1.81228106920507e-11\\
6753.17284724615	1.48276404144098e-11\\
6824.2583249563	1.21316126949049e-11\\
6895.34380266644	9.92578876111317e-12\\
6966.42928037659	8.12103757414035e-12\\
7037.51475808674	6.6444342981568e-12\\
};
\addlegendentry{Exponential fit}

\end{axis}
\end{tikzpicture}%
\caption{An exponential distribution fitted over the distribution of the time length of a certain event. The hypothesis that this distribution is exponential is not rejected with $p=0.9588$.}
\end{figure}
\begin{figure}[H]\label{figure:transitionFitBad}
\centering
\setlength\fwidth{0.5\textwidth}
% This file was created by matlab2tikz.
%
%The latest updates can be retrieved from
%  http://www.mathworks.com/matlabcentral/fileexchange/22022-matlab2tikz-matlab2tikz
%where you can also make suggestions and rate matlab2tikz.
%
\definecolor{mycolor1}{rgb}{0.00000,0.44700,0.74100}%
\definecolor{mycolor2}{rgb}{0.85000,0.32500,0.09800}%
%
\begin{tikzpicture}

\begin{axis}[%
width=0.951\fwidth,
height=0.75\fwidth,
at={(0\fwidth,0\fwidth)},
scale only axis,
xmin=0,
xmax=2000,
xlabel style={font=\color{white!15!black}},
xlabel={Transition time (s)},
ymin=0,
ymax=0.001,
ylabel style={font=\color{white!15!black}},
ylabel={Probability density},
axis background/.style={fill=white},
legend style={legend cell align=left, align=left, draw=white!15!black}
]
\addplot [color=mycolor1]
  table[row sep=crcr]{%
0.0524647823054307	0.0288351023507269\\
71.1379424924512	0.00477975484886509\\
142.223420202597	0.0021737804944533\\
213.308897912743	0.00106094057171538\\
284.394375622888	0.000588968118761183\\
355.479853333034	0.000345845919894974\\
426.56533104318	0.000198918518897595\\
497.650808753326	0.000113933644900919\\
568.736286463471	6.75785315548609e-05\\
639.821764173617	4.29349634419453e-05\\
710.907241883763	2.97223950998137e-05\\
781.992719593909	2.23108103002681e-05\\
853.078197304054	1.77656872203651e-05\\
924.1636750142	1.46309199441671e-05\\
995.249152724346	1.22133086801677e-05\\
1066.33463043449	1.02039805604726e-05\\
1137.42010814464	8.4748935315763e-06\\
1208.50558585478	6.97378035674813e-06\\
1279.59106356493	5.67959141940436e-06\\
1350.67654127507	4.57783797498582e-06\\
1421.76201898522	3.65322321026178e-06\\
1492.84749669537	2.8904778709422e-06\\
1563.93297440551	2.27102706775509e-06\\
1635.01845211566	1.77410838811837e-06\\
1706.1039298258	1.38167449807944e-06\\
1777.18940753595	1.07601036165363e-06\\
1848.2748852461	8.41094155351757e-07\\
1919.36036295624	6.63607127951816e-07\\
1990.44584066639	5.31670086932575e-07\\
2061.53131837653	4.35320021598002e-07\\
2132.61679608668	3.66958603221902e-07\\
2203.70227379682	3.19580781439991e-07\\
2274.78775150697	2.88158027801596e-07\\
2345.87322921712	2.68327692431458e-07\\
2416.95870692726	2.56607895074816e-07\\
2488.04418463741	2.50663596411475e-07\\
2559.12966234755	2.48273684177677e-07\\
2630.2151400577	2.48177001435067e-07\\
2701.30061776784	2.4889869847913e-07\\
2772.38609547799	2.49698883149228e-07\\
2843.47157318814	2.50090072777687e-07\\
2914.55705089828	2.49497761418247e-07\\
2985.64252860843	2.47795409273047e-07\\
3056.72800631857	2.44923664754111e-07\\
3127.81348402872	2.40781584904165e-07\\
3198.89896173886	2.35424172152641e-07\\
3269.98443944901	2.28832649236764e-07\\
3341.06991715916	2.21341984715854e-07\\
3412.1553948693	2.13026807531044e-07\\
3483.24087257945	2.04111932183557e-07\\
3554.32635028959	1.94662013206055e-07\\
3625.41182799974	1.84833293669467e-07\\
3696.49730570989	1.74771187638821e-07\\
3767.58278342003	1.64607876355666e-07\\
3838.66826113018	1.54411402731886e-07\\
3909.75373884032	1.44395769432139e-07\\
3980.83921655047	1.34582334427765e-07\\
4051.92469426061	1.24990569055811e-07\\
4123.01017197076	1.15799506252128e-07\\
4194.09564968091	1.06984932961904e-07\\
4265.18112739105	9.85815061763463e-08\\
4336.2666051012	9.06131443865642e-08\\
4407.35208281134	8.30943982918764e-08\\
4478.43756052149	7.60317764642685e-08\\
4549.52303823163	6.94249988485161e-08\\
4620.60851594178	6.32681595574546e-08\\
4691.69399365193	5.75507875424562e-08\\
4762.77947136207	5.22587994764125e-08\\
4833.86494907222	4.73753437118977e-08\\
4904.95042678236	4.28815376200362e-08\\
4976.03590449251	3.87571031305619e-08\\
5047.12138220265	3.49809070312881e-08\\
5118.2068599128	3.15314136866006e-08\\
5189.29233762295	2.83870584271616e-08\\
5260.37781533309	2.55265500590058e-08\\
5331.46329304324	2.29291108363107e-08\\
5402.54877075338	2.05746619196872e-08\\
5473.63424846353	1.84439618680429e-08\\
5544.71972617367	1.65187051412403e-08\\
5615.80520388382	1.47815869658128e-08\\
5686.89068159397	1.32163402701495e-08\\
5757.97615930411	1.18077497536913e-08\\
5829.06163701426	1.05416475348633e-08\\
5900.1471147244	9.4048942372079e-09\\
5971.23259243455	8.38534883045835e-09\\
6042.3180701447	7.4718300476749e-09\\
6113.40354785484	6.65407175288855e-09\\
6184.48902556499	5.92267423582479e-09\\
6255.57450327513	5.26905305966352e-09\\
6326.65998098528	4.68538678185137e-09\\
6397.74545869542	4.16456460355452e-09\\
6468.83093640557	3.70013477686744e-09\\
6539.91641411572	3.28625440670189e-09\\
6611.00189182586	2.91764112267119e-09\\
6682.08736953601	2.58952696170862e-09\\
6753.17284724615	2.29761469123187e-09\\
6824.2583249563	2.03803671225656e-09\\
6895.34380266644	1.80731660910587e-09\\
6966.42928037659	1.60233335466341e-09\\
7037.51475808674	1.42028813516673e-09\\
};
\addlegendentry{Observed}

\addplot [color=mycolor2]
  table[row sep=crcr]{%
0.0524647823054307	0.00910370674229288\\
71.1379424924512	0.00476468301045731\\
142.223420202597	0.00249373193060728\\
213.308897912743	0.00130516530230486\\
284.394375622888	0.000683095261937675\\
355.479853333034	0.000357517270845061\\
426.56533104318	0.000187116799185415\\
497.650808753326	9.79328815490109e-05\\
568.736286463471	5.12559499213589e-05\\
639.821764173617	2.68262544794627e-05\\
710.907241883763	1.40402807966887e-05\\
781.992719593909	7.34837899196036e-06\\
853.078197304054	3.84598246939758e-06\\
924.1636750142	2.01290395760705e-06\\
995.249152724346	1.05351035133158e-06\\
1066.33463043449	5.51384509016622e-07\\
1137.42010814464	2.88582714350393e-07\\
1208.50558585478	1.51037944773544e-07\\
1279.59106356493	7.90499902697484e-08\\
1350.67654127507	4.13730534470428e-08\\
1421.76201898522	2.16537604329967e-08\\
1492.84749669537	1.13331093990871e-08\\
1563.93297440551	5.93150409367033e-09\\
1635.01845211566	3.10442082347339e-09\\
1706.1039298258	1.62478664720126e-09\\
1777.18940753595	8.5037815394171e-10\\
1848.2748852461	4.45069514786417e-10\\
1919.36036295624	2.3293974812739e-10\\
1990.44584066639	1.2191562094225e-10\\
2061.53131837653	6.38079964850214e-11\\
2132.61679608668	3.33957238946527e-11\\
2203.70227379682	1.74785988572716e-11\\
2274.78775150697	9.14792022407186e-12\\
2345.87322921712	4.78782338958295e-12\\
2416.95870692726	2.5058431040443e-12\\
2488.04418463741	1.31150402827061e-12\\
2559.12966234755	6.86412813872501e-13\\
2630.2151400577	3.5925360570158e-13\\
2701.30061776784	1.88025559257057e-13\\
2772.38609547799	9.84085068955323e-14\\
2843.47157318814	5.15048819302717e-14\\
2914.55705089828	2.69565401034619e-14\\
2985.64252860843	1.41084694715602e-14\\
3056.72800631857	7.38406746807929e-15\\
3127.81348402872	3.86466104513014e-15\\
3198.89896173886	2.02267991974772e-15\\
3269.98443944901	1.05862687826298e-15\\
3341.06991715916	5.5406238843791e-16\\
3412.1553948693	2.89984258462461e-16\\
3483.24087257945	1.51771482617877e-16\\
3554.32635028959	7.94339080961189e-17\\
3625.41182799974	4.15739877254091e-17\\
3696.49730570989	2.17589250839957e-17\\
3767.58278342003	1.13881503005682e-17\\
3838.66826113018	5.96031130985066e-18\\
3909.75373884032	3.11949789673577e-18\\
3980.83921655047	1.63267766092285e-18\\
4051.92469426061	8.54508139680366e-19\\
4123.01017197076	4.4723105990638e-19\\
4194.09564968091	2.34071054044964e-19\\
4265.18112739105	1.22507721966336e-19\\
4336.2666051012	6.41178893418323e-20\\
4407.35208281134	3.35579151066183e-20\\
4478.43756052149	1.75634862261174e-20\\
4549.52303823163	9.19234843508431e-21\\
4620.60851594178	4.81107615334044e-21\\
4691.69399365193	2.51801309716443e-21\\
4762.77947136207	1.31787353918506e-21\\
4833.86494907222	6.89746477982967e-22\\
4904.95042678236	3.60998373321996e-22\\
4976.03590449251	1.88938732854747e-22\\
5047.12138220265	9.88864421860335e-23\\
5118.2068599128	5.17550229138525e-23\\
5189.29233762295	2.70874584786278e-23\\
5260.37781533309	1.41769893146931e-23\\
5331.46329304324	7.41992927049625e-24\\
5402.54877075338	3.88343033609456e-24\\
5473.63424846353	2.03250336027671e-24\\
5544.71972617367	1.06376825435491e-24\\
5615.80520388382	5.5675327337182e-25\\
5686.89068159397	2.91392609378263e-25\\
5757.97615930411	1.52508582995914e-25\\
5829.06163701426	7.98196904755004e-26\\
5900.1471147244	4.1775897870453e-26\\
5971.23259243455	2.18646004824858e-26\\
6042.3180701447	1.14434585162284e-26\\
6113.40354785484	5.98925843248482e-27\\
6184.48902556499	3.13464819400711e-27\\
6255.57450327513	1.64060699850538e-27\\
6326.65998098528	8.58658183298116e-28\\
6397.74545869542	4.49403102886004e-28\\
6468.83093640557	2.35207854315005e-28\\
6539.91641411572	1.23102698615549e-28\\
6611.00189182586	6.44292872385759e-29\\
6682.08736953601	3.37208940238997e-29\\
6753.17284724615	1.76487858628714e-29\\
6824.2583249563	9.23699241819421e-30\\
6895.34380266644	4.83444184754223e-30\\
6966.42928037659	2.53024219563413e-30\\
7037.51475808674	1.32427398455981e-30\\
};
\addlegendentry{Exponential fit}

\end{axis}
\end{tikzpicture}%
\caption{The estimated probability density function for the time length of a certain event. The hypothesis that this distribution is exponential is rejected at $p=5\times10^{-4}$.}
\end{figure}