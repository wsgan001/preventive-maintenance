\chapter{Introduction}\label{chapter:Introduction}
When an essential asset within an organization fails, this can have big consequences for the organization.
For instance, if the machine of a manufacturer breaks, the production may stop until it is (correctively) repaired.
Hence, it might be efficient to occasionally inspect and repair the machine before the machine breaks.
This motivates looking for preventive maintenance policies to plan such repairs.
%The field of maintenance theory concerns the optimal planning of such repairs.

Preventive maintenance problems can be classified based on various aspects.
First of all, there is the distinction between perfect and imperfect maintenance.
For perfect maintenance, the asset has the same lifetime distribution after maintenance as a new asset.
For imperfect maintenance, this is not always the case.
\cite{Pham1996} summarizes results for various preventive maintenance problems with imperfect maintenance.
Although imperfect maintenance might be more realistic in practice, we will assume perfect maintenance for simplicity.

Another distinction can be made based on the options for moments at which maintenance can be scheduled.
For simplicity, we assume that the machine is continuously monitored and we can decide to immediately repair the asset at any given time.
However, in practice, it might be that maintenance can only be done at some discrete planned or unplanned moments \cite{Kalosi2016}.

The goal of preventive maintenance is usually to optimize a certain goal function.
Chapter 4 of \cite{Zacks2012} discusses preventive maintenance aiming at maximizing the availability of assets.
In our problem definition, the cost of performing corrective maintenance exceeds that of preventive maintenance and we aim at minimizing the total (discounted) maintenance cost.

The solution to a preventive maintenance problem is a maintenance policy that prescribes when preventive maintenance should be performed.
These policies can be classified as either age-based or condition-based.
In age-based maintenance, the decision to perform preventive maintenance is done based only on the age of the machine.
Often, more aspects are observed that help predict the fitness of the asset.
When the decision to do preventive maintenance is based on other quantities than the age of the machine, this is called condition-based maintenance.
\cite{Kalosi2016} models the condition of the asset as a CTMC with a failure state and states corresponding to a perfect condition and a satisfactory condition.
The decision to perform preventive maintenance is then done based on the state the asset is in.
In this thesis, we will opt for an age-based maintenance policy that is dynamically adapted by observations of the asset.
Hence, this could be viewed as a hybrid between age-based maintenance and condition-based maintenance.

Various mathematical models have been developed to model the degradation of assets.
\cite{Derman1963} models the deterioration of the asset as a CTMC where there is a drift towards the failure state.
In this thesis, we will model the degradation of the asset as a Markov Modulated Fluid Model (also known as a Markov modulated fluid queue or stochastic Fluid model) with jumps.

The research is motivated by the real-world problem of deciding when to repair a Philips manufacturing machine.
Usage data of this machine will be analyzed in this thesis.

The remainder of this thesis is organized as follows:
In chapter \ref{chapter:literatureOverview}, some concepts from the fields of dynamic programming and survival analysis are summarized.  
In chapter \ref{chapter:AgeBased}, a simple age-based preventive maintenance problem is addressed and methods to find the optimal maintenance policy are introduced.
This problem is extended in chapter \ref{chapter:SimpleFluid} to include jumps that instantaneously decrease the age of the machine by a constant.
We prove that this problem is equivalent to the age-based problem with an adjusted discount exponent.
In chapter \ref{chapter:Mmfm}, we change the degradation model to a MMFM with jumps.
Results from previous chapters are extended, resulting in a method to find the optimal preventive maintenance policy for this problem.
In chapter \ref{chapter:DataAnalysis}, the data of the Philips machine is analysed and in chapter \ref{chapter:ParameterEstimation}, a method is proposed to estimate the parameters of an MMFM given usage data.
Finally, the results are summarized and some directions for further research are presented.

