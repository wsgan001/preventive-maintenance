\documentclass[a4paper]{thesis}
\usepackage[english]{babel}
\usepackage[utf8x]{inputenc}
\usepackage{amsfonts}
\usepackage{amsmath} % Equation numbering
\usepackage[square,sort,comma,numbers]{natbib}
\usepackage[toc,page]{appendix} % For appendices

\usepackage{graphicx} % Figures
\usepackage{grffile} % For recognising the png-extension
\usepackage{float} % For floating figures

\usepackage{changes} % For track changed extensions
\usepackage{todonotes}

\usepackage{dsfont} % For indicator function

\usepackage{hyperref}
\bibliographystyle{plainnat}

% For example environment
\usepackage{amsthm}
\theoremstyle{definition}
\let\example\relax
\newtheorem{example}{Example}[chapter]
\newtheorem{remark}{Remark}[chapter]
\newtheorem{corollary}{Corollary}[chapter]

% For theorem environment
\newtheorem{theorem}{Theorem}[section]

% For lemma environment
\newtheorem{lemma}{Lemma}
\begin{document}
\chapter{Iteration Methods}
\section{Convergence of Value Iteration}
In this section, the convergence of value iteration for the simple discounted problem is proven.
The proof is done similarly as in Bertsekas volume 1 p.298-299.[Will be replaced by proper citation in report]
The Bellman equations that we are going to use, are
\begin{equation}
V(x_k)=\begin{cases}
\min\{c+\alpha_\delta V(1),\alpha_\delta \mathbb{E}[V(S(x_k))]\},&\text{if}\ x_k>0 \\
c+a+\alpha_\delta V(1),&\text{else.}
\end{cases}
\end{equation}
Where $\mathbb{E}[V(S(x_k))]=\mathbb{P}(\omega_k=0)V(0)+\mathbb{P}(\omega_k=1)V(x_k+1)$.

We define an operator $T$. Such that at each iteration the value function is updated as follows
\begin{equation}
\begin{split}
V_{k+1}(x)=TV_k=\begin{cases}
\min\{c+\alpha_\delta V_k(1),\alpha_\delta\mathbb{P}(\omega(x)=0)V_k(0)+\alpha_\delta\mathbb{P}(\omega(x)=1)V_k(x+1)\},&\text{if}\ x>0 \\
c+a+\alpha_\delta V_k(1),&\text{else.}
\end{cases}
\end{split}
\end{equation}

And the corresponding policy is given by
\begin{equation}
\begin{split}
\mu_{k+1}(x)=\begin{cases}
1,&\text{if}\ x=0\ \text{or}\\&c+\alpha_\delta V_k(1)<\alpha_\delta\mathbb{P}(\omega(x)=0)V_k(0)+\alpha_\delta\mathbb{P}(\omega(x)=1)V_k(x+1)\} \\
x+1,&\text{else.}
\end{cases}
\end{split}
\end{equation}
For this $T$, it holds that for every $V_a,V_b$, the following property holds [Bertsekas]
$$
\forall_x [V_a\leq V_b]\Rightarrow \forall_x [TV_a\leq TV_b].
$$
Also, defining $e(x)=1$, we have [Bertsekas]
$$
T(V+Ce)(x_0) = TV(x_0)+\alpha_\delta C.
$$

Furthermore, for every $\pi=\{\mu_0,\mu_1,...\}$ with corresponding $x_m$, we have
\begin{equation}
\begin{split}
V_\pi(x_0)&=\sum\limits_{m=0}^{\infty}\alpha_\delta^mg(x_m,\mu_m(x_m))\\
&\leq \sum\limits_{k=0}^\infty\alpha_\delta^k(c+a)\\
&=\frac{c+a}{1-\alpha_\delta}.
\end{split}
\end{equation}
Moreover, for the optimal cost $V^*$ we have that for all $x_0$ $V^*(x_0)\leq V_\pi(x_0)\leq \frac{c+a}{1-\alpha_\delta}$. Also, $TV^*=V^*$ because $V^*$ satisfies the Bellman equations.

Now we choose an initial value $V_0$ such that there exists an $M$ such that for each $x$, $|V_0(x)|\leq M$ holds.
The following inequality now holds.
We can now write
\begin{equation}V^*(x_0)-\frac{c+a}{1-\alpha_\delta}\leq V_0(x_0)\leq M + V^*(x_0).
\end{equation}
If we apply $T$ $k$ times to this equation and let $k\rightarrow\infty$, we get
\begin{equation}
\begin{split}
&T^k(V^*(x_0)-\frac{c+a}{1-\alpha_\delta})\\
&=V^*(x_0)-\alpha_\delta^k\frac{c+a}{1-\alpha_\delta}\\
&\leq T^k V_0(x_0)\\
&=V^k(x_0)\\
&\leq T^k(M + V^*(x_0))\\
&\leq \alpha_\delta^kM+V^*(x_0).
\end{split}
\end{equation}
And in the limit, we get $V^k(x_0)\rightarrow V^*(x_0)$. Such that the convergence is proven for all bounded positive $V_0$.
\section{Convergence of custom iteration}
The convergence of the iteration method for the simple discounted problem will now be proven.
The following equation will be solved
\begin{equation}\label{eq:notBellman}
V(\mu^*)=\inf\limits_{\mu>0}\mathbb{P}(Q>\mu)e^{-\beta \mu}(c+V(\mu^*))+\mathbb{P}(Q\leq \mu)\mathbb{E}[e^{-\beta Q}|Q\leq \mu](c+a+V(\mu^*))]. 
\end{equation}
Note that these are not the Bellman equations since the discount depends on the chosen action.
Let $\alpha_\mu=\mathbb{P}(Q>\mu)e^{-\beta \mu} \mathbb{P}(Q\leq \mu)\mathbb{E}[e^{-\beta Q}|Q\leq \mu]$ denote the discount when choosing control limit $\mu$.
Note that this is decreasing in $\mu$ since $\frac{d}{d\mu}\alpha_\mu=-\beta\mathbb{P}(Q>\mu)e^{-\beta \mu}<0$.
Since $\lim\limits_{\mu\rightarrow 0}V(\mu)=\infty$, we know that for every $B>0$ for sufficiently small $\varepsilon$, we have $\mu<\varepsilon\Rightarrow V(\mu)>B$.


The cost that is incurred when a control $\mu$ is chosen equals
$$
g(\mu):=\mathbb{P}(Q>\mu)e^{-\beta \mu}c+\mathbb{P}(Q\leq \mu)\mathbb{E}[e^{-\beta Q}|Q\leq \mu](c+a).
$$
Note that $g(\mu)<c+a$ for all $\mu$ so that 
$$V(\mu^*)=\sum\limits_{n=0}^\infty \alpha_{\mu^*}^kg(\mu^*)\leq \sum\limits_{n=0}^\infty \alpha_\varepsilon^k(c+a)=\frac{c+a}{1-\alpha_\varepsilon}$$

The iteration is given by
\begin{equation}\label{eq:Miteration}
V_{n+1}=TV_n=\inf\limits_{\mu_{n+1}>0}g(\mu_{n+1})+\alpha_{\mu_{n+1}}V_n
\end{equation}
By $\mu(V)$ we will denote the $\mu$ at which $TV$ is attained.
We will prove that $T(A_1+A_2)\leq TA_1+\alpha_\varepsilon A_2$,$T(A_1)\leq T(A_2)$ and $T(A_1-A_2)\geq TA_1-\alpha_\varepsilon A_2$ for $A_1\geq A_2$:
\begin{itemize}
\item $T(A_1+A_2)=g(\mu(A_1+A_2))+\alpha_{\mu(A_1+A_2)}(A_1+A_2)\leq g(\mu(A_1))+\alpha_{\mu(A_1)}(A_1+A_2)\leq g(\mu(A_1))+\alpha_{\mu(A_1)}A_1+\alpha_\varepsilon A_2=TA_1+\alpha_\varepsilon A_1$
\item $T(A_1)=g(\mu(A_1))+\alpha_{\mu(A_1)}A_1\leq g(\mu(A_2))+\alpha_{\mu(A_2)}A_1\leq g(\mu(A_2))+\alpha_{\mu(A_2)}A_2=T(A_2)$
\item $T(A_1-A_2)=g(\mu(A_1-A_2))+\alpha_{\mu(A_1-A_2)}(A_1-A_2)\geq g(\mu(A_1-A_2))+\alpha_{\mu(A_1-A_2)}A_1 - \alpha_\varepsilon A_2\geq g(\mu(A_1))+\alpha_{\mu(A_1)}A_1-\alpha_\varepsilon A_2=TA_1-\alpha_\varepsilon A_2$
\end{itemize}

If our initial $V_0<B$, then the following inequality now holds
\begin{equation}
V(\mu^*)-\frac{c+a}{1-\alpha_\varepsilon}\leq V_{0}\leq V_{\mu^*}+B.
\end{equation}
If we now apply $T$ $k$ times on this inequality, we get

\begin{equation}
V_{\mu^*}-\alpha_\varepsilon^k\frac{c+a}{1-\alpha_\varepsilon} \leq T^k(V_{\mu^*}-\frac{c+a}{1-\alpha_\varepsilon})\leq  T^kV_{0}=V_k\leq T^k(V_{\mu^*}+B) \leq V_{\mu^*}+\alpha_\varepsilon^kB.
\end{equation}
Concluding $\lim\limits_{k\rightarrow\infty}V_k=V_{\mu^*}$.
So that the convergence for value iteration is proven.
Note that the difficulty of this iteration still lies in finding the $\mu_{n+1}$ that minimizes \eqref{eq:Miteration}.
For increasing hazard rates, there is at most one $\mu$ such that

$$
h(\mu)= \beta\frac{c+V_n}{a}.
$$

And $\mu_{n+1}$ should be chosen as either this $\mu$ or $\infty$.

\end{document}