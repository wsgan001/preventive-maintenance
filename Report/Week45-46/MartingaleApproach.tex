\documentclass[a4paper]{thesis}
\usepackage[english]{babel}
\usepackage[utf8x]{inputenc}
\usepackage{amsfonts}
\usepackage{amsmath} % Equation numbering
\usepackage[square,sort,comma,numbers]{natbib}
\usepackage[toc,page]{appendix} % For appendices

\usepackage{graphicx} % Figures
\usepackage{grffile} % For recognising the png-extension
\usepackage{float} % For floating figures

\usepackage{changes} % For track changed extensions
\usepackage{todonotes}

\usepackage{dsfont} % For indicator function

\usepackage{hyperref}
\bibliographystyle{plainnat}

% For example environment
\usepackage{amsthm}
\theoremstyle{definition}
\let\example\relax
\newtheorem{example}{Example}[chapter]
\newtheorem{remark}{Remark}[chapter]
\newtheorem{corollary}{Corollary}[chapter]

% For theorem environment
\newtheorem{theorem}{Theorem}[section]

% For lemma environment
\newtheorem{lemma}{Lemma}

\begin{document}
\chapter{Martingale Approach}
In this section, the results of the martingale approach are summarized.
To quickly summarize: The first approach resulted in the correct optimal policy but the chosen random variable was never a martingale.
The analysis just seemed to involve minizing the expected total discounted cost just like earlier.
The approach similar to the notes from Glynn gave a martingale but the resulting conclusion was $\mu=\mu^*$, which isn't very useful.

\section{Definitions and notation}
We denote the time step by $\delta>0$.
We consider the martingale on time stages $x_i=i\delta$ ($i=0,1,...$).
Time is then divided into intervals $\Delta_i=(x_{i-1},x_i]$ ($i=1,2,...$).
The $n$'th lifetime is a positive random variable $Q_n$.
These $Q_n$'s are i.i.d. random variables.
When the machine is repaired preventively, a cost $c$ is paid.
When the machine breaks and is repaired correctively, a cost $c+a$ is paid.
Costs that occur at a time $t$ in the future, are discounted by a discount factor $e^{-\beta t}$ for some $\beta>0$.
Let $v^*$ be the expected total discounted cost for the optimal control limit.
When a machine breaks in an interval $\Delta_i$, it is repaired at the end of the interval (i.e., at $x_i$).
When a machine happens to break exactly at the time that preventive maintenance is scheduled, the cost for corrective maintenance is paid.

\section{Approach from the last meeting}
We consider the cost of one run using control limit $\mu$, with terminal cost $v^*$.
This cost is represented by the following random variable
$$
V_\mu=(c+\mathds{1}\{Q_0\geq\mu\}a+v^*)e^{-\beta Q_0\wedge\mu}.
$$
We consider the following sequence of random variables
\begin{equation}
M_n^\mu=\mathds{1}\{Q_0\wedge\mu>x_n\}e^{\beta x_n}\mathbb{E}[V_\mu|Q_0\wedge\mu>x_n].
\end{equation}
We are now going to minimize
\begin{equation}
\begin{split}
g_n(\mu)&=\mathbb{E}[M_{n+1}^\mu-M_n^\mu|M_n^\mu]\\
&=\mathbb{E}[M_{n+1}^\mu|M_n^\mu]-\mathbb{E}[M_n^\mu|M_n^\mu]\\
&=\mathbb{E}[M_{n+1}^\mu|M_n^\mu]-M_n^\mu
\end{split}
\end{equation}
Note that
\begin{equation}
\begin{split}
\mathbb{E}[M_{n+1}^\mu]&=\mathbb{E}[\mathds{1}\{Q_0\wedge\mu>x_{n+1}\}e^{\beta x_{n+1}}\mathbb{E}[V_\mu|Q_0\wedge\mu>x_{n+1}]]\\
&=\mathbb{P}(Q_0\wedge\mu>x_{n+1})e^{\beta x_{n+1}}\mathbb{E}[V_\mu|Q_0\wedge\mu>x_{n+1}]\\
&=\mathbb{P}(Q_0\wedge\mu>x_{n+1})e^{\beta x_{n+1}}\mathbb{E}[V_\mu|Q_0\wedge\mu>x_{n+1}]\\
&=\mathbb{P}(Q_0\wedge\mu>x_{n})e^{\beta x_{n+1}}\left(
\begin{split}
&\mathbb{E}[V_\mu|Q_0\wedge\mu>x_{n}]\\
&-\mathbb{P}(Q_0\in\Delta_{n+1}|Q_0>x_n)\mathbb{E}[V_\mu|Q_0\in\Delta_{n+1}]
\end{split}\right)\\
&=\mathbb{P}(Q_0>x_{n+1}|Q_0>x_n)e^{\beta(x_{n+1}-x_n)}(\mathbb{E}[M_n^\mu]-e^{\beta x_n}\mathbb{P}(Q_0\in\Delta_{n+1}|Q_0>x_n)\mathbb{E}[V_\mu|Q_0\in\Delta_{n+1}])\\
&=\begin{split}
&\mathbb{P}(Q_0>x_{n+1}|Q_0>x_n)e^{\beta(x_{n+1}-x_n)}\mathbb{E}[M_n^\mu]\\
&-e^{\beta x_{n+1}}\mathbb{P}(Q_0>x_{n+1}|Q_0>x_n)\mathbb{P}(Q_0\in\Delta_{n+1}|Q_0>x_n)\mathbb{E}[V_\mu|Q_0\in\Delta_{n+1}]
\end{split}
\end{split}
\end{equation}
So that for $g_n(\mu)$, we get
\begin{equation}
\begin{split}
g_n(\mu)&=\mathbb{E}[M_{n+1}^\mu|M_n^\mu]-M_n^\mu\\
&=\begin{split}
&\mathbb{P}(Q_0>x_{n+1}|Q_0>x_n)e^{\beta(x_{n+1}-x_n)}\mathbb{E}[M_n^\mu|M_n^\mu]\\
&-e^{\beta x_{n+1}}\mathbb{P}(Q_0>x_{n+1}|Q_0>x_n)\mathbb{P}(Q_0\in\Delta_{n+1}|Q_0>x_n)\mathbb{E}[V_\mu|Q_0\in\Delta_{n+1}]\\
&-M_n^\mu
\end{split}\\
&=\begin{split}
&(\mathbb{P}(Q_0>x_{n+1}|Q_0>x_n)e^{\beta(x_{n+1}-x_n)}-1)M_n^\mu\\
&-e^{\beta x_{n+1}}\mathbb{P}(Q_0>x_{n+1}|Q_0>x_n)\mathbb{P}(Q_0\in\Delta_{n+1}|Q_0>x_n)\mathbb{E}[V_\mu|Q_0\in\Delta_{n+1}].
\end{split}\\
\end{split}
\end{equation}
If we now take the derivative of $g_n(\mu)$ to $\mu$, the second term disappears as it does not depend on $\mu$.
The factor $(\mathbb{P}(Q_0>x_{n+1}|Q_0>x_n)e^{\beta(x_{n+1}-x_n)}-1)$ also does not depend on $\mu$.
So only the derivative of $M_n^\mu$ is of interest.
\begin{equation}
\begin{split}
\frac{d}{d\mu}M_n^\mu&=\frac{d}{d\mu}\mathds{1}\{Q_0\wedge\mu>x_n\}e^{\beta x_n}\mathbb{E}[V_\mu|Q_0\wedge\mu>x_n]\\
&=\mathds{1}\{Q_0\wedge\mu>x_n\}e^{\beta x_n}\frac{d}{d\mu}\mathbb{E}[V_\mu|Q_0\wedge\mu>x_n].
\end{split}
\end{equation}
We rewrite this expectation to make it easier to derive
\begin{equation}
\begin{split}
\frac{d}{d\mu}\mathbb{E}[V_\mu|Q_0>x_n]&=\frac{d}{d\mu}\mathbb{E}[(c+\mathds{1}\{Q_0\geq\mu\}a+v^*)e^{-\beta Q_0\wedge\mu}|Q_0>x_n]\\
&=\frac{d}{d\mu}((c+v^*)\mathbb{E}[e^{-\beta Q_0\wedge\mu}|Q_0>x_n]+a\mathbb{P}(Q_0leq\mu|Q_0>x_n)\mathbb{E}[e^{-\beta Q}|Q\in(x_n,\mu]])\\
&=-\beta\frac{\mathbb{P}(Q_0>\mu)}{\mathbb{P}(Q_0>x_n)}(c+v^*)e^{-\beta\mu}+a \frac{f(\mu)}{\mathbb{P}(Q_0>x_n)}e^{-\beta\mu}\\
&=0\\
&\Rightarrow f(\mu)=\beta\frac{(c+v^*)\mathbb{P}(Q_0>\mu)}{a}.
\end{split}
\end{equation}
As you can see, this solution is exactly the same as in the earlier approaches.
We also need to assure that $\frac{d^2}{d\mu^2}g_n(\mu)>0$.
Again, we only need to derive $\mathbb{E}[V_\mu|Q_0>x_n]$.
\begin{equation}
\begin{split}
\frac{d^2}{d\mu^2}\mathbb{E}[V_\mu|Q_0>x_n]&=\frac{d}{d\mu}(-\beta\frac{\mathbb{P}(Q_0>\mu)}{\mathbb{P}(Q_0>x_n)}(c+v^*)e^{-\beta\mu}+a \frac{f(\mu)}{\mathbb{P}(Q_0>x_n)}e^{-\beta\mu})\\
&=\frac{\beta^2\mathbb{P}(Q>\mu)+\beta f(\mu)}{\mathbb{P}(Q_0>x_n)}(c+v^*)e^{-\beta\mu}+\frac{-\beta f(\mu)+f'(\mu)}{\mathbb{P}(Q_0>x_n)}ae^{-\beta\mu}.
\end{split}
\end{equation}
We now multiply by $\frac{1}{a}\mathbb{P}(Q_0>x_n)e^{\beta\mu}$ to get
$$
(\beta^2\mathbb{P}(Q>\mu)+\beta f(\mu))\frac{c+v^*}{a}+(-\beta f(\mu)+f'(\mu))=\beta^2\mathbb{P}(Q>\mu)\frac{c+v^*}{a}+\beta f(\mu)(\frac{c+v^*}{a}-1)+f'(\mu).
$$
Now we substitute $f(\mu)=\beta\frac{(c+v^*)\mathbb{P}(Q_0>\mu)}{a}$ and get
\begin{equation}
\begin{split}
&\beta^2\mathbb{P}(Q>\mu)\frac{c+v^*}{a}+\beta^2\mathbb{P}(Q_0>\mu)\frac{(c+v^*)}{a}(\frac{c+v^*}{a}-1)+f'(\mu)\\
&=\beta^2\mathbb{P}(Q_0>\mu)(\frac{c+v^*}{a})^2+f'(\mu)
\end{split}
\end{equation}
For the final steps, we need the following simple lemma:
\begin{lemma}
Let $Q$ be a random variable with increasing failure rate.
Then 
$$f'(x)>\frac{f(x)^2}{\mathbb{P}(Q_0>\mu)}$$
\end{lemma}
\begin{proof}
The failure rate is increasing, so its derivative is positive.
Hence
\begin{equation}
\begin{split}
\frac{d}{dx}\frac{f(x)}{\mathbb{P}(Q>x)}&=\frac{f'(x)\mathbb{P}(Q>x)+f(x)^2}{\mathbb{P}(Q>x)^2}>0\\
&\Rightarrow f'(x)>-\frac{f(x)^2}{\mathbb{P}(Q>x)}
\end{split}
\end{equation}
\end{proof}
We now apply this lemma
$$
\beta^2\mathbb{P}(Q_0>\mu)(\frac{c+v^*}{a})^2+f'(\mu)>\beta^2\mathbb{P}(Q_0>\mu)(\frac{c+v^*}{a})^2-\frac{f(\mu)^2}{\mathbb{P}(Q>\mu)}.
$$
We multiply by $\mathbb{P}(Q>\mu)$ and substitute $f(\mu)=\beta\frac{(c+v^*)\mathbb{P}(Q_0>\mu)}{a}$ again
\begin{equation}
\left(\beta\mathbb{P}(Q_0>\mu)\frac{c+v^*}{a}\right)^2-f(\mu)^2=\left(\beta\mathbb{P}(Q_0>\mu)\frac{c+v^*}{a}\right)^2-\left(\beta\mathbb{P}(Q_0>\mu)\frac{c+v^*}{a}\right)^2=0.
\end{equation}
Since we only multiplied by positive values, we conclude that $\frac{d^2}{d\mu^2}g_n(\mu)>0$ and the found solution is indeed optimal.

However, the sequence of random variables is not a martingale for any policy.
This can easily be seen by the fact that $\mathbb{E}[M_0^\mu]>0$ while for any $n$ such that $x_n>\mu$, we have $\mathbb{E}[M_n^\mu]=0$.
Hence we proceed with a definition of a martingale similar to the notes of Glynn.
\section{Approach similar to Glynn chapter 11}
In the notes from Glynn, the martingale is taken to be
\begin{equation}
\sum\limits_{j=0}^{T\wedge n-1} r(X_j,A_j) +\mathds{1}\{T>n-1\}V^*(X_n),
\end{equation}
i.e. the cost of using controls $(A_j: j\geq 0)$ up until stage $n-1$ and using the expected value of the rest of the cost using the optimal policy.
Which is of course a supermartingale for every policy and a martingale for every optimal policy.

In our approach, we define the martingale $M_n^\mu$ to be the total discounted cost of having used control limit $\mu$ up until time $x_n$ and taking the expected discounted cost of using optimal control limit $\mu^*$ for the rest of time.
Let $R_0=0$ and $R_{n+1}=R_n+Q_n\wedge \mu$ be the time of the $n$'th repair.
For convenience, we define the following random variables
\begin{itemize}
\item $R^-(x)=\max\{R_i|R_i\leq x\}$ to be the time of the last repair at time $x$.
\item $R^+(x)=\min\{R_i|R_i> x\}$ to be the time of the next repair at time $x$.
\item $K(x)=\max\{i|R_i\leq x\}$ to be the number of repairs that have occurred before time $x$.
\item $Q(x)=Q_{K(x)}$ to be the (total) lifetime of the current machine.
\end{itemize}
We denote expectations and probabilities conditioned to the observations up to time $x$ by a subscript $x$.
For example
$$
\mathbb{E}_x[X]=\mathbb{E}[X|R_0,...,R_{K(x)}].
$$
Furthermore, let 
$$V^*(x)=\mathbb{E}_x[(c+a\mathds{1}\{Q(x)\geq \mu^*a+v^*)e^{-\beta(R^-(x)+Q(x)\wedge\mu^*}\})]$$
be the expected discounted cost of all costs after $x$, using the optimal control limit.
We then arrive at the following definition of the supermartingale
\begin{equation}\label{eq:martingaleDef}
M_n^\mu=\sum\limits_{k=0}^{K(x_n)-1}(c+a\mathds{1}\{Q_k\geq\mu\})e^{-\beta R_{k+1}} + V^*(x_n).
\end{equation}
This is a martingale for $\mu=\mu^*$.

When we try to minimize
\begin{equation}\label{eq:gDef}
\begin{split}
g_n(\mu)&=e^{\beta x_{n+1}}\mathbb{E}_{x_n}[M_{n+1}^\mu-M_n^\mu]\\
&=\mathbb{E}_{x_n}[\sum\limits_{k=K(x_n)}^{K(x_{n+1})-1}(c+a\mathds{1}\{Q_k\geq\mu\})e^{-\beta R_{k+1}}-(V^*(x_n)-V^*(x_{n+1})].
\end{split}
\end{equation}
We neglect the possibility of two repairs within an interval of time $\delta$, i.e. we assume that
\begin{equation}
\mathbb{P}(K(x_{n+1})-K(x_n)>1)=o(\delta^*).
\end{equation}
Note that $V^*(x_n)-V^*(x_{n+1})$ equals the expected discounted cost in the interval $(x_n,x_{n+1}]$ so that
\begin{equation}
\begin{split}
e^{\beta x_{n+1}}(V^*(x_n)-V^*(x_{n+1}))&=\mathbb{E}_{x_n}[\mathds{1}\{R^-(x_n)+Q(x)\wedge\mu^*\in\Delta_{n+1}\}c\\
&+\mathds{1}\{R^-(x_n)+Q(x)\in\Delta_{n+1}\}a]+o(\delta^2)\\
&=\mathbb{P}_{x_n}(Q(x)> x_{n+1}-R^-(x_n))\mathds{1}\{\mu^*=x_{n+1}-R^-(x_n)\}c \\
&+ \mathbb{P}_{x_n}(Q(x)\leq x_{n+1}-R^-(x_n))(c+a)+o(\delta^2).
\end{split}
\end{equation}
Similarly, we can rewrite the other part of \eqref{eq:gDef} to
\begin{equation}
\begin{split}
&e^{\beta x_{n+1}}\mathbb{E}_{x_n}[\sum\limits_{k=K(x_n)}^{K(x_{n+1})-1}(c+a\mathds{1}\{Q_k\geq\mu\})e^{-\beta R_{k+1}}]\\
&=\mathbb{P}_{x_n}(Q(x)> x_{n+1}-R^-(x_n))\mathds{1}\{\mu=x_{n+1}-R^-(x_n)\}c\\
&+\mathbb{P}_{x_n}(Q(x)\leq x_{n+1}-R^-(x_n))(c+a)+o(\delta^2).
\end{split}
\end{equation}
Combining these, results in
\begin{equation}
\begin{split}
g_n(\mu)&=\mathbb{P}_{x_n}(Q(x)> x_{n+1}-R^-(x_n))\mathds{1}\{\mu=x_{n+1}-R^-(x_n)\}c\\
&+\mathbb{P}_{x_n}(Q(x)\leq x_{n+1}-R^-(x_n))(c+a)\\
&-\mathbb{P}_{x_n}(Q(x)> x_{n+1}-R^-(x_n))\mathds{1}\{\mu^*=x_{n+1}-R^-(x_n)\}c \\
&- \mathbb{P}_{x_n}(Q(x)\leq x_{n+1}-R^-(x_n))(c+a)\\
&=\mathbb{P}_{x_n}(Q(x)> x_{n+1}-R^-(x_n))(\mathds{1}\{\mu=x_{n+1}-R^-(x_n)\}-\mathds{1}\{\mu^*=x_{n+1}-R^-(x_n)\})c.
\end{split}
\end{equation}
$M_n^\mu$ is a martingale if $g_n(\mu)=0$ for all $n$.
Hence, we conclude that $\mu=\mu^*$, which isn't very helpful.
\end{document}