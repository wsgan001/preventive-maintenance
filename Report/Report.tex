\input{Sections/imports.tex}
\begin{document}
\title{Preventive Maintenance}
\author{Martijn G\"{o}sgens\\
\and
S\'{a}ndor Kolumb\'{a}n\\
\and
Stella Kapodistria
}
\begin{titlepage}

\newcommand{\HRule}{\rule{\linewidth}{0.5mm}} % Defines a new command for the horizontal lines, change thickness here

\center % Center everything on the page
 
%----------------------------------------------------------------------------------------
%	HEADING SECTIONS
%----------------------------------------------------------------------------------------

\textsc{\LARGE Eindhoven University of Technology}\\[1.5cm] % Name of your university/college
\textsc{\Large Final Bachelor Project}\\[0.5cm] % Major heading such as course name
\textsc{\large 2WH40}\\[0.5cm] % Minor heading such as course title

%----------------------------------------------------------------------------------------
%	TITLE SECTION
%----------------------------------------------------------------------------------------

\HRule \\[0.4cm]
{ \huge \bfseries Preventive Maintenance}\\[0.4cm] % Title of your document
\HRule \\[1.5cm]
 
%----------------------------------------------------------------------------------------
%	AUTHOR SECTION
%----------------------------------------------------------------------------------------

\begin{minipage}{0.4\textwidth}
\begin{flushleft} \large
\emph{Author:}\\
Martijn \textsc{G\"{o}sgens} % Your name
\end{flushleft}
\end{minipage}
~
\begin{minipage}{0.4\textwidth}
\begin{flushright} \large
\emph{Supervisor:} \\
S\'{a}ndor \textsc{Kolumb\'{a}n}\\ % Supervisor's Name
Stella \textsc{Kapodistria}
\end{flushright}
\end{minipage}\\[2cm]

% If you don't want a supervisor, uncomment the two lines below and remove the section above
%\Large \emph{Author:}\\
%John \textsc{Smith}\\[3cm] % Your name

%----------------------------------------------------------------------------------------
%	DATE SECTION
%----------------------------------------------------------------------------------------

{\large \today}\\[2cm] % Date, change the \today to a set date if you want to be precise

%----------------------------------------------------------------------------------------
%	LOGO SECTION
%----------------------------------------------------------------------------------------

%\includegraphics{logo.png}\\[1cm] % Include a department/university logo - this will require the graphicx package
 
%----------------------------------------------------------------------------------------

\vfill % Fill the rest of the page with whitespace

\end{titlepage}


\begin{abstract}
	In this thesis, we model the breakdown of a machine as a Markov modulated fluid model (MMFM) and find a replacement policy minimizing the total discounted cost.
	At each transition of the MMFM's underlying Markov chain, the fluid level can instantaneously increase by a constant amount, where the amount depends on the origin and destination state of the Markov chain.
	Numeric methods to compute the total discounted cost for a given stationary replacement policy and iteration methods to find the optimal replacement policy are presented.
\end{abstract}

\chapter*{Executive summary?}
[Describe characteristics of a machine whose deterioration that can be modeled by a MMFM with jumps: different activities that in different degrees wear out the machine, the machine has no schedule (actions as CTMC), partial repairs between activities and an initial fitness of the machine.]

[List assumptions]

[The type of observations for which the policy is useful: trace data where at each time the current activity is known and failures are observed.]

[Mention the kind of historical data needed for the parameter estimation.]

[The type of (online) policy that is presented and how it would be implemented.]

[Explain possibilities of forecasting time until preventive repair.]

\tableofcontents


\chapter{Introduction}
[Mention the structure of the report: Age based maintenance, simple fluid model with jumps, MMFM]
\section{Preventive maintenance}
[Mention reasons to do preventive maintenance (corrective maintenance is expensive)]
\section{Philips machine}

\chapter{Literature overview}
\section{Dynamic programming}
\subsubsection{Discounted vs. long run average cost}
[Explain relation and pros and cons of both]
\subsubsection{Stationary policies}
\section{Survival analysis}
\subsection{Classification of lifetime distributions}
[increasing and decreasing hazard rate. No preventive maintenance for decreasing hazard rates.]


%\chapter{Analysis of problems}
\chapter{Age-based maintenance}\label{chapter:AgeBased}
A machine is considered that is subject to deterioration over time.
If no further observations are made while the machine is active, any choice to repair the machine can only be based on its age.
In this chapter we will investigate methods to find an optimal preventive maintenance policy in terms of total discounted cost.
% Problem formulation and definitions
\section{Problem formulation and definition}\label{section:AgeBasedDefinition}
In this section, the problem of choosing when to repair the asset is defined as a Markov decision process.
Calendar time is discretized in steps of size $\delta$, i.e. the $k$'th decision epoch is at time $t_k=k\delta$.
We refer to the interval $(k\delta,(k+1)\delta]$ as the $k$'th time interval.
At epoch $k\in\mathbb{N}\cup\{0\}$, we denote the state of the asset by $x_k\in X=\mathbb{N}_0$ and initially $x_0=1$.
When $x_k=x$, this means that after the $k$'th time interval, the asset has age $x\delta$.

The lifetime of the asset has a distribution function $F$.
We denote the reliability function by $\bar F(x):=1-F(x)$ and the conditional reliability function by $\bar F(x;y)=\frac{\bar F(x)}{\bar F(y)}$.
We assume that this distribution function is continuously differentiable so that the probability density function by $f$ and the hazard rate 
\[
h(x)=\frac{f(x)}{\bar F(x)}
\] 
are both continuous.

\subsection{Degradation model}
To model the degradation of the asset, we introduce random variables $\omega_k:=\omega_k(x_k)$ at decision epoch $k$ which only depend on $x_k$.
$$
\omega_k(x_k):=\begin{cases}
1,&\text{if the asset will reach age }\ \delta x_{k}\text{ given}\\
&\text{that it reached age}\ \delta(x_k-1)\ \text{ at time } t_k \\
0,&\text{otherwise}.
\end{cases}
$$
If the lifetimes of the asset are i.i.d. random variables with distribution $F$ then for $x>0$
$$
\mathbb{P}(\omega_k(x_k)=1)=\frac{1-F(\delta x_k)}{1-F(\delta (x_k-1))}.
$$
%and
%\[
%\mathbb{P}(\omega_k(x_k)=0)=\frac{F(\delta x_k)-F(\delta (x_k-1))}{1-F(\delta (x_k-1))}=\delta h(\delta x_k)+o(\delta^2)
%\]
%for small $\delta$.
%This hazard rate is assumed to be increasing.
%In the rest of the thesis we assume that the lifetime of the asset has an increasing hazard rate so that $\mathbb{P}(\omega_k(x)=1)$ is decreasing in $x$.

\subsection{Control actions}
At the $k$'th decision epoch, we shall choose an action $u_k$ from the action set $U(x_k)$.
Where
$$
U(x_k):=\begin{cases}
\{a_W,a_R\},&\text{if}\ x_k>0 \\
\{a_R\},&\text{if}\ x_k=0.
\end{cases}
$$
These actions are
\begin{itemize}
	\item $a_R$:
	Repair (or replace) the asset.
	\item $a_W$:
	Do nothing.
\end{itemize}

\subsection{State evolution}
During a time interval, a few things can happen:
\begin{itemize}
	\item If the asset is repaired, its age will be $\delta$ at the next stage.
	\item If the asset fails, its age will be $0$ at the next stage.
	\item If the asset does not fail and no repair is done, the age of the asset will increase by $\delta$.
\end{itemize}
Hence, the state evolves in the following way
$$
x_{k+1}=f(x_k,u_k,\omega_k):=\begin{cases}
1,&\text{if}\ u_k=a_R \\
0,&\text{if}\ u_k=a_W\ \text{and}\ \omega_k=0 \\
x_k+1,&\text{if}\ u_k=a_W\ \text{and}\ \omega_k=1.
\end{cases}
$$
For convenience, we define the random variable $S(x_k):=f(x_k,a_W,\omega_k(x_k))$ to denote the age of the asset one time interval after it was $x_k$.
Note that in the above definition, repairing the asset takes exactly one time interval.

\subsection{Costs and discounting}
Preventively repairing the asset has a cost $c>0$.
When the asset needs to be repaired correctively, an additional cost $a>0$ needs to be paid.
Hence, when the process is in state $x_k$ and the action $u_k$ is chosen, the following cost is incurred
$$
g(x_k,a_k):=\begin{cases}
c+a,&\text{if}\ x_k=0 \\
c,&\text{if}\ x_k>0\ \text{and}\ u_k=a_R \\
0,&\text{else}.
\end{cases}
$$
Furthermore, a discount $\alpha_\delta$ is introduced such that costs $n$ decision stages in the future are discounted by $\alpha_\delta^n$.
In the rest of the thesis, we will use a discount factor
$$
a_\delta=e^{-\beta\delta}
$$
for some discount rate $\beta>0$.
We consider the expected total discounted cost from decision epoch $k$ on
$$
V_\delta(x_k;k)=\sum\limits_{m=k}^\infty \alpha_\delta^{m-k}g(x_m,u_m).
$$


\subsection{Optimal stationary policy and the Bellman equations}
As the problem as defined above is a Markov decision process with $g$ independent of the decision epoch, there exists an optimal stationary policy \cite{Puterman2008}.
Hence, we want to find a stationary policy $\mu:X\rightarrow \{a_W,a_R\}$ that chooses the action $u_k=\mu(x_k)$ that minimizes the expected total discounted cost $V_\delta(x_0,\mu)$.
For a policy $\mu$, $V_\delta(x_k,\mu)$ is given by
$$
V_\delta(x_k,\mu)=g(x_k,\mu(x_k))+\alpha_\delta \mathbb{E}[V_\delta(f(x_k,\mu(x_k),\omega_k(x_k)),\mu)].
$$
The Bellman equations for the optimal cost $V^*_\delta$ read
\begin{equation}\label{eq:AgeBasedBellman}
V^*_\delta(x_k)=\begin{cases}
\min\{c+\alpha_\delta V^*_\delta(1),\alpha_\delta \mathbb{E}[V^*_\delta(S(x_k))]\},&\text{if}\ x_k>0 \\
c+a+\alpha_\delta V^*_\delta(1),&\text{else.}
\end{cases}
\end{equation}
Under the optimal policy, the total discounted cost $V_\delta(x_k,\mu;k)=V^*_\delta(x_k)$ for all $x_k,k$.
Since we only consider stationary policies, the $k$ will also be suppressed in the notation of $V_\delta(x_k,\mu)$.

\subsection{Continuous-time MDP}
In the remainder of the chapter, we will consider a continuous-time MDP that is obtained by letting $\delta\rightarrow0$.
We will occasionally return to the discrete MDP to prove certain properties.
For small $\delta$ the discount $\alpha_\delta$ is given by
\[
\alpha_\delta=1-\delta\beta+o(\delta^2).
\]
When action $a_W$ is chosen, at age $x_k$, the transition probabilities become
$$
\mathbb{P}(\omega_k(x_k)=1)=\frac{1-F(\delta x_k)}{1-F(\delta (x_k-1))}=1-\delta h(\delta x_k)+o(\delta^2),
$$
and
\[
\mathbb{P}(\omega_k(x_k)=0)=\frac{F(\delta x_k)-F(\delta (x_k-1))}{1-F(\delta (x_k-1))}=\delta h(\delta x_k)+o(\delta^2).
\]
This hazard rate $h$ is assumed to be increasing.
For the value functions, we will write $V^*(x):=\lim\limits_{\delta\rightarrow0}V_\delta^*(\lfloor x/\delta\rfloor)$ and similarly $V(x,\mu):=\lim\limits_{\delta\rightarrow0}V_\delta(\lfloor x/\delta\rfloor,\mu)$.
Moreover $V^*(0^+):=\lim\limits_{\delta\rightarrow0}V_\delta^*(1)$.
Note that we have to write $V^*(0^+)$ instead of $V^*(0)$ for a new asset as the latter corresponds to the asset being broken.

\subsection{Alternative models}
The problem can be formalized in many different ways.
We will briefly show some alternatives to the modeling choices that were made in the above problem definition.
\subsubsection{Instantaneous repairs}
If we would want to make repairs take zero time, we would have to change the definition of $x_{k+1}$ to 
$$
x_{k+1}=f_2(x_k,u_k,\omega_k):=\begin{cases}
0,&\text{if}\ \omega_k=0 \\
2,&\text{if}\ u_k=a_R\ \text{and}\ \omega_k=1\\
x_k+1,&\text{if}\ u_k=a_W\ \text{and}\ \omega_k=1.
\end{cases}
$$
This would however, introduce the possibility of having to correctively repair the asset twice in a row.
However, when we let $\delta\rightarrow 0$, this would not make any difference.

\subsubsection{Stochastic inter-decision times}
Another possibility would be to have positive random i.i.d. inter-decision times $\Delta_k$ and to use a continuous discount such that costs at time $t$ are discounted by $e^{-\beta t}$ for $\beta>0$.
The state space could then be modeled as $X=\mathbb{R}_0^+\cup\{x_{BREAK}\}$ where $x_{BREAK}$ is the state where the asset is broken.
The state evolution would be as follows:

$$
x_{k+1}=f(x_k,u_k,\omega_k,\Delta_k):=\begin{cases}
\Delta_k,&\text{if}\ u_k=a_R \\
x_f,&\text{if}\ u_k=a_W\ \text{and}\ \omega_k=0 \\
x_k+\Delta_k,&\text{if}\ u_k=a_W\ \text{and}\ \omega_k=1.
\end{cases}
$$
The Bellman equations should then be changed to
$$
V^*(x_k)=\begin{cases}
\min\{c+\mathbb{E}[e^{-\beta \Delta} V^*(\Delta)],\mathbb{E}[e^{-\beta \Delta} V^*(f(x_k,a_W,\omega_k,\Delta))],&\text{if}\ x_k\neq x_f \\
c+a+\mathbb{E}[e^{-\beta \Delta} V^*(\Delta)],&\text{else.}
\end{cases}
$$
Where repair would again take one inter-decision time. Where $\Delta$ is of the same family of i.i.d. random variables as the $\Delta_i$'s.

%\subsubsection{Remaining lifetime in state information}
%The remaining lifetime could also be included in the state information.
%We would then denote the cost when the machine still has a time $q$ to live by $V(q)$ and in each time step, this remaining lifetime would decrease by $\delta$ until the remaining lifetime is negative, when the machine must be repaired.
%If the initial remaining lifetime would be given by the random variable $Q\sim F$, the total discounted cost would then be $\mathbb{E}[V(Q)]$ and the Bellman equations would be:
%$$
%V(q)=\min\{
%c+\alpha_{\delta} \mathbb{E}[V(Q)],
%\alpha_{\delta} V(q-\delta)
%\}
%$$
%for $q>0$ and
%$$
%V(q)=c + a + \alpha_{\delta} \mathbb{E}[V(Q)],
%$$
%else.
%These equations are easy to solve although the resulting policy is not very useful in a realistic setting:
%If $0<q\leq\delta$, then 
%\[
%\begin{split}
%V(q)&=\min\{
%c+\alpha_{\delta} \mathbb{E}[V(Q)],\alpha_{\delta} V(q-\delta)\}\\
%&=\min\{
%c+\alpha_{\delta} \mathbb{E}[V(Q)],\alpha_{\delta}(c+a+\alpha_{\delta} \mathbb{E}[V(Q)])\}\\
%&=c+\alpha_{\delta} \mathbb{E}[V(Q)]
%\end{split}
%\]
%(Assuming $a>\frac{1-\alpha_\delta}{\alpha_\delta}(c+\alpha_\delta\mathbb{E}[V(Q)])$, note that for sufficiently small discount factor $\alpha_\delta$, this does not hold.).
%And by induction, we can prove that for $(k-1)\delta<q\leq k\delta$ ($k>0$)
%\[\begin{split}
%V(q)&=\min\{
%c+\alpha_{\delta} \mathbb{E}[V(Q)],\alpha_{\delta} V(q-\delta)\}\\
%&=\min\{
%c+\alpha_{\delta} \mathbb{E}[V(Q)],\alpha_{\delta} \alpha_{\delta}^{k-2}(c+\alpha_\delta\mathbb{E}[V(Q)])\}\\
%&=\alpha_{\delta}^{k-1}(c+\alpha_\delta\mathbb{E}[V(Q)]).
%\end{split}\]
%
%As can be seen, the cost does not depend on $a$ as correctively repairing never occurs. This is because in this formulation, the time-to-live is known when making the decision (when choosing the minimum). Such that we will repair at the last opportunity before the machine breaks down.
%If, for instance, we choose $\alpha_\delta=e^{-\beta \delta}$ for $\beta>0$, we have for $(k-1)\delta<q\leq k\delta$
%$$
%V(q)=e^{-\beta(k-1)\delta}(c+\alpha_\delta\mathbb{E}[V(Q)]).
%$$
%If we let $\delta$ approach zero, the cost approaches 
%$$
%V(q)=e^{-\beta q}(c+\alpha_\delta\mathbb{E}[V(Q)]).
%$$
%As can be seen, this is the cost of repairing the machine preventively at exactly the instant that it will break, which is indeed optimal but can unfortunately not be realized as the time-to-live cannot be directly observed.
% Structure of optimal policy
\section{Structure of optimal policy}
In this section, we will establish that for the age-based maintenance problem, the optimal policy is a stationary control limit policy.
This means that repair is chosen if and only if the age has exceeded a certain threshold $\mu^*$ (the control limit).
If no repair is chosen, then we set $\mu^*=\infty$.

\subsection{Stationary policy}\label{section:AgeBasedStationaryPolicy}
%Referring back to the Bellman equations of the age-based problem \eqref{eq:AgeBasedBellman}, you can see that when $x_k>0$, repair is chosen whenever
%\[ c+\alpha_\delta V^*_\delta(1) <\alpha_\delta \mathbb{E}[V^*_\delta(S(x_k))]. \]
%Where the left hand side is a constant and the right hand side only depends on $S(x_k)$, which only depends on the age of the asset by assumption.
%Hence, we have established that the optimal policy only depends on the age $x_k$ so that it must be a stationary policy.
In \cite{Puterman2008}, it is proven that for a discounted infinite-horizon countable-state MDP where the cost function does not depend on the decision stage, there exists a stationary policy that is optimal.
This proves that there exists an optimal stationary policy to this age-based preventive maintenance policy.

\subsection{Control limit}\label{section:AgeBasedControlLimit}
By letting $\delta\rightarrow 0$, we can see from the Bellman equations \eqref{eq:AgeBasedBellman} that repair is chosen whenever
\begin{equation}\label{eq:AgeBasedRepairCondition}
 c+\alpha V^*(0^+) \leq V^*(x). 
 \end{equation}
Now we can distinguish two cases:
\begin{enumerate}
	\item There is no age $x$ that satisfies \eqref{eq:AgeBasedRepairCondition} and preventive repair is never the optimal choice.
	In this case we set control limit $\mu^*=\infty$.
	\item There are ages $x_1,x_2,...$ that satisfy \eqref{eq:AgeBasedRepairCondition}.
	The control limit will now simply be the smallest such age.
	What happens for ages greater than this $\mu^*$ is not relevant as these will never be reached.
\end{enumerate}
Later in this chapter, we will prove that the right hand side of \eqref{eq:AgeBasedRepairCondition} is increasing for $x<\mu^*$.
Hence, we have established that the optimal policy must be of control limit type where the asset is repaired whenever its age exceeds some threshold $\mu^*$.
% Computation of total discounted cost
\section{Computation of total discounted cost}
In this section, the Bellman equations will be used to find the expected total discounted cost of the optimal cost.
The following Bellman equations will be considered
\[V_\delta(x)=\begin{cases}
\min\{c+\alpha_\delta V_\delta(1),\alpha_\delta \mathbb{E}[V_\delta(S(x))]\},&\text{if}\ x>0 \\
c+a+\alpha_\delta V_\delta(1),&\text{else.}
\end{cases}\]
Where $\mathbb{P}(S(x)=0)=\mathbb{P}(Q_0\leq x+\delta|Q_0\geq x)=\delta h(x)+o(\delta^2)$ (for lifetime $Q_0\sim F(x)$ and corresponding $\bar F(x)=1-F(x)$, $\bar F(x;y)=\bar F(x)/\bar F(y)$, probability density $f$, hazard rate $h(x)$) and $\mathbb{P}(S(x)=x+1)=1-\delta h(x)+o(\delta^2)$ and $\alpha_\delta=e^{-\beta \delta}=1-\beta \delta + o(\delta^2)$ for $\beta>0$.
We define $V_\delta(n\delta):=V_\delta(n)$ and for convenience, we define $V_\delta(0^+):=V_\delta(\delta)$.
We assume that for $x$
$$
c+\alpha_\delta V_\delta(0^+)>\alpha_\delta \mathbb{E}[V_\delta(S(x)\delta)],
$$
i.e. the optimal control limit $\mu^*>\delta x$.
Now, we can write $V_\delta(x)$ in the following way
\[V_\delta(x)=\alpha_\delta \bar F(x+\delta;x) (c+a+\alpha_\delta V_\delta(0^+))
+\alpha_\delta \bar F(x+\delta;x)V_\delta(x+\delta)
\]
We are now going to let $\delta$ approach zero.
\begin{equation}
\begin{split}
\lim\limits_{\delta\rightarrow 0} V_\delta(x) &=
\lim\limits_{\delta\rightarrow 0}(1-\beta\delta+o(\delta^2)) (\delta h(x)+o(\delta^2))(c+a+(1-\beta\delta+o(\delta^2)) V'(0^+))\\
&+(1-\beta\delta+o(\delta^2)) (1-\delta h(x)+o(\delta^2))V_\delta(x+\delta).
\end{split}
\end{equation}
Gathering the terms of $o(\delta^2)$, we get
\begin{equation}\label{eq:gatheredDelta}
\begin{split}
\lim\limits_{\delta\rightarrow 0} V_\delta(x) =
\lim\limits_{\delta\rightarrow 0}\delta h(x)(c+a+ V_\delta(0^+))+(1-\delta\beta-\delta h(x)) V_\delta(x+\delta)+o(\delta^2).
\end{split}
\end{equation}
By moving one $V_\delta(x+\delta)$ to the left and dividing by $-\delta$, we get
\begin{equation}\label{eq:AgeBasedBellmanODE}
\begin{split}
\frac{d}{dx}V(x)&=\lim\limits_{\delta\rightarrow 0} \frac{V_\delta(x+\delta)-V_\delta(x)}{\delta} \\
&=\lim\limits_{\delta\rightarrow 0} -h(x)(c+a+ V_\delta(0^+))+(\beta+ h(x)) V_\delta(x+\delta)+o(\delta)\\
&=-h(x)(c+a+ V(0^+))+(\beta+ h(x)) V(x).
\end{split}
\end{equation}
Where
$$
V(x):=\lim\limits_{\delta\rightarrow 0} V_\delta(x).
$$
(Note that we needed to define $V(0^+)$ in this way since $V(0^+)=V(0)-c-a$).
\begin{remark}\label{remark:AgeBasedWeirdODE}
	The differential equation \eqref{eq:AgeBasedBellmanODE} seems counterintuitive as for high hazards, the expected total discounted cost would be decreasing for increasing age of the machine.
	We will return to this later on.
\end{remark}
We will now try to solve this O.D.E.
We use the method of the integrating factor.
Our integrating factor will be
$$
e^{\int\limits_0^x (-\beta - h(q))dq}=e^{-\beta x -H(x)}.
$$
Where $H(x)$ is the cumulative hazard function.
We get
$$
V(x)=e^{\beta x +H(x)} [C + \int\limits_0^x e^{-\beta q -H(q)}(-h(q)(c+a+V(0^+)))dq]
$$
$$
=\frac{e^{\beta x}}{\bar F(x)} [C - (c+a+V(0^+))\int\limits_0^x e^{-\beta q}f(q)dq].
$$
Using the identities 
\[e^{H(x)}=(e^{-H(x)})^{-1}=\frac{1}{\bar F(x)},\]
and
\[h(x)e^{-H(x)}=f(x).\]
The $C$ is an integrating constant and since $\lim\limits_{x\rightarrow 0}V(x)=V(0^+)$ should hold, we find $C=V(0^+)$.
We can rewrite the expression to
$$
V(x)=\frac{e^{\beta x}}{\bar F(x)} [V(0^+)  - (c+a+V(0^+))F(x)\mathbb{E}[e^{-\beta Q_0}|Q_0<x]].
$$
Which results in the following theorem:
\begin{theorem}
	When the machine has age $x$, the expected remaining total discounted cost equals
	\begin{equation}\label{eq:AgeBasedSolvedBellman}
	\begin{split}
	V(x)=\min\{&c+V(0^+),\\
	&\frac{ e^{\beta x}}{\bar F(x)} [V(0^+) - (c+a+V(0^+))F(x)\mathbb{E}[e^{-\beta Q_0}|Q_0<x]]
	\}
	\end{split}
	\end{equation}
	and preventive maintenance is chosen if and only if $V(x)\geq c+V(0^+)$.
\end{theorem}
Unfortunately, the value of $V(0^+)$ depends on the policy that is chosen and it seems difficult to solve $V(x)=c+V(0^+)$ analytically for $x$.
Control limit $\mu^*$ is then chosen as the smallest positive $x$ that satisfies $V(x)=c+V(0^+)$ if such $x$ exist and $\mu^*=\infty$ else.
The policy that we just derived, schedules preventive maintenance at time $\mu^*$ if the machine has not already failed by then.
We denote the total discounted cost of this policy by $V(0^+,\mu^*)$.

For any (possibly sub-optimal) control limit $\mu$, we can derive the expected remaining total discounted cost.
The length of one run of the machine is the minimum of its lifetime $Q_0$ and the chosen control limit $\mu$.
At the end of each run, at least $c$ is paid.
If the run ends because the machine broke (i.e. $Q_0<\mu$), an additional cost of $a$ is paid.
Hence, we get the following expression for $V(0^+,\mu)$
\begin{equation}\label{eq:AgeBasedPolicyTDC}
V(0^+,\mu)=aF(\mu)\mathbb{E}[e^{-\beta Q_0}|Q_0\leq \mu]+(c+V(0^+,\mu))\mathbb{E}[e^{-\beta(Q_0\wedge\mu)}].
\end{equation}
We get a similar expression for $V(x,\mu)$:

\begin{theorem}
	Given control limit $\mu$ and a machine of age $x$, the expected remaining total cost equals 
	\begin{equation}\label{eq:AgeBasedPolicyRemainingTDC}
	V(x,\mu)=aF(\mu;x)\mathbb{E}[e^{-\beta Q_0}|x<Q_0\leq \mu]+(c+V(0^+,\mu))\mathbb{E}[e^{-\beta(Q_0\wedge\mu)}|Q_0>x].
	\end{equation}
\end{theorem}

\begin{remark}
	By taking the derivative of \eqref{eq:AgeBasedPolicyRemainingTDC}, one can see that this also adheres to the differential equation \eqref{eq:AgeBasedBellmanODE} that resulted from the Bellman equations.
\end{remark}

\begin{remark}
	\eqref{eq:AgeBasedPolicyTDC} can also be rewritten to get an explicit expression for $V(0^+,\mu)$:
	\begin{equation}\label{eq:AgeBasedOptimalTDC}
	V(0^+,\mu)=\frac{aF(\mu)\mathbb{E}[e^{-\beta Q_0}|Q_0\leq \mu]+c\mathbb{E}[e^{-\beta(Q_0\wedge\mu)}]}{1-\mathbb{E}[e^{-\beta (Q_0\wedge\mu)}]}.
	\end{equation}
\end{remark}

\begin{remark}
	If $\mu^*=\infty$, then the expected total discounted cost equals
	\[V(0^+,\infty)=\mathbb{E}[e^{-\beta Q_0}](c+a+V(0^+,\infty)).\]
	Which can be rewritten to
	\[V(0^+,\infty)=\frac{\tilde{F}(-\beta)}{1-\tilde{F}(-\beta)}(c+a),\]
	where $\tilde{F}$ is the moment generating function of $Q_0$.
\end{remark}

\begin{example}
	Let $Q\sim\text{Exp}(\lambda)$.
	Because of the memoryless property, we would expect $V(x,\mu^*)$ to be constant.
	Filling this in into \eqref{eq:AgeBasedBellmanODE}, we get
	$$
	\frac{d}{dx}V(x,\mu^*)=0=-\lambda(c+a+ V(0^+,\mu^*))+(\beta+ \lambda) V(x,\mu^*)=-\lambda(c+a+ V(0^+,\mu^*))+(\beta+ \lambda) V(0^+,\mu^*).
	$$
	Which results in
	$$
	V(0^+,\mu^*)=\frac{\lambda}{\beta}(c+a)
	$$
	which equals exactly the total discounted cost for control limit $\mu^*\infty$.
\end{example}
% Analysis of the optimal policy
\section{Analysis of the optimal policy}\label{section:AgeBasedOptimalPolicy}
Instead of finding the optimal control limit $\mu^*$ by solving the Bellman equations, we can also minimize $V(0^+,\mu)$ by looking for extreme values of the expected total discounted cost.
Although we could use \eqref{eq:AgeBasedPolicyTDC} for the total discounted cost, we will use a slightly different formula to simplify the analysis.
Instead of using $V(0^+,\mu)$ on the right hand side of \eqref{eq:AgeBasedPolicyTDC}, we will use $V(0^+,\mu^*)$.
This corresponds to minimizing the expected total discounted cost for a machine using control limit $\mu$ in the first run and optimal control limit $\mu^*$ afterwards.
Obviously, this would be minimized by $\mu=\mu^*$.
We will look at extreme values of
\begin{equation}\label{eq:AgeBasedHatTDC}
\hat{V}(0^+,\mu)=aF(\mu)\mathbb{E}[e^{-\beta Q_0}|Q_0\leq \mu]+(c+V(0^+,\mu^*))\mathbb{E}[e^{-\beta(Q_0\wedge\mu)}].
\end{equation}
Note that 
\[
\begin{split}
&F(\mu)\mathbb{E}[e^{-\beta Q_0}|Q_0\leq \mu]=\int\limits_0^\mu f(x)e^{-\beta x}dx\\
&\Rightarrow \frac{d}{d\mu}\left[F(\mu)\mathbb{E}[e^{-\beta Q_0}|Q_0\leq \mu]\right]=f(\mu)e^{-\beta \mu}.
\end{split}
\]
Furthermore,
\[
\begin{split}
&\mathbb{E}[e^{-\beta(Q_0\wedge\mu)}]=\int\limits_0^\mu f(x)e^{-\beta x}dx+\bar{F}(\mu)e^{-\beta\mu}\\
&\Rightarrow \frac{d}{d\mu}\mathbb{E}[e^{-\beta(Q_0\wedge\mu)}]=-\beta\bar{F}(\mu)e^{-\beta\mu}.
\end{split}
\]
So that taking the derivative of \eqref{eq:AgeBasedHatTDC} yields
\[\frac{d}{d\mu}\hat{V}(0^+,\mu)=af(\mu)e^{-\beta\mu}-(c+V(0^+,\mu^*))\beta\bar{F}(\mu)e^{-\beta\mu}.\]
We are interested in the zeroes of this derivative:
\[
\begin{split}
&af(\mu)e^{-\beta\mu}-(c+V(0^+,\mu^*))\beta\bar{F}(\mu)e^{-\beta\mu}=0\\
&\Rightarrow h(\mu)=\frac{f(\mu)}{\bar{F}(\mu)}=\beta\frac{c+V(0^+,\mu^*)}{a}
\end{split}
.\]
Note that the right hand side of this equation is a constant and the left hand side is increasing by assumption.
Hence, there is at most one $\mu$ that satisfies the above equation.
From the above bounds it can also be seen that $\hat{V}(0^+,\mu)$ is decreasing when $h(\mu)$ is smaller than this constant and increasing when it is larger.
We will now establish that if there is one $\mu$ that satisfies the equation above, it is also the global minimum of $\hat{V}(0^+,\mu)$:
\begin{lemma}
	If there is a $\hat{\mu}$ that satisfies
	\begin{equation}\label{eq:AgeBasedHazardBound}
	h(\hat{\mu})=\beta\frac{c+V(0^+,\mu^*)}{a},
	\end{equation}
	then this $\hat{\mu}$ is the optimal control limit.
	\begin{proof}
		From the previous derivation, it follows that this $\hat{\mu}$ is a stationary point of $\hat{V}(0^+,\mu)$.
		Since $h$ is increasing by assumption, we know that
		\begin{itemize}
			\item $\mu<\hat{\mu}\Rightarrow h(\mu)<h(\hat{\mu})$ so that $\frac{d}{d\mu}\hat{V}(0^+,\mu)<0$.
			\item $\mu>\hat{\mu}\Rightarrow h(\mu)>h(\hat{\mu})$ so that $\frac{d}{d\mu}\hat{V}(0^+,\mu)>0$.
		\end{itemize}
		Which establishes that $\hat{\mu}$ is the global minimum of $\hat{V}(0^+,\mu)$.
		Concluding the control limit $\hat{\mu}$ that satisfies \eqref{eq:AgeBasedHazardBound} equals the optimal control limit $\mu^*$.
	\end{proof}
\end{lemma}
\begin{corollary}
	From \eqref{eq:AgeBasedHazardBound} and the fact that $\hat{V}(0^+,\mu)$ is decreasing at $\mu=0$, it follows that
	\[
	h(0)<\beta\frac{c+V(0^+,\mu^*)}{a}.
	\]
\end{corollary}
\begin{corollary}
If for all $\mu>0$
\[
h(\mu)<\beta\frac{c+V(0^+,\mu^*)}{a},
\]
then $\hat{V}(0^+,\mu)$ is strictly decreasing and has an asymptotic minimum, concluding $\mu^*=\infty$.
Note that this also implies that for decreasing hazard rates $\mu^*=\infty$.
\end{corollary}

\begin{remark}
	Using the Bellman equations, it can also be proven without the assumption of an increasing hazard rate that if there is an optimal control limit $\mu^*$, $\mu^*$ must satisfy \eqref{eq:AgeBasedHazardBound} and the hazard rate must be increasing at $\mu^*$.
	For a proof of this, we refer to appendix \ref{AppendixAgeBasedControlLimit}.
\end{remark}

\begin{remark}\label{remark:AgeBasedValueNonDecreasing}
	Returning to remark \ref{remark:AgeBasedWeirdODE}:
	The differential equation \eqref{eq:AgeBasedBellmanODE} seemed counterintuitive as the total discounted cost would be decreasing for high hazards.
	
	Consider the cost of one run of the machine, we replace the repair cost by $c^*=c+V(0^+,\mu^*)$ so that the cost of the expected discounted cost of the first repair equals the expected total discounted cost of the original problem.
	$V(x,\mu^*)$ now corresponds to the cost of this altered problem, but starting with a machine of age $x$.
	If $V(x,\mu^*)$ were to be decreasing in the neighborhood of some $x$, this would mean that for that $x$, the problem would have a lower expected optimal cost if we started with a slightly older machine.
	This seems to conflict with the assumption that $h$ is increasing so that the machine deteriorates over time.
	
	It is difficult to prove that this is not the case in general but results from value iteration seem to suggest $V(x,\mu^*)$ is nondecreasing in $x$.
	We can, however, prove that if $\frac{d}{dx}V(x,\mu^*)<0$ for some $x$, then no control limit exists:
	
	Let $x'<\mu^*$, it holds that
	\[
	V(x',\mu^*)\leq V(\mu^*,\mu^*)=c+V(0^+,\mu^*).
		\]
	Now we will prove that if $\frac{d}{dx}V(x',\mu^*)<0$, this implies that $\frac{d^2}{dx^2}V(x',\mu^*)<0$.
	We take the derivative of \eqref{eq:AgeBasedBellmanODE}:
	\[
	\frac{d^2}{dx^2}V(x,\mu^*)=-h'(x)[c+a+V(0^+,\mu^*)-V(x,\mu^*)]+(\beta+h(x))\frac{d}{dx}V(x,\mu^*).
	\]
	We know that for our $x'$, $\frac{d}{dx}V(x,\mu^*)<0$ and $V(x,\mu^*)\leq c+V(0^+,\mu^*)$.
	Furthermore, by assumption $h'(x')>0$.
	Concluding
	\[
	\frac{d^2}{dx^2}V(x',\mu^*)<0.
	\]
	This implies that if $\frac{d}{dx}V(x',\mu^*)<0$ for some $x'$, $V(x,\mu^*)$ is strictly decreasing after $x'$ so that for all $x>x'$
	\[
	V(x,\mu^*)< c+V(0^+,\mu^*),
	\]
	which implies that $\mu^*=\infty$.
	
	Hence, by contraposition, if an optimal control limit $\mu^*<\infty$ exists, $\frac{d}{dx}V(x,\mu^*)\geq0$ for all $x\in[0,\mu^*]$.
	This allows us to make the following lower bound on $V(x,\mu^*)$:
\end{remark}


% Computing the optimal policy
\section{Policy iteration}\label{section:AgeBasedOptimalPolicyComputation}
We know that for the optimal policy $\mu^*$, \eqref{eq:AgeBasedHazardBound} holds.
This allows us to choose a control limit based on the total discounted cost.
Unfortunately, the total discounted cost also depends on the control limit.
Multiple numerical methods could be used to find the optimal control limit.
In this section we propose a modified policy iteration method to find the optimal control limit and the total discounted cost.
Alternatively, value iteration could be used to solve the Bellman equations or the expected total discounted cost \eqref{eq:AgeBasedPolicyTDC} could simply be minimized numerically for $\mu$.

\subsection{Description of iteration method}\label{section:AgeBasedIterationDescription}
We know that if a control limit $\hat{\mu}$ satisfies
\[h(\hat{\mu})=\beta\frac{c+V(0^+,\mu^*)}{a},\]
then $\hat{\mu}=\mu^*$.
At the $k+1$'th iteration, we will update the estimate of the optimal control limit $\mu^*$ by finding the $\hat{\mu}^{(k+1)}$ that minimizes
\[
aF(\hat{\mu}^{(k+1)})\mathbb{E}[e^{-\beta Q_0}|Q_0\leq \hat{\mu}^{(k+1)}]+(c+\hat{V}^{(k)})\mathbb{E}[e^{-\beta(Q_0\wedge\hat{\mu}^{(k+1)})}],
\]
where $\hat{V}^{(k)}$ is the current estimate of $V(0^+,\mu^*)$.
This could be found by looking for the control limit that satisfies
\begin{equation}\label{eq:AgeBasedIterationBound}
h(\hat{\mu}^{(k+1)})=\beta\frac{c+\hat{V}^{(k)}}{a}.\end{equation}
For convenience, we define the function $\mu(\hat{V}^{(k)}):=\hat{\mu}^{(k+1)}$.
Note that by lemma \ref{lemma:AgeBasedControlLimit}
\begin{equation}\label{eq:AgeBasedIterationControlConvergence}
\mu(V(0^+,\mu^*))=\mu^*.
\end{equation}
The estimation of the expected total discounted cost will be updated in the following way:
\[\hat{V}^{(k+1)}=aF(\hat{\mu}^{(k+1)})\mathbb{E}[e^{-\beta Q_0}|Q_0\leq \hat{\mu}^{(k+1)}]+(c+\hat{V}^{(k)})\mathbb{E}[e^{-\beta(Q_0\wedge\hat{\mu}^{(k+1)})}].\]
We define an operator $T$ to denote one iteration for the estimate of the expected total discounted cost:
\[T(\hat{V}^{(k)}):=\hat{V}^{(k+1)}.\]
Similarly:
\[T^m(\hat{V}^{(k)})=\hat{V}^{(k+m)}.\]
Note that $T^m(V(0^+,\mu^*))=V(0^+,\mu^*)$.
For this iteration, we need an initial value of the expected total discounted cost $\hat{V}^{(0)}$.
This iteration can be interpreted in the following way:
If the asset were to run just once and at the end of the run (so either after paying $c$ or $c+a$ for preventive or corrective repair respectively), a cost $\hat{V}^{(k)}$ will be paid, then $\hat{\mu}^{(k+1)}$ will be the control limit that minimizes the expected total discounted cost for this scenario.
In this way, we can interpret $\hat{V}^{(k)}$ as the expected total discounted cost when in the first run $\hat{\mu}^{(k)}$ will be used as the control limit, $\hat{\mu}^{(k-1)}$ will be used as control limit in the second run, etcetera, $\hat{\mu}^{(1)}$ in the last run and afterwards the terminal cost $\hat{V}^{(0)}$ will be paid.
%We can view these chosen control limits as actions in a deterministic decision process.

\subsection{Proof of convergence}
The convergence of the proposed iteration method will now be proven.
Let $\alpha_\mu=\mathbb{E}[e^{-\beta (Q_0\wedge\mu)}]$ denote the expected discount over one run of the asset using control limit $\mu$.
The expected cost that is incurred in one  run when control limit $\mu$ is used equals
$$
g(\mu):=aF(\mu)\mathbb{E}[e^{-\beta Q_0}|Q_0\leq \mu]+c\mathbb{E}[e^{-\beta(Q_0\wedge\mu)}].
$$
We can now write
$$
V(0^+,\mu^*)=\sum\limits_{k=0}^\infty \alpha_{\mu^*}^kg(\mu^*).
$$
And we can rewrite $T$ in the following way
\[
T(V)=g(\mu(V))+\alpha_{\mu(V)}V.
\]
Note that $\alpha_\mu$ is decreasing in $\mu$ since $\frac{d}{d\mu}\alpha_\mu=-\beta\bar F(\mu)e^{-\beta \mu}<0$.
For technical reasons, we have to assume that for all $\hat{V}^{k}$, $\mu(\hat{V}^{k})\geq\varepsilon>0$.
In practice, this is not a big problem since we know that $\mu^*>0$ as $\lim\limits_{\mu\rightarrow 0}V(0^+,\mu)=\infty$.
Hence we can pick a $\varepsilon$ sufficiently small so that we are convinced that $\varepsilon<\mu^*$ and just adjust the definition of $\mu(\hat{V}^{(k)})$ to the maximum of $\varepsilon$ and $\mu(\hat{V}^{(k)})$.
For $T$, we will prove the following properties:
\begin{lemma}\label{lemma:Tproperties}
	For $A_1,A_2$ such that $A_1\geq A_2\geq 0$: 
	\begin{enumerate}
		\item $T(A_1+A_2)\leq TA_1+\alpha_\varepsilon A_2$,
		\item $T(A_1)\geq T(A_2)$,
		\item $T(A_1-A_2)\geq TA_1-\alpha_\varepsilon A_2$.
	\end{enumerate}
	\begin{proof}
		\begin{enumerate}
			\item 
			\begin{equation}
			\begin{split}
			T(A_1+A_2)&=g(\mu(A_1+A_2))+\alpha_{\mu(A_1+A_2)}(A_1+A_2)\\
			&\leq g(\mu(A_1))+\alpha_{\mu(A_1)}(A_1+A_2)\\
			&\leq g(\mu(A_1))+\alpha_{\mu(A_1)}A_1+\alpha_\varepsilon A_2\\
			&=TA_1+\alpha_\varepsilon A_2
			\end{split}
			\end{equation} 
			where the first inequality follows from the fact that $\mu(A_1+A_2)$ minimizes $g(\mu)+\alpha_\mu (A_1+A_2)$ and the second from the fact that $a_\varepsilon>a_{\mu(A_1+A_2)}$.
			\item 
			\begin{equation}
			\begin{split}
			T(A_2)&=g(\mu(A_2))+\alpha_{\mu(A_2)}A_2\\
			&\leq g(\mu(A_1))+\alpha_{\mu(A_1)}A_2\\
			&\leq g(\mu(A_1))+\alpha_{\mu(A_1)}A_1\\
			&=T(A_1)
			\end{split}
			\end{equation}
			where the first inequality follows from the fact that $\mu(A_2)$ minimizes $g(\mu)+\alpha_\mu A_2$ and the second from $A_1\geq A_2$.
			\item 
			\begin{equation}
			\begin{split}
			T(A_1-A_2)&=g(\mu(A_1-A_2))+\alpha_{\mu(A_1-A_2)}(A_1-A_2)\\
			&\geq g(\mu(A_1-A_2))+\alpha_{\mu(A_1-A_2)}A_1 - \alpha_\varepsilon A_2\\
			&\geq g(\mu(A_1))+\alpha_{\mu(A_1)}A_1-\alpha_\varepsilon A_2\\
			&=TA_1-\alpha_\varepsilon A_2
			\end{split}
			\end{equation}
			where the first inequality follows from $a_\varepsilon>a_{\mu(A_1-A_2)}$ and the second from the fact that $\mu(A_1)$ minimizes $g(\mu)+\alpha_\mu A_1$.
		\end{enumerate}
	\end{proof}
\end{lemma}
Note that $g(\mu)<c+a$ for all $\mu$ so that 
$$
V(0^+,\mu^*)=\sum\limits_{n=0}^\infty \alpha_{\mu^*}^kg(\mu^*)\leq \sum\limits_{n=0}^\infty \alpha_\varepsilon^k(c+a)=\frac{c+a}{1-\alpha_\varepsilon}.
$$
If our initial $0\leq \hat{V}^{(0)}<B$, then the following inequality now holds
$$
V(0^+,\mu^*)-\frac{c+a}{1-\alpha_\varepsilon}\leq 0\leq \hat{V}^{(0)}\leq B\leq V(0^+,\mu^*)+B.
$$
If we now apply $T$ $k$ times on this inequality, we get
\[\begin{split}
V(0^+,\mu^*)-\alpha_\varepsilon^k\frac{c+a}{1-\alpha_\varepsilon} &\leq T^k(V(0^+,\mu^*)-\frac{c+a}{1-\alpha_\varepsilon})\\
&\leq  T^k\hat{V}^{(0)}=V_k\\
&\leq T^k(V(0^+,\mu^*)+B) \\
&\leq V(0^+,\mu^*)+\alpha_\varepsilon^kB.
\end{split}\]
Where the first and last inequalities follow from Lemma \ref{lemma:Tproperties}.
This proves that $\hat{V}^{(k)}$ converges.
Convergence in $\hat\mu^{(k)}$ follows from \eqref{eq:AgeBasedIterationControlConvergence}.
Concluding:
\begin{theorem}
	The modified policy iteration method as described in section \ref*{section:AgeBasedIterationDescription} converges, i.e.
	\[\lim\limits_{k\rightarrow\infty}\hat{V}^{(k)}=V(0^+,\mu^*),\]
	and
	\[\lim\limits_{k\rightarrow\infty}\hat{\mu}^{(k)}=\mu^*.\]
\end{theorem}
% Structural properties
\section{Structural properties}\label{section:AgeBasedStructuralProperties}
In this section, the effect of changing problem parameters is investigated.

\subsection{Effect on control limit}
From \eqref{eq:AgeBasedHazardBound}, we can see what effects changing $h,\beta,c$ and $a$ has on the control limit:
\[h(\hat{\mu})=\beta\frac{c+V(0^+,\mu^*)}{a}.\]
\begin{remark}
	Decreasing  $a$ would result in an increase of the right hand side (the hazard bound) of this equation, increasing the control limit.
	This is expected, as this would decrease the incentive to prevent the asset from failing.
\end{remark}
\begin{remark}
	Decreasing $c$ would result in a decrease of the hazard bound, decreasing the control limit.
	This is also natural as this would make $a$ relatively larger.
\end{remark}
\begin{remark}
	If we would multiply $c$ and $a$ by a constant, then the control limit would not change as the sizes of these costs relative to each other remains the same.
	This can be proven using remark \ref{remark:AgeBasedCostRatio}.
\end{remark}
\begin{remark}\label{remark:AgeBasedControlLimitDiscountIncrease}
	Increasing $\beta$ will result in an increase in hazard bound, resulting in an increase in the control limit.
	This can be explained by the fact that with a higher discount, costs further in the future weigh less so that you would want to move costs as far into the future as possible.
\end{remark}
\begin{remark}
	A higher hazard (or a faster increasing hazard) results in a lower control limit as expected.
\end{remark}
\begin{remark}\label{remark:AgeBasedControlLimitDiscountAndHazardIncrease}
	If the hazard rate and the discount exponent are both multiplied by a constant, then the hazard rate remains the same as this constant can just be divided out of this equation.
\end{remark}

\subsection{Effect on expected total discounted cost}
From \eqref{eq:AgeBasedOptimalTDC}, we can see what effects changing $h,\beta,c$ and $a$ has on the expected total discounted cost:
\[V(0^+,\mu)=\frac{aF(\mu)\mathbb{E}[e^{-\beta Q_0}|Q_0\leq \mu]+c\mathbb{E}[e^{-\beta(Q_0\wedge\mu)}]}{1-\mathbb{E}[e^{-\beta (Q_0\wedge\mu)}]}.\]
\begin{remark}
	Decreasing $c$ or $a$ results in a decrease of this total cost as expected.
\end{remark}
\begin{remark}\label{remark:AgeBasedCostRatio}
	If we multiply both $c$ and $a$ by a constant, the cost will also be multiplied by this constant.
	This is natural as changing the currency the costs are expressed in should not be relevant.
\end{remark}
\begin{remark}
	A higher hazard would result in a higher expected value for the discount, decreasing the denominator and increasing the total discounted cost.
	This is natural as an asset that breaks quicker is more expensive to maintain.
\end{remark}
\begin{remark}\label{remark:AgeBasedTDCDiscountIncrease}
	Increasing the discount increases the denominator and the numerator so that the discounted cost decreases, as expected.
\end{remark}
For various parameters and distributions, the optimal control limit and total discounted cost are summarized in appendix \ref{AppendixComputationsTable}.

\chapter{Simple fluid model with jumps}\label{chapter:SimpleFluid}
The problem of the previous chapter can be seen as a very simple Markov modulated fluid model:
Initially, the bucket has an amount of fluid $Q\sim F$.
The fluid decreases constantantly with rate $1$ and no fluid jumps occur.
In this chapter, we extend this model by allowing jumps to occur according to a Poisson process with rate $\lambda$.
The jumps all have the same constant (and known size) $J$.
In the real world, these fluid jumps could correspond to a partial repair of the machine.
The presence of jumps introduces the following complications:
\begin{itemize}
	\item The hazard of the machine failing at some time $t$ cannot be directly from the age only but also depends on the number of jumps that occurred before $t$.
	\item Furthermore, the times at which these jumps occurred also matter.
	When a jump occurs at time $t$, it is certain that the fluid quantity is at least $J$ so that you know for certain that in the interval $[t,t+J)$ the machine cannot fail.
\end{itemize}
In this chapter, the expected total discounted cost is calculated and methods are introduced to find the optimal replacement policy.
% Problem formulation and definition
\section{Problem formulation and definition}
In this section, we extend the definition of age-based maintenance from section \ref{section:AgeBasedDefinition}.
First we define the underlying stochastic process of the state of the machine deteriorating over time and instantaneously increasing at the occurrence of jumps.
After that, we define a Markov decision process similarly to section \ref{section:AgeBasedDefinition}.

\subsection{Stochastic machine breakdown}
We define the random process $Q(t)$ as the fluid level at age $t$.
Initially, the fluid level is given by $Q(0)=Q_0\sim F$.
Then over time this level decreases at a constant rate of $1$.
The jumps occur according to a Poisson process with rate $\lambda$, i.e. the time interval between two consecutive jumps is exponentially distributed with rate $\lambda$.
Hence, let $P_\lambda(t_1,t_2)$ be a Poisson distributed random variable with rate $\lambda (t_2-t_1)$.
The fluid process is absorbing at $Q(t)=0$.
Then the fluid level at time $t$ is given by 
\[
Q(t)\stackrel d=(Q_0+P_\lambda(0,t)J-t)\vee 0.
\]
Where $\stackrel d=$ denotes that the random variables on the left and right have the same distribution and $A\vee B$ denotes the maximum of $A$ and $B$.
$Q(t)$ has the property that
\[
\begin{split}
Q(t_1+t_2)&\stackrel d=( Q_0+P_\lambda(0,t_1)J+P_\lambda(t_1,t_1+t_2)J-t_1-t_2)\vee0\\
&\stackrel d= (Q(t_1)+P_\lambda(0,t_2)J-t_2)\vee0.
\end{split}
\]
The machine breaks when the fluid level reaches $0$, i.e. at the time $T^*$ given by
\[
T^*=\inf\{t|Q(t)=0\}.
\]
When the machine is repaired, it starts with an age of zero again.

To derive the distribution of $Q(t)$ for a certain $t$ from the observed jumps, it seems we need to keep track of the exact times at which these jumps occurred.
This would be a very inconvenient format of the state of the machine.
Luckily, this information can be condensed into a simpler state description.
First, we will illustrate this with the following example:
\begin{example}
	If the first jump of the process would have occur at some time $t$, then:
	\begin{enumerate}
		\item At time $t$, $Q(t)\geq J$.
		Hence, we have a lower bound on the current fluid level.
		\item Initially, the fluid level was at least $t$, i.e. $Q_0\geq t$.
		Hence, we have a certain lower bound on the initial fluid level.
	\end{enumerate}
	If now, after some time $\tau<J$ another jump occurs, we know that:
	\begin{enumerate}
		\item At time $t+\tau$, $Q(t)\geq 2J-\tau$.
		The passage of time has hence decreased our lower bound of the current fluid level by $\tau$ and the jump has increased this bound by $J$.
		\item Our lower bound of the initial fluid level has remained unchanged.
		When our lower bound of the current fluid level is positive, our lower bound of the initial fluid level remains the same.
	\end{enumerate}
\end{example}
This suggests that the only two parameters we need to keep track of, are the lower bound of the current fluid level $L_c(t)$ and the lower bound of the initial fluid level $L_0(t)$.
We will refer to this $L_0$ as the drained initial fluid.
The machine cannot break if 
\[L_c(t)>0,\]
since we know with certainty that there is still remaining fluid.
We will refer to this quantity $L_c(t)$ as the fluid buffer.
The machine breaks whenever
\[L_0(t)=Q_0,\]
i.e. when the initial fluid level is drained.
Hence, we maintain the two quantities $L_c(t)$ and $L_0(t)$ as the description of the state the machine is in.
\[
X(t)=(L_0(t),L_c(t)).
\]
Initially
\[
X(0)=x_{NEW}:=(0,0).
\]
These two quantities evolve in the following way:
\begin{itemize}
	\item When $X(t)=(l_0,l_c)$ and a jump occurs $L_c$ increases by $J$.
	Hence, the state changes in the following way
	\begin{equation}\label{eq:SimpleJumpEvolution}
	(l_0,l_c)\stackrel{J}{\mapsto} (l_0,l_c+J).
	\end{equation}
	\item When the state of the asset at some time $t$ is given by $X(t)=(l_0,l_c)$ and a time period of length $\tau$ passes without a jump occurring or the machine failing, the fluid level decreases by $\tau$.
	If our buffer contains at least $\tau$ fluid (i.e. $l_c>\tau$), then the buffer will decrease by $\tau$.
	If our buffer is empty, the drained initial fluid will increase by $\tau$ as we now know with certainty that $Q(t)>\tau$ since the machine didn't fail.
	If $0<l_c<\tau$. Then first the buffer will drain and the remaining fluid $\tau-l_c$ will be drained from the initial fluid.
	This can be summarized in the following way
	\begin{equation}\label{eq:SimpleAgeEvolution}
	(l_0,l_c)\stackrel\tau\mapsto (l_0+\tau-\min\{l_c,\tau\},l_c-\min\{l_c,\tau\}).
	\end{equation}
	\item When the fluid $Q(t)$ reaches zero, it remains zero until it is being repaired.
\end{itemize}
\begin{theorem}\label{theorem:SimpleCurrentLevel}
	Using the $L_c(t)$ and the $L_0(t)$ as defined above.
	The current fluid level is given by
	\begin{equation}\label{eq:SimpleCurrentLevel}
	Q(t)=L_c(t)+Q_0-L_0(t).
	\end{equation}
	\begin{proof}
		At the start of the process, $X(0)=(0,0)$ so that
		\[
		Q(0)=0+Q_0-0=Q_0.
		\]
		When a jump occurs, we know that $Q(t)$ increases by $J$.
		By \eqref{eq:SimpleJumpEvolution}, $L_c$ also increases by $J$ such that jumps preserve \eqref{eq:SimpleCurrentLevel}.
		When a time $\tau$ passes without a jump occurring, $Q(t)$ decreases by $\tau$.
		From \eqref{eq:SimpleAgeEvolution}, we see that $L_c-L_0$ also decreases by $\tau$.
		This completes the proof.
	\end{proof}
\end{theorem}

\begin{example}\label{example:SimpleFluidState}
	For example, if we have the jump quantity $J=3$ and the initial fluid level for a run of the machine equals $5$.
	If after 1.5 time units and after 2.5 time units a jump occurred, $Q,L_0$ and $L_c$ would evolve as in figure \ref{figure:SimpleFluidExampleQuantities}.
	As you can see, when $Q(t)=0$, it holds that $L_0(t)=Q(0)$.
\end{example}
\begin{figure}[H]\label{figure:SimpleFluidExampleQuantities}
\centering
\setlength\fwidth{0.7\textwidth}
% This file was created by matlab2tikz.
%
%The latest updates can be retrieved from
%  http://www.mathworks.com/matlabcentral/fileexchange/22022-matlab2tikz-matlab2tikz
%where you can also make suggestions and rate matlab2tikz.
%
\definecolor{mycolor1}{rgb}{0.00000,0.44700,0.74100}%
\definecolor{mycolor2}{rgb}{0.85000,0.32500,0.09800}%
\definecolor{mycolor3}{rgb}{0.92900,0.69400,0.12500}%
%
\begin{tikzpicture}

\begin{axis}[%
width=0.951\fwidth,
height=0.75\fwidth,
at={(0\fwidth,0\fwidth)},
scale only axis,
xmin=0,
xmax=12,
xlabel style={font=\color{white!15!black}},
xlabel={Time},
ymin=0,
ymax=8,
ylabel style={font=\color{white!15!black}},
ylabel={Fluid},
axis background/.style={fill=white},
legend style={legend cell align=left, align=left, draw=white!15!black}
]
\addplot [color=mycolor1]
  table[row sep=crcr]{%
0	5\\
1.5	3.5\\
1.51	6.5\\
3.5	4.5\\
3.51	7.5\\
7.5	3.5\\
11	0\\
};
\addlegendentry{$Q$}

\addplot [color=mycolor2]
  table[row sep=crcr]{%
0	0\\
1.5	1.5\\
1.51	1.5\\
3.5	1.5\\
3.51	1.5\\
7.5	1.5\\
11	5\\
};
\addlegendentry{$L_0$}

\addplot [color=mycolor3]
  table[row sep=crcr]{%
0	0\\
1.5	0\\
1.51	3\\
3.5	1\\
3.51	4\\
7.5	0\\
11	0\\
};
\addlegendentry{$L_c$}

\end{axis}
\end{tikzpicture}%
\caption{$Q(t),L_0(t),L_c(t)$ for example \ref{example:SimpleFluidState}.}
\end{figure}

\begin{corollary}\label{corollary:SimpleFluidDistribution}
	Theorem \ref{theorem:SimpleCurrentLevel} implies that for $X(t)=x=(L_0(t),L_c(t))$, $Q(t)\sim F_x$ where
	\begin{equation}\label{eq:SimpleCurrentDistribution}
	\begin{split}
	F_x(q)&:=\mathbb{P}(Q(t)<q)\\
	&=\mathbb{P}(L_c(t)+Q_0-L_0(t)<q|Q_0>L_0(t))\\
	&=\mathbb{P}(Q_0<q+L_0(t)-L_c(t)|Q_0>L_0(t))\\
	&=\frac{F(q+L_0(t)-L_c(t))-F(L_0(t))}{\bar{F}(L_0(t))}.\\
	\end{split}
	\end{equation}
\end{corollary}
\begin{remark}
	Note that we can write $L_0(t)$ in the following way:
	\[
	L_0(t)=\max\limits_{0\leq\tau\leq t} Q_0-Q(\tau).
	\]
	As by \eqref{eq:SimpleCurrentLevel}, 
	\[
	L_0(t)-L_c(t)=Q_0-Q(t),
	\]
	and $L_0$ cannot decrease and only increases whenever $L_c=0$.
\end{remark}
\begin{corollary}
	From the previous remark, it can also be seen easily that the machine will fail when the used initial fluid equals the initial fluid level as
	\[
	L_0(t)=Q_0\Rightarrow Q(t)=0.
	\]
\end{corollary}

We use the same discretization $t_k=\delta k$ as in the previous chapter.
For $x_k=X(t_k)$, we describe the random variables $\omega_k=\omega_k(x_k)$ of the Markov decision process in the following way
\[
\omega_k(x_k):=\begin{cases}
\Omega_{SURVIVE}&\ \begin{split}&\text{if the machine does not break and no jump occurs}\\
&\text{in the $k$'th time interval.}\end{split}\\
\Omega_{BREAK}&\ \text{if the machine breaks in the $k$'th time interval.}\\
\Omega_{JUMP}&\ \begin{split}&\text{if the machine does not break and a jump occurs}\\
&\text{in the $k$'th time interval.}\end{split}\\
\end{cases}
\]
Note that the machine can only break whenever $L_c(t)=0$.
Assuming only one jump can occur in a time interval and letting $x_k=X(t_k)=(l_0,l_c)$, we get the following probabilities:
\[
\begin{split}
\mathbb{P}(\omega_k(x_k)=\Omega_{SURVIVE})&=\begin{cases}
e^{-\lambda \delta}=1-\delta\lambda+o(\delta^2)&\text{ if }l_c>0,\\
\begin{split}
e^{-\lambda \delta} & \bar{F}_{x_k}(l_0+\delta)\\
=1&-\delta h(l_c)-\delta\lambda+o(\delta^2)
\end{split}&\text{ if }l_c=0.\\
\end{cases}\\
\mathbb{P}(\omega_k(x_k)=\Omega_{BREAK})&=\begin{cases}
0&\text{ if }l_c>0,\\
\begin{split}
e^{-\lambda \delta}F_{x_k}&(l_0+\delta)\\
=&\delta h(l_c)+o(\delta^2)
\end{split}&\text{ if }l_c=0.\\
\end{cases}\\
\mathbb{P}(\omega_k(x_k)=\Omega_{JUMP})&=\begin{cases}
1-e^{-\lambda \delta} & \text{ if }l_c>0,\\
\begin{split}
1-&e^{-\lambda \delta}\\
=&\delta\lambda+o(\delta^2)
\end{split}&\text{ if }l_c=0.\\
\end{cases}
\end{split}
\]
Where $F_x$ is given by corollary \ref{corollary:SimpleFluidDistribution}.

\subsection{Control actions}
We introduce a state $x_{BREAK}$ for when the machine is broken.
and in this state the only available action is $a_R$.
In all other states, both actions $a_W$ and $a_R$ may be chosen.
The definitions of these actions remains the same as in the definition of the age-based maintenance problem.

\subsection{State evolution}
Initially $x_{NEW}=(0,0)$.
The state of the Markov decision process now evolves in the following way:
\[
x_{k+1}=f(x_k, u_k, \omega_k):=\begin{cases}
x_{NEW}&\ \text{if }u_k=a_R,\\
\begin{split}(l_0+\delta-\min\{l_c,\delta\}&,\\l_c-\min\{l_c,\delta\}&)\end{split}&\ \text{if }u_k=a_W\text{ and }\omega_k=\Omega_{SURVIVE},\\
(l_0,l_c+J-\delta)&\ \text{if }u_k=a_W\text{ and }\omega_k=\Omega_{JUMP},\\
x_{BREAK}&\ \text{if }u_k=a_W\text{ and }\omega_k=\Omega_{BREAK}.\\
\end{cases}
\]
Here we assumed that jumps occur at the start of time intervals.
For small $\delta$, this approximates the fluid model defined above.
Again, we use the definition of the random variable $S(x_k):=f(x_k,a_W,\omega_k(x_k))$ as the state after $x_k$.

\subsection{Costs and discounting}
The costs and discounting remain the same as in the age-based maintenance problem.

\subsection{Optimal policy and Bellman equations}
In the next section, we will prove that the optimal policy is a stationary policy.
Hence, we want to find a stationary policy $\mu:X\rightarrow \{a_W,a_R\}$ that chooses the action $u_k=\mu(x_k)$ that minimizes the expected total discounted cost $V_\delta(x_k,\mu)$ for each state.
Similarly as in the definition of age-based maintenance, $V_\delta(x_k,\mu)$ is given by
\[V_\delta(x_k,\mu)=g(x_k,\mu(x_k))+\alpha_\delta \mathbb{E}[V_\delta(S(x_k),\mu)].\]
The Bellman equations for the optimal cost $V_\delta(x_k,\mu^*)$ read
\begin{equation}\label{eq:SimpleFluidBellman}
V_\delta(x_k,\mu^*)=\begin{cases}
c+a+\alpha_\delta V_\delta(x_{NEW},\mu^*),&\text{ if }x_k=x_{BREAK},\\
\min\left\{\begin{split}&c+\alpha_\delta V_\delta(x_{NEW},\mu^*),\\&\alpha_\delta \mathbb{E}[V_\delta(S(x_k),\mu^*)]\end{split}\right\},&\text{else.}\\
\end{cases}
\end{equation}
$\mu$ is optimal if $V_\delta(x,\mu)=V_\delta(x,\mu^*)$ for all $x$.
$\mathbb{E}[V_\delta(S(x_k),\mu^*)]$ when $x_k\neq x_{BREAK}$ is given by 
\begin{equation}\label{eq:SimpleFluidNextState}
\begin{split}
&\mathbb{E}[V_\delta(S(l_0,l_c),\mu^*)]\\
&=\begin{cases}
\begin{split}
&(1-e^{-\lambda \delta})V_\delta(l_0,l_c+J-\delta,\mu^*)\\
&+e^{-\lambda \delta}V_\delta(l_0,l_c-\delta,\mu^*),
\end{split}&\ \text{If $l_c=0$,}\\
\begin{split}
&e^{-\lambda \delta} \bar{F}_{t_k}(l_0+\delta)V_\delta(l_0+\delta,0,\mu^*)\\
&+ e^{-\lambda \delta}F_{t_k}(l_0+\delta)V_\delta(x_{BREAK},\mu^*)\\
&+(1-e^{-\lambda \delta})V_\delta(l_0,J-\delta,\mu^*),
\end{split}&\ \text{If $l_c>0$.}\\
\end{cases}\\
&=\begin{cases}
\begin{split}
&\lambda\delta V_\delta(l_0,l_c+J-\delta,\mu^*)\\
&+(1-\lambda \delta)V_\delta(l_0,l_c-\delta,\mu^*)+o(\delta^2),
\end{split}
&\ \text{If $l_c=0$,}\\
\begin{split}
&(1-\lambda \delta-\delta h(l_0))V_\delta(l_0+\delta,0,\mu^*)\\
&+ \delta h(l_0)V_\delta(x_{BREAK},\mu^*)\\
&+\lambda \delta V_\delta(l_0,J-\delta,\mu^*)+o(\delta^2),
\end{split}&\ \text{If $l_c>0$.}\\
\end{cases}
\end{split}
\end{equation}

\subsection{Alternative models}
The occurrence of fluid jumps can be modeled in many different ways.
We will briefly mention some alternatives to design choices that were made in the definition above.
\subsubsection{Decisions at jumps only}
We could also model the problem such that the choice to repair the machine can only be made the instant after a jump occurs.
This might be more realistic as the jump could be caused by some mechanic performing some partial maintenance and a mechanic might be needed to completely repair the machine.

\subsubsection{Jumps not according to a Poisson process}
The time in between the jumps could also have another distribution than the exponential distribution.
This would, however, make the problem significantly more complicated as the memorylessness simplifies the problem slightly.
% Structure of optimal policy
\section{Structure of optimal policy}
In this section, we will establish that for the simple fluid model with jumps, the optimal policy is a stationary policy to repair whenever the buffer is empty and the used initial fluid $L_0(t)$ exceeds a certain control limit.
By this, we mean that there exists some optimal control limit $\mu^*>0$ and repair should be chosen whenever the buffer $L_c(t)=0$ and $L_0(t)\geq\mu^*$.

\subsection{Stationary policy}
%From the Bellman equations \eqref{eq:SimpleFluidBellman}, it can be seen that when $x_k\neq x_{BREAK}$, repair is chosen when
%\[ c+\alpha_\delta V_\delta(x_{NEW},\mu^*) <\alpha_\delta \mathbb{E}[V_\delta(S(x_k),\mu^*)]. \]
%Where the left hand side is a constant and the right hand side only depends on $S(x_k)$, which only depends on $L_0$ and $L_c$ and not on $k$ by theorem \ref{theorem:SimpleCurrentLevel}.
%Hence, we have established that the optimal policy must again be a stationary policy.
Any discretization of $L_0$ and $L_c$ results in a countable state space and the cost function is again independent of the decision epoch.
Hence, by section 6.2.4 of \cite{Puterman2008}, any discrete approximation of the continuous-time MDP has a stationary optimal policy.

\subsection{Empty buffer}
Repairing when $L_c>0$ cannot be optimal since it is certain that the machine will not break in the next $L_0$ time units and waiting more would decrease the cost.
This proves that in the optimal policy $L_c$ must be zero whenever repair is chosen.

\subsection{Control limit}
The optimal policy is a control limit policy by the same argument as for the age-based problem (see section \ref{section:AgeBasedControlLimit}). 
%For states $x\neq x_{BREAK}$, repair is chosen whenever
%\begin{equation}\label{eq:SimpleFluidRepairCondition}
%c+\alpha_\delta V_\delta(x_{NEW},\mu^*) <\alpha_\delta \mathbb{E}[V_\delta(S(x_k),\mu^*)].
%\end{equation}
%
%Now we can distinguish two cases:
%\begin{enumerate}
%	\item There is no value $l_0$ so that $x=(l_0,0)$ satisfies \eqref{eq:AgeBasedRepairCondition} and preventive repair is never the optimal choice.
%	In this case we set control limit $\mu^*=\infty$.
%	\item There are values $\mu$ so that $x=(\mu,0)$ satisfy \eqref{eq:AgeBasedRepairCondition}.
%	We can now simply set the control limit $\mu^*$ to the smallest such values.
%	What happens for larger values than this $\mu^*$ is not relevant as these will never be reached.
%\end{enumerate}

Hence, we have established that the optimal policy is a control limit on the used initial fluid $L_0$.
% Computation of total discounted cost
\section{Computation of total discounted cost}
In this section the expected total discounted cost is calculated corresponding to following a control limit policy with control limit $\mu$.
As in the age-based maintenance problem, we can use the Bellman equations to find differential equations to which the total discounted cost adheres.
But for this problem, we will opt for the simpler approach of directly calculating the total discounted costs corresponding to the policies.

%When in a run of the asset $Q_0\leq\mu$, we know that we will have to correctively repair after this run.
%Else, we will have to preventively repair.
The challenge lies in calculating the discount over these costs.
In the age-based problem, this was simple as there were no jumps and the discount after $q$ fluid (age) was depleted was simply $e^{-\beta q}$.
In this problem, we need the distribution of the time it takes until $q$ initial fluid is depleted.

\subsection{Time until control limit}
Let $T_t(q)$ be the random variable denoting the time until the fluid level has decreased $q$ with respect to $Q(t)$, i.e.
$$
T_t(q)=\min\{\tau\geq0|Q(t+\tau)\leq Q(t)-q\}.
$$
Note that, using this definition, $T_0(\mu)$ equals the time until the control limit is reached ($L_0(t)=\mu$) and $T_0(Q_0)$ equals the time until the asset fails.
\begin{lemma}\label{lemma:TimeUntilFluidLemma}
	For any $A,B>0$:
	$A\leq B\Leftrightarrow T_t(A)\leq T_t(B)$
	\begin{proof}
		$\Rightarrow$: 
		\[\begin{split}
		A\leq B&\Rightarrow Q(t)-B\leq Q(t)-A\\
		&\Rightarrow (Q(t+\tau)\leq Q(t)-B\Rightarrow Q(t+\tau)\leq Q(t)-A)\\
		&\Rightarrow T_t(A)\leq T_t(B)\\
		\end{split}\]
		$\Leftarrow$: We will prove that $A>B\Rightarrow T_t(A)> T_t(B)$:\\
		We know that
		$$
		A>B\Rightarrow Q(t)-B > Q(t)-A.
		$$
		Since $Q(t)$ is piecewise continuous and does not decrease at the discontinuities, we know that 
		$$
		Q(t+T_t(B))=Q(t)-B>Q(t)-A\Rightarrow T_t(A)> T_t(B).
		$$
	\end{proof}
\end{lemma}
To find the distribution of $T_t(q)$, we will condition on the number of jumps.
Let $N_t(q)$ be the random variable denoting the number of jumps that occur in the interval $(t,t+T_t(q)]$.
We will now compute its distribution.
The probability that zero jumps occur equals the probability that the exponentially distributed time interval is larger than $q$:
$$
\mathbb{P}(N_t(q)=0)=e^{-\lambda q}.
$$
The probability that exactly one jump occurs equals the probability that exactly one Poisson event happens in the interval $(t,t+q]$ while none happen in $(t+q,t+q+J]$. Resulting in
\begin{equation}\label{eq:AgeBasedNumJumpProb1}
\mathbb{P}(N_t(q)=1)=\lambda q e^{-\lambda q} e^{-\lambda J}=\lambda q e^{-\lambda (q+J)}.
\end{equation}

For each $k\geq0$, by conditioning on the time until the first jump, we get the following recursion
\begin{equation}\label{eq:SimpleFluidJumpsRecursion}
\begin{split}
\mathbb{P}(N_t(q)=k+1)&=\int\limits_0^{q}\lambda e^{-\lambda x}\mathbb{P}(N_t(q-x+J)=k)dx\\
&=\int\limits_0^{q}\lambda e^{-\lambda (q-y)}\mathbb{P}(N_t(y+J)=k)dy,
\end{split}
\end{equation}
since after this first jump, the fluid level equals $q-x+J$ and $k$ jumps should occur. The second equality follows after the substition $y=q-x$.
The solution of this recursion, is given by the following lemma.
\begin{lemma}
	The probability that $k$ jumps occur before the fluid is decreased by $q$ equals
	
	\[\mathbb{P}(N_t(q)=k)=\lambda q\frac{(\lambda (q+kJ))^{k-1}}{k!}e^{-\lambda(q+kJ)}.\]
	
	\begin{proof}
		By using $k=1$, it can be seen that this satisfies \eqref{eq:AgeBasedNumJumpProb1}.
		By substituting this into \eqref{eq:SimpleFluidJumpsRecursion}, we see that the recursion is also satisfied.
	\end{proof}
\end{lemma}

Now we can define the quantity $D(q)$ as the expected discount over the time until $q$ initial fluid is used in the following way
\begin{equation}\label{eq:SimpleFluidDiscountDefinition}
D(q):=\mathbb{E}[e^{-\beta T_t(q)}]=\sum\limits_{k=0}^\infty e^{-\beta(q+kJ)}\mathbb{P}(N_t(q)=k).
\end{equation}
Note that this $D$ does not depend on $t$ because the time intervals in between jumps are memoryless.
This quantity has the following properties:
\begin{lemma}\label{lemma:SimpleFluidDiscountLogLinear}
	\[D(A)D(B)=D(A+B).\]
	\begin{proof}
		\begin{equation}
		\begin{split}
		D(A)D(B)&=\left[\sum\limits_{k=0}^\infty e^{-\beta(A+kJ)}\mathbb{P}(N_t(A)=k)\right]\left[\sum\limits_{m=0}^\infty e^{-\beta(B+mJ)}\mathbb{P}(N_t(B)=m)\right]\\
		&=\sum\limits_{n=0}^\infty e^{-\beta(A+B+nJ)}\sum\limits_{k=0}^n \mathbb{P}(N_t(A)=k)\mathbb{P}(N_t(B)=n-k)\\
		&=\sum\limits_{n=0}^\infty e^{-\beta(A+B+nJ)} \mathbb{P}(N_t(A+B)=n).\\
		\end{split}
		\end{equation}
		The last step holds since on the second last line, the second sum equals the probability that $n$ jumps occur before a fluid quantity $A+B$ is drained, conditioned on the number of jumps that occur before a quantity $A$ is drained.
	\end{proof}
\end{lemma}

\begin{lemma}\label{lemma:SimpleFluidDiscountODE}
	$$\frac{d}{dA}D(A)=-(\beta+\lambda)D(A)+\lambda D(A+J)$$
	\begin{proof}
		This can be seen from taking the derivative of \eqref{eq:SimpleFluidDiscountDefinition}.
	\end{proof}
\end{lemma}
Using these lemmas we can find a simpler expression for $D(q)$ than \eqref{eq:SimpleFluidDiscountDefinition}:
\begin{theorem}\label{theorem:SimpleFluidDiscountExpression}
	The expected discount over the time until the fluid has decreased by $q$, is given by
	\[
	D(A)=e^{-(\beta+\lambda(1-D(J)))A}.
	\]
	\begin{proof}
		Using lemma \ref{lemma:SimpleFluidDiscountLogLinear}, we can rewrite the derivative of $D(A)$ given by \ref{lemma:SimpleFluidDiscountODE} to
		\[
		\frac{d}{dA}D(A)=-(\beta+\lambda)D(A)+\lambda D(A)D(J)=-(\beta+\lambda(1-D(J)))D(A).
		\]
		Which leads to solution
		\[
		D(A)=Ce^{-(\beta+\lambda(1-D(J)))A}.
		\]
		For some integration constant $C$. Since $D(0)=1$, we know that $C=1$, which completes the proof.
	\end{proof}
\end{theorem}
\begin{corollary}
	The value $D(J)$ is now implicitly given by
	\[
	D(J)=e^{-(\beta+\lambda(1-D(J)))J}.
	\]
	This quantity can be approximated by a method of successive approximation.
	For the parameters that were used in appendix \ref{AppendixComputationsTable}, this quantity converged within ten iterations up to five decimals.
\end{corollary}
\begin{remark}\label{remark:SimpleFluidDiscountEquivalence}
	From theorem \ref{theorem:SimpleFluidDiscountExpression}, it can be seen that this expected discount factor of the simple fluid model is actually the same as the discount factor for the age-based model with adjusted discount exponent $\beta^*=\beta+\lambda(1-D(J))$.
	That is,
	\[
	D(q)=e^{-\beta^*q}.
	\]
\end{remark}

Now we will derive the expected total discounted cost:
If $T_0(Q_0)\leq T_0(\mu)$, the asset will break in the first run and the expected value at which the corrective repair cost is discounted is $\mathbb{E}[D(Q_0)|T_0(Q_0)\leq T_0(\mu)]$.
Note that $\mathbb{P}(T_0(Q_0)\leq T_0(\mu))=\mathbb{P}(Q_0\leq\mu)$ by lemma \ref{lemma:TimeUntilFluidLemma}.
If $T_0(Q_0)> T_0(\mu)$, the expected value at which the preventive repair cost is discounted is $D(\mu)$.
This results in the following expression for the expected total discounted cost:
\begin{theorem}
	The expected total discounted cost of an asset with control limit $\mu$, is given by
	\begin{equation}\label{eq:SimpleFluidImplicitTDC}
	V(x_{NEW},\mu)=F(\mu)\mathbb{E}[D(Q_0)|Q_0\leq \mu]a+\mathbb{E}[D(Q_0\wedge\mu)](c+V(x_{NEW},\mu)).
	\end{equation}
\end{theorem}

\begin{corollary}
	Alternatively \eqref{eq:SimpleFluidImplicitTDC} can be written as
	\begin{equation}\label{eq:SimpleFluidExplicitTDC}
	V(x_{NEW},\mu)=\frac{F(\mu)\mathbb{E}[D(Q_0)|Q_0\leq \mu]a+\mathbb{E}[D(Q_0\wedge\mu)]c}{1-\mathbb{E}[D(Q_0\wedge\mu)]}.
	\end{equation}
\end{corollary}

\begin{remark}\label{remark:SimpleFluidTDCEquivalence}
	Using remark \ref{remark:SimpleFluidDiscountEquivalence}, we can see that \eqref{eq:SimpleFluidImplicitTDC} is again the same as for the age-based maintenance problem \eqref{eq:AgeBasedHatTDC} for the adjusted discount factor $\beta^*=\beta+\lambda(1-D(J))$.
	%Hence, the same methods can be used to find the optimal policy and the corresponding total discounted cost.
\end{remark}
% Heuristic policies
\section{Heuristic policies}
Instead of directly trying to find the optimal policy for the proposed simple fluid model, we will first try a heuristic policy.
If we compare the current problem with the age-based maintenance problem and look at the differences concerning the control limit that the jumps introduce, we see the following two differences:
\begin{enumerate}
	\item The presence of jumps increases the time until the control limit is reached, which decreases the discount factors, which decreases the expected total discounted cost.
	\item The presence of jumps changes the structure of the control limit slightly.
	The fact that a jump can occur between now and the breakdown of the asset, increases the control limit.
\end{enumerate}
The first difference can be anticipated using the new formula for the expected total discounted cost \eqref{eq:SimpleFluidExplicitTDC}.
The second difference could be dealt with by simply assuming no jump will occur before the next failure.
In this way, the control limit will be given by the same equation as for the simple discounted problem \eqref{eq:AgeBasedHazardBound}, but with the new total discounted cost $V(x_{NEW},\mu^*)$ on the right hand side.
This results in a heuristic policy for which the same iteration methods can be used with the same convergence properties as in the previous chapter.
Appendix \ref{AppendixComputationsTable} contains total discounted costs and control limits using this heuristic policy for various problem parameters.
% The optimal policy
\section{Optimal policy}
In this section, we will look for the optimal control limit $\mu^*$.
From remark \ref{remark:SimpleFluidDiscountEquivalence}, it follows that if we use the adjusted discount $\beta^*=\beta+\lambda(1-D(J))$, this problem completely reduces to the age-based maintenance problem.
Hence, we can use \eqref{eq:AgeBasedHazardBound} with this adjusted discount as condition for the optimal control limit.
The same properties proven in chapter \ref{chapter:AgeBased} also apply for this problem and the same policy iteration can also be applied to compute the optimal control limit and corresponding expected total discounted cost.
There is one important conceptual difference between the control limit of the age-based problem and this simple fluid problem:
if $\lambda J>1$, the system is instable and the fluid level can diverge to infinity.
This means that there is a positive probability that the control limit will never be reached (i.e. $T_t(Q_0\wedge\mu)=\infty$) and that the asset will never fail.
% Structural properties
\section{Structural properties}\label{section:SimpleStructuralProperties}
In this section, the effect of changing the parameters to the expected total discounted cost and the control limit are investigated.
By the equivalence explained in the previous chapter, the same structural properties as in section \ref{section:AgeBasedStructuralProperties} for the parameters $c,a$ and $h$ apply.
Remarks \ref{remark:AgeBasedControlLimitDiscountIncrease}, \ref{remark:AgeBasedControlLimitDiscountAndHazardIncrease} and \ref{remark:AgeBasedTDCDiscountIncrease} also apply for the adjusted discount $\beta^*$.
Hence, we discuss the effect changing $\lambda,J$ and $\beta$ has to the adjusted discount $\beta^*$.
\[
\beta^*=\beta+\lambda(1-D(J))
\]
\begin{remark}
	Obviously, increasing $\beta$ will result in an increase in $\beta^*$.
\end{remark}
\begin{remark}
Similarly, increasing $\lambda$ will result in an increase in $\beta^*$ if $J>0$ ($J=0$ implies $D(J)=1$).
This means that frequent jumps increase the control limit and decrease the total discounted cost, which seems natural.
\end{remark}
\begin{remark}
	Increasing $J$ results in a decrease in $D(J)$, which results in an increase in $\beta^*$.
	This means that greater jumps, again, increase the control limit and decrease the total discount cost, which also seems natural.
\end{remark}
Then there is also another important subtlety: For the age-based maintenance problem we needed to assume that the distribution of the age of the asset has an increasing hazard rate.
Similarly, we need the assumption that the distribution of the initial fluid level has an increasing hazard rate.
This does, however, not mean that the lifetime distribution of this asset with fluid jumps has an increasing hazard rate.
For instance if the system is unstable ($\lambda J>1$), the probability of ever reaching an empty fluid level decreases as the fluid level increases and the expected fluid level increases over its age so that the hazard will be decreasing.

Appendix \ref{AppendixComputationsTable} contains computed values of the optimal control limit and the corresponding expected total discounted cost.

\chapter{Markov Modulated Fluid Model with jumps}
In this chapter, we extend the model of the simple fluid model with jumps from the previous chapter to a Markov modulated fluid model (MMFM) with various states, fluid rates, various fluid jump sizes and rates.
The fluid jumps occur when a transition occurs in the underlying Markov chain.
The MMFM that is used is similar to the first-order fluid model considered in \cite{Gribaudo2007}, with the addition of constant jumps.
The size of the jumps are constant.
The various fluid rates, transition rates and jump sizes introduce the following complications with regard to computing the total discounted cost and the optimal policy:
\begin{enumerate}
	\item The computation of the expected discount factors $D(q)$ is more difficult as we need to take multiple
	paths with different probabilities and jump sizes into account.
	Moreover, it is also relevant from which state you start and where you end.
	\item There are different control limits $\mu_i$ for different states $s_i$, all depending on each other.
	\item A jump can cause the machine to be repaired when the amount of used fluid has exceeded the control limit.
\end{enumerate}
These complications will be explained and tackled in the following sections.
% Problem formulation and definition
\section{Problem formulation and definition}
In this section, we extend the problem definition of preventive maintenance on an asset modeled as the simple fluid model from the previous chapter.
First we extend the stochastic process from the last chapter to a MMFM, then we define a Markov decision process.

\subsection{Stochastic asset degradation}
First, we define a left-continuous Continuous Time Markov Chain (CTMC) $S(t)$ with state space $S=\{s_1,...,s_N\}$, initial state $s_1$ and transition rates $\lambda_{ij}$ from state $s_i$ to $s_j$, with $\lambda_{ii}=-\sum_{j\neq i}\lambda_{ij}$.
Furthermore, we define $\lambda_i=-\lambda_{ii}$ for convenience.
In figure \ref{figure:MmfmNotation}, a MMFM with states $s_i,s_j$ is drawn.
We draw MMFMs with multiple states in a similar way.
\begin{figure}
	\centering
	\begin{tikzpicture}[->,>=stealth',shorten >=1pt,auto,node distance=3cm,
	thick,main node/.style={circle,draw,font=\sffamily\Large\bfseries}]
	
	\node[main node] (1) {$s_i:r_i$};
	\node[main node] (2) [right of=1] {$s_j:r_j$};
	
	\path[every node/.style={font=\sffamily\small}]
	(1) edge [bend left] node {$\lambda_{ij}:J_{ij}$} (2)
	(2) edge [bend left] node {$\lambda_{ji}:J_{ji}$} (1);
	\end{tikzpicture}
	\caption{A drawing of a MMFM with two states $s_i$ and $s_j$ with fluid rates $r_i$ and $r_j$ respectively.
		The transition from $s_i$ to $s_j$ has rate $\lambda_{ij}$ and jump quantity $J_{ij}$.
		Similarly, the transition from $s_j$ to $s_i$ has rate $\lambda_{ji}$ and jump quantity $J_{ji}$}
	\label{figure:MmfmNotation}
\end{figure}
We let $S(t)$ denote the (index of the) state the CTMC is in at time $t$.
Similar to the previous chapter, we represent the fluid level at time $t$ by $Q(t)$ with $Q(0)=Q_0\sim F$.
When the CTMC is in state $s_i$, the fluid level $Q(t)$ decreases with rate $r_i>0$:
\[
\frac{d}{dt}Q(t)=-r_{S(t)}.
\]
When a transition from $s_i$ to $s_j$ occurs, the fluid rate instantaneously increases by $J_{ij}\geq 0$.
Again, the asset breaks when the fluid reaches zero so that the process is absorbing at $Q(t)=0$.
When the asset is repaired, the process is restarted with an initial fluid level with distribution $F$.

Similar to the simple fluid model, the state information can be condensed with a few variables.
These are the amount of used initial fluid $L_0(t)$, the buffer level $L_c(t)$ and the current CTMC-state $S(t)$.
The definitions of $L_0$ and $L_c$ are exactly the same as in the previous chapter:
\begin{enumerate}
	\item $L_0(t)$ is the lower bound of the initial fluid level $Q_0$ known at time $t$.
	\item $L_c(t)$ is the lower bound of the current fluid level $Q(t)$ known at time $t$.
\end{enumerate}
Hence,
\[
X(t)=(S(t),L_0(t),L_c(t)),
\]
when the asset is not broken (i.e. the fluid level is positive) and initially $X(0)=x_{NEW}=(1,0,0)$.
Again, we define a special state $x_{BREAK}$ for when the asset is broken.
When $X(t)=x_{BREAK}$, the quantities $L_0(t)$ and $L_c(t)$ are undefined.

The values $L_0$ and $L_c$ evolve in a similar way as in the previous chapter:
\begin{itemize}
	\item When a transition from $s_i$ to $s_j$ occurs at time $t$, the buffer $L_c$ increases by $J_{ij}$:
	\begin{equation}\label{eq:MmfmJumpEvolution}
	X(t^+)=(j,L_0(t),L_c(t)+J_{ij}).
	\end{equation}
	\item When no jump occurs in a neighborhood of $t$, $S(t)$ remains constant and $L_0$ and $L_c$ evolve according to
	\begin{equation}\label{eq:MmfmAgeEvolution}
	\begin{split}
	\frac{d}{dt}L_c(t)&=\begin{cases}
	0&\ \text{if }L_c(t)=0,\\
	-r_{S(t)}&\ \text{else.}
	\end{cases}\\
	\frac{d}{dt}L_0(t)&=\begin{cases}
	0&\ \text{if }L_c(t)>0,\\
	r_{S(t)}&\ \text{else.}
	\end{cases}
	\end{split}
	\end{equation}
	Note that we then always have
	\[
	\frac{d}{dt}\left[L_c(t)-L_0(t)\right]=\frac{d}{dt}Q(t)=-r_{S(t)}.
	\]
	\item When the asset breaks (i.e. whenever $Q(t)=0$ or equivalently $L_0(t^-)=Q_0$) the process moves to $x_{BREAK}$:
	\[
	X(t)=x_{BREAK}.
	\]
\end{itemize}
With these definitions, $Q(t)$ is again given by \eqref{eq:SimpleCurrentLevel} and it has distribution $F_{X(t)}(q)$,  again given by \eqref{eq:SimpleCurrentDistribution}.

\begin{example}\label{example:MmfmState}
	Consider the MMFM depicted by figure \ref{figure:MmfmExample}.
	Now consider the following run of the asset:
	\begin{itemize}
		\item The asset starts in $s_1$ with initial fluid level $5$
		\item After 1 time unit, a transition to $s_2$ occurs
		\item The asset stays in $s_1$ for half a time unit, then it transitions to $s_3$
		\item It stays in this state for $1.5$ time units before transitioning to $s_1$ again.
		\item Here it breaks after $2$ time units.
	\end{itemize}
	Figure \ref{figure:MmfmExampleQuantities} shows the evolution of the quantities $Q(t),L_(t)$ and $L_c(t)$ over this time period.
	As you can see, $L_0(t)$ is nondecreasing and the asset breaks when $L_0(t)=Q(0)$.
\end{example}
\begin{figure}
\centering
\begin{tikzpicture}[->,>=stealth',shorten >=1pt,auto,node distance=3cm,
thick,main node/.style={circle,draw,font=\sffamily\Large\bfseries}]

\node[main node] (1) {$s_1:1$};
\node[main node] (2) [right of=1] {$s_2:4$};
\node[main node] (3) [below right of=1] {$s_3:2$};

\path[every node/.style={font=\sffamily\small}]
(1) edge node {$1:3$} (2)
(2) edge node {$2:0$} (3)
(3) edge node {$2:0$} (1);
\end{tikzpicture}
\caption{Depiction of the MMFM for example \ref{example:MmfmState}.}
\label{figure:MmfmExample}
\end{figure}
\begin{figure}[H]
\centering
\setlength\fwidth{0.7\textwidth}
% This file was created by matlab2tikz.
%
%The latest updates can be retrieved from
%  http://www.mathworks.com/matlabcentral/fileexchange/22022-matlab2tikz-matlab2tikz
%where you can also make suggestions and rate matlab2tikz.
%
\definecolor{mycolor1}{rgb}{0.00000,0.44700,0.74100}%
\definecolor{mycolor2}{rgb}{0.85000,0.32500,0.09800}%
\definecolor{mycolor3}{rgb}{0.92900,0.69400,0.12500}%
%
\begin{tikzpicture}

\begin{axis}[%
width=0.951\fwidth,
height=0.75\fwidth,
at={(0\fwidth,0\fwidth)},
scale only axis,
xmin=0,
xmax=5,
xlabel style={font=\color{white!15!black}},
xlabel={Time},
ymin=0,
ymax=7,
ylabel style={font=\color{white!15!black}},
ylabel={Fluid},
axis background/.style={fill=white},
legend style={legend cell align=left, align=left, draw=white!15!black}
]
\addplot [color=mycolor1]
  table[row sep=crcr]{%
0	5\\
1	4\\
1.01	7\\
1.5	5\\
2	4\\
3	2\\
5	0\\
};
\addlegendentry{Fluid}

\addplot [color=mycolor2]
  table[row sep=crcr]{%
0	0\\
1	1\\
1.01	1\\
1.5	1\\
2	1\\
3	3\\
5	5\\
};
\addlegendentry{Lower}

\addplot [color=mycolor3]
  table[row sep=crcr]{%
0	0\\
1	0\\
1.01	3\\
1.5	1\\
2	0\\
3	0\\
5	0\\
};
\addlegendentry{LowerC}

\end{axis}
\end{tikzpicture}%
\caption{$Q(t),L_0(t),L_c(t)$ for example \ref{example:MmfmState}.}
\label{figure:MmfmExampleQuantities}
\end{figure}

Now, for defining the MDP, we will briefly return to discretized time.
For $x_k=X(t_k)$, we describe the random variables $\omega_k=\omega_k(x_k)$ of the Markov decision process as the (index of the) state of the continuous Markov chain at the next decision stage or $\Omega_{BREAK}$ if the asset will break before then:
\[
\omega_k(x_k):=\begin{cases}
\Omega_{BREAK},&\ \text{if the asset breaks,}\\
S(t_{k+1}),&\ \text{else.}
\end{cases}
\]
A jump then occurs when $S(t_k)\neq\omega_k(x_k)\neq\Omega_{BREAK}$.
Assuming only one jump can occur in a time interval, $\omega_k$ has the following probabilities:
\[
\begin{split}
\mathbb{P}(\omega_k(i,l_0,l_c)=\Omega_{BREAK})&=\begin{cases}
	0&\text{ if }l_c>0,\\
	\begin{split}
	e^{-\lambda_{i}\delta}F_{x_k}(l_0+r_i\delta)&\\
	=\delta r_ih(l_0)+o(\delta^2)&
	\end{split}&\text{ if }l_c=0.\\
\end{cases}\\
\mathbb{P}(\omega_k(i,l_0,l_c)=i)&=\begin{cases}
	e^{-\lambda_i \delta}=1-\delta\lambda_i+o(\delta^2)&\text{ if }l_c>0,\\
	\begin{split}
	&e^{-\lambda_i \delta} \bar{F}_{x_k}(l_0+r_i\delta)\\
	&=1-\delta r_ih(l_0)-\delta\lambda_i+o(\delta^2)
	\end{split}&\text{ if }l_c=0.\\
\end{cases}\\
\mathbb{P}(\omega_k(i,l_0,l_c)=j)&=\begin{cases}
	\frac{\lambda_{ij}}{\lambda_i}(1-e^{-\lambda_i \delta})=\delta\lambda_{ij}+o(\delta^2) & \text{ if }l_c>0,\\
	\begin{split}
	\frac{\lambda_{ij}}{\lambda_i}(1-e^{-\lambda_i \delta})\bar{F}_{x_k}(l_0+r_i\delta)&\\
	=\delta\lambda_{ij}+o(\delta^2)&
	\end{split}&\text{ if }l_c=0.\\
\end{cases}
\end{split}
\]
Where $i\neq j$.

\subsection{Control actions}
Similar to the previous chapter, we have a repair action $a_R$ and a wait action $a_W$ and $a_W$ may only be chosen whenever $x_k\neq x_{BREAK}$.
The definitions of these actions remain the same as in the definition of the age-based maintenance problem.

\subsection{State evolution}
Initially, $x_{NEW}=(s_1,0,0)$.
For $x_k=(i,l_0,l_c)$, the state of the Markov decision process now evolves in the following way:
\[
x_{k+1}=f(x_k, u_k, \omega_k):=\begin{cases}
x_{NEW}&\ \text{if }u_k=a_R,\\
\begin{split}(i,l_0+r_i\delta-\min\{l_c,r_i\delta\}&,\\l_c-\min\{l_c,r_i\delta\}&)\end{split}&\ \text{if }u_k=a_W\text{ and }\omega_k=i,\\
(j,l_0,l_c+J_{ij}-r_j\delta)&\ \text{if }u_k=a_W\text{ and }\omega_k=j\neq i,\\
x_{BREAK}&\ \text{if }u_k=a_W\text{ and }\omega_k=\Omega_{BREAK}.\\
\end{cases}
\]
In this definition, we assumed that jumps occur at the start of time intervals.
Again, we use the definition of the random variable $S(x_k):=f(x_k,a_W,\omega_k(x_k))$ as the state after $x_k$.

\subsection{Costs and discounting}
The costs and discounting remain the same as in the age-based maintenance problem.

\subsection{Optimal policy and Bellman equations}
We want to find a stationary policy $\pi= \{\mu_1,...,\mu^N\}$ that chooses the action $u_k=\pi(x_k)=\mu_i(i,l_0,l_c)$ that minimizes the expected total discounted cost $V_\delta(x_k,\pi)$ for each state $x_k=(i,l_0,l_c)$.
Similarly as in the definition of age-based maintenance, $V_\delta(x_k,\pi)$ is given by
\[V_\delta(x_k,\pi)=g(x_k,\pi(x_k))+\alpha_\delta \mathbb{E}[V_\delta(S(x_k),\pi)].\]
The Bellman equations for the optimal cost $V_\delta(x_k,\pi^*)$ read
\begin{equation}\label{eq:MmfmBellman}
V_\delta(x_k,\pi^*)=\begin{cases}
c+a+\alpha_\delta V_\delta(x_{NEW},\pi^*),&\text{ if }x_k=x_{BREAK},\\
\min\left\{\begin{split}&c+\alpha_\delta V_\delta(x_{NEW},\pi^*),\\&\alpha_\delta \mathbb{E}[V_\delta(S(x_k),\pi^*)]\end{split}\right\},&\text{else.}\\
\end{cases}
\end{equation}
$\pi$ is optimal if $V_\delta(x,\pi)=V_\delta(x,\pi^*)$ for all $x\in X$.
$\mathbb{E}[V_\delta(S(x_k),\pi^*)]$ when $x_k\neq x_{BREAK}$ is given by 
\begin{equation}\label{eq:MmfmNextState}
\begin{split}
&\mathbb{E}[V_\delta(S(i,l_0,l_c),\pi^*)]\\
&=\begin{cases}
\begin{split}
&\sum\limits_{j\neq i}(1-e^{-\lambda_i \delta})V_\delta(j,l_0,l_c+J_{ij}-r_j\delta,\pi^*)\\
&+e^{-\lambda_i \delta}V_\delta(i,l_0,l_c-r_i\delta,\pi^*),
\end{split}&\ \text{If $l_c>0$,}\\
\begin{split}
&e^{-\lambda_i \delta} \bar{F}_{t_k}(l_0+r_i\delta)V_\delta(i,l_0+r_i\delta,0,\pi^*)\\
&+ e^{-\lambda_i \delta}F_{t_k}(l_0+r_i\delta)V_\delta(x_{BREAK},\pi^*)\\
&+\sum\limits_{j\neq i}(1-e^{-\lambda_i \delta})V_\delta(j,l_0,J_{ij}-r_j\delta,\pi^*),
\end{split}&\ \text{If $l_c=0$.}\\
\end{cases}\\
&=\begin{cases}
\begin{split}
&\sum\limits_{j\neq i}\lambda_{ij}\delta V_\delta(j,l_0,l_c+J_{ij}-r_j\delta,\pi^*)\\
&+(1-\lambda_i \delta)V_\delta(i,l_0,l_c-r_i\delta,\pi^*)+o(\delta^2),
\end{split}
&\ \text{If $l_c>0$,}\\
\begin{split}
&(1-\lambda_i \delta-\delta r_ih(l_0))V_\delta(i,l_0+r_i\delta,0,\pi^*)\\
&+ \delta r_ih(l_0)V_\delta(x_{BREAK},\pi^*)\\
&+\sum\limits_{j\neq i}\lambda_{ij} \delta V_\delta(j,l_0,J_{ij}-r_j\delta,\pi^*)+o(\delta^2),
\end{split}&\ \text{If $l_c=0$.}\\
\end{cases}
\end{split}
\end{equation}

\subsection{Continuous-time MDP}
Again, we will consider a continuous-time MDP by letting $\delta\rightarrow0$.

\begin{remark}
	Note that the simple fluid model of the previous chapter corresponds to a MMFM model with two states, both with fluid rate $1$, transition rate $\lambda$ and jump size $J$.
	This MMFM is drawn in figure \ref{figure:SimpleFluidAsMmfm}.
	\begin{figure}[H]\label{figure:SimpleFluidAsMmfm}
		\centering
		\begin{tikzpicture}[->,>=stealth',shorten >=1pt,auto,node distance=3cm,
		thick,main node/.style={circle,draw,font=\sffamily\Large\bfseries}]
		
		\node[main node] (1) {$s_1:1$};
		\node[main node] (2) [right of=1] {$s_2:1$};
		
		\path[every node/.style={font=\sffamily\small}]
		(1) edge [bend left] node {$\lambda:J$} (2)
		(2) edge [bend left] node {$\lambda:J$} (1);
		\end{tikzpicture}
		\caption{The MMFM corresponding to the simple fluid problem from the previous chapter.}
	\end{figure}
\end{remark}

%\begin{corollary}\label{corollary:MmfmNonExponential}
%	Similarly, we can construct a MMFM to model the simple fluid problem to approximate any inter-decision distribution.
%	
%	Since the class Phase-Type distributions is dense in the space of positive continuous distributions \cite{Ocinneide1999}, we can construct a MMFM to approximate any distribution of inter-jump times for the simple fluid problem.
%	For example, in figure \ref{figure:SimpleFluidErlangInterJumps}, we have constructed a MMFM model for a simple fluid problem with Erlang inter-jump times with rate $\lambda$ and shape $3$.
%	In a similar way, the MMFM problem can also be extended to allow nonexponential transition times, by replacing each original CTMC-state by a set of CTMC-states.
%	\begin{figure}[H]\label{figure:SimpleFluidErlangInterJumps}
%		\centering
%		\begin{tikzpicture}[->,>=stealth',shorten >=1pt,auto,node distance=3cm,
%		thick,main node/.style={circle,draw,font=\sffamily\Large\bfseries}]
%		
%		\node[main node] (1) {$s_1:1$};
%		\node[main node] (2) [right of=1] {$s_2:1$};
%		\node[main node] (3) [right of=2] {$s_3:1$};
%		
%		\path[every node/.style={font=\sffamily\small}]
%		(1) edge node {$\lambda:0$} (2)
%		(2) edge node {$\lambda:0$} (3)
%		(3) edge [bend left] node {$\lambda:J$} (1);
%		\end{tikzpicture}
%		\caption{The MMFM corresponding to the simple fluid problem with Erlang($3,\lambda$) distributed inter-jump times.}
%	\end{figure}
%\end{corollary}

\subsection{Alternative models}
Again, there are various alternatives to the design choices that were made in the definition of the MMFM above.
We will briefly mention some alternatives with their characteristics.

\subsubsection{Decisions as jumps only}
Again, we could model the problem so that the choice to repair the asset can only be made at the instant that a transition occurs.
This might be more realistic for similar reasons as for the simple fluid model: The jump could be caused by some mechanic performing some partial maintenance and a mechanic might be needed to completely repair the asset so that CTMC-transitions are the only opportunities to repair the asset.

\subsubsection{Transitions to the same state}
We could also allow transitions from certain CTMC-states to themselves (again at exponentially distributed time intervals).
This could simply be modeled by adding a copy $s'$ for each of these states $s$ to the CTMC (with the same outgoing transitions) and transitions between $s$ and $s'$ with the desired transition rate and jump size.

\subsection{Transitions in a semi-Markov process}
Instead of exponentially distributed time intervals between transitions we could also consider a semi-Markov model where the distributions of the transition times are not exponential.
This complicates the model as we lose the memorylessness property, so that we must keep track of the time from the last transition.
%Since corollary \ref{corollary:MmfmNonExponential} allows us to approximate any inter-decision distribution using a CTMC, it will not be necessary to look into this alternative model.

\subsubsection{Second-order fluid model}
Similarly to the second-order fluid model of \cite{Gribaudo2007}, we could model the depletion of fluid (in between jumps) as Brownian motion.
This would make the model more complicated but might also make it more realistic.
% Structure of optimal policy
\section{Structure of optimal policy}
In this section, we will establish that for the MMFM preventive maintenance problem, the optimal policy is a stationary policy to repair in CTMC-state $s_i$ whenever the buffer $L_c$ is empty and the drained initial fluid $L_0$ exceeds a certain control limit $\mu_i^*$.
Note that 'stationary' in this sense means independent of the time. 
The control limit does depend on the state of the CTMC.

\subsection{Stationary policy}
By the same reasoning as for the previous chapter, we can prove that every discrete approximation of the continuous-time MDP has a countable state space so that section 6.2.4 of \cite{Puterman2008} again proves that the there exists a stationary policy that is discount-optimal.
%By the same reasoning as for the previous problems, we can prove that in each state $x\in X$, the optimal choice depends only on the distribution $F_x$ of the remaining fluid, which only depends on the state $x$ of the process.

\subsection{Empty buffer}
By the same reasoning as for the simple fluid problem, an optimal policy will never perform preventive maintenance if the buffer $L_c$ is non-empty.

\subsection{Control limit}
Compared to the previous problem, proving that the optimal policy is a control limit policy is slightly more difficult:
In the simple fluid problem, it was never possible to reach states 'after the control limit' so that it wasn't relevant whether for higher fluid levels repair should also always be chosen.
For this problem, it is possible that these states are reached.
This is illustrated by the following example:
\begin{example}
	Consider a MMFM with two states and control limits $\mu_1<l<\mu_2$ for some $q$.
	If at some time $t$, $X(t)=(2,l,L_c(t))$, then it is possible that for some $t'>t$, $X(t')=(1,l,0)$ so that $L_0(t')>\mu_1$ and the control limit is exceeded.
\end{example}
Although no rigorous proof was found that asserts the optimal policy is a control limit policy, results from value iteration seem to suggest that this is always the case.
This also seems natural as for states with drained initial fluid $L_0$ higher than the control limit, the asset is in a worse condition than at the control limit and is more likely to fail earlier so that repair would still be chosen after the control limit.

Hence, we will consider stationary policies of the form $\pi=[\mu_1,...,\mu_N]$, where preventive repair is chosen in states $x=(i,l_0,l_c)\in X$ if and only if $l_c=0$ and $l_0\geq \mu_i$.
% Computation of total discounted cost
\section{Computation of total discounted cost}
In this section, the expected total discounted cost when using a control limit policy $\pi=[\mu_1,...,\mu_N]$ is calculated.

Similar to \eqref{eq:SimpleFluidDiscountDefinition}, but keeping in mind that there are multiple CTMC-states, we define
\[
D_{i}^t(q):=\mathbb{E}[e^{-\beta T_t(q)}|S(t)=i],
\]
as the expected discount over the time until the fluid level $Q(t)$ is decreased by $q$, given that the CTMC was in $s_i$ at time $t$.
In this definition, we disregard failures and policies (i.e. we briefly assume that $Q_0=\infty$ and all control limits $\mu_i=\infty$).
Furthermore, we define
\[
D_{ij}^t(q):=\mathbb{E}[e^{-\beta T_t(q)}\mathds{1}\{S(t+T_t(q))=j\}|S(t)=i].
\]
Note that these expectations do not depend on calendar time but only on the state of the background CTMC when the fluid is at level $Q(t)$.
Hence, we will omit $t$ in these notations.
Note that
\[
D_{i}(q)=\sum\limits_{j}D_{ij}(q).
\]
When we take the policy $\pi=[\mu_1,...,\mu_N]$ into account, states cannot be reached via paths that go over a control limit.
We define similar quantities:
\[
\begin{split}
&D_{ij}^t(q,\pi,l)\\
&:=\mathbb{E}\left[e^{-\beta T_t(q)}\mathds{1}\left\{\begin{split}
&S(t+T_t(q)=j),\\
&\forall \tau\in[t,t+T_t(q)):
\begin{split}
&L_c(\tau)>0\\&\text{ or } L_0(\tau)<\mu_{S(\tau)}
\end{split}
\end{split}\right\}\middle| S(t)=i,L_0(t)=l\right],
\end{split}
\]
and
\[
D_{i}^t(q,\pi,l):=\sum\limits_{j:\mu_j\leq l+q}D_{ij}^t(q,\pi,l).
\]
Where these are similar to the earlier defined $D_{ij}$ but without the asset being repaired before fluid level $Q(t)-q$ is reached.
These expectations also do not depend on $t$.
Hence, $t$ will again be omitted in the notation.
If preventive repair is chosen, then it is chosen when $L_0$ equals some random variable $R$.
We define the following quantity
\[
\begin{split}
\Gamma_i^t(q,\pi,l):=\lim\limits_{\delta\rightarrow 0}\frac1\delta&\mathbb{P}(q\leq R<q+\delta|S(t)=i,L_0(t)=l)\\
&\times\mathbb{E}[e^{-\beta T_t(R)}|q\leq R<q+\delta,S(t)=i,L_0(t)=l].
\end{split}
\]
The interpretation of this quantity can be difficult, it could be seen as the density of $R$ multiplied by the expected discount factor given that repair is chosen at this level of $L_0$.
Note that $\Gamma_i^t(q,\pi,l)$ also does not depend on $t$ so that $t$ will again be omitted in the notation.
\begin{remark}
	These defined quantities should be interpreted as neither probabilities nor discount factors but more as a combination of these two: The expected discount factor given some event, multiplied by the probability (or density) of this event.
	We will refer to $D_{ij}(q),D_{ij}(q,\pi,l)$ and $D_{i}(q,\pi,l)$ as 'discounted probabilities' and $\Gamma_i(q,\pi,l)$ as a 'discounted density'.
\end{remark}
These discount quantities will be computed in the next section.
We will now derive the expected total discounted cost of a policy $\pi$.
Each run of the asset can end in two ways:
\begin{enumerate}
	\item The asset breaks,
	\item or preventive maintenance is chosen.
\end{enumerate}
We will split the total discounted cost in terms corresponding to these two scenarios.

\subsubsection{The asset breaks}
When the asset breaks, it means that $L_0(t)$ has reached $Q_0$ without encountering a control limit.
Hence, the repair costs are discounted at $D_1(Q_0,\pi,0)$.
The expected value of this, is given by
\[
\mathbb{E}[D_{1}(Q_0,\pi,0)]=\int\limits_0^\infty f(q)D_{1}(q,\pi,0)dq.
\]
We get the following term in the expression of the total discounted cost
\[
\mathbb{E}[D_{1}(Q_0,\pi,0)](c+a+V(x_{NEW},\pi)).
\]

\subsection{Preventive maintenance is chosen}
When preventive maintenance is chosen, it can either be chosen in a state $x=(i,l_0,0)\in X$ where
\begin{enumerate}
	\item $l_0=\mu_i$, preventive maintenance is chosen 'at the control limit';
	\item or $l_0>\mu_i$, preventive maintenance is chosen 'after the control limit'.
\end{enumerate}
We will again split the total discounted cost in terms corresponding to these two scenarios.

\subsubsection{Preventive maintenance is chosen at the control limit}
In this case, repair is chosen in a state $s_i$ with $l_0=\mu_i$.
For this to happen, it must be the case that $Q_0>\mu_i$, with probability $\bar{F}(\mu_i)$.
Preventive maintenance has a cost $c$, discounted at $D_{i}(\mu_i,\pi,0)$.
Hence, we get the following term:
\[
\sum\limits_i\bar{F}(\mu_i)D_{i}(\mu_i,\pi,0)(c+V(x_{NEW},\pi)).
\]

\subsubsection{Preventive maintenance is chosen after the control limit}
In this case, repair is chosen in some CTMC-state $s_i$ and $l_0>\mu_i$.
For this to be able to happen, $Q_0>l_0$ must hold with probability $\bar{F}(l_0)$.
This event would have cost $c$ and discounted density $\Gamma_0(l_0,\pi,0)$.
Hence, this results in the following term:
\[
\left[\int\limits_1^\infty \bar{F}(q)\Gamma_0(q,\pi,0)dq\right](c+V(x_{NEW},\pi)).
\]
Concluding:
\begin{theorem}
	The expected total discounted cost of a policy $\pi$ is given by
	\begin{equation}\label{eq:MmfmTDC}
	\begin{split}
	V(x_{NEW},\pi)=&\mathbb{E}[D_{1}(Q_0,\pi,0)]\left(c+a+V(x_{NEW},\pi)\right)\\
	&+\left[\int\limits_0^\infty \bar{F}(q)\Gamma_1(q,\pi,0)dq+\sum\limits_i\bar{F}(\mu_i)D_{i}(\mu_i,\pi,0)\right]\left(c+V(x_{NEW},\pi)\right).
	\end{split}
	\end{equation}
\end{theorem}
% Computation of discounted probabilities
\section{Computation of discounted probabilities}
In this section, we will show how to compute the discounted probabilities $D_{ij}(q,\pi,l)$ and $D_{i}(q,\pi,l)$ and the discounted density $\Gamma_i(q,\pi,l)$.
We will do this, by first deriving $D_{ij}(q)$.

\begin{remark}
Note that
\[
D_i^t(q)=\mathbb{E}[e^{-\beta T_t(q)}|S(t)=i]
\]	
equals the value of the moment generating function of $T_t(q)$ (conditioned on $S(t)=i$) evaluated at $-\beta$.
Furthermore, when all jump quantities are 0 (i.e. $J_{ij}=0$ for all $i,j$), $T_t(q)$ corresponds to the busy period of the fluid flow process considered in \cite{Asmussen1994}. 
\cite{Asmussen1994} derives the moment generating function of this busy period, which also corresponds to our results.
\end{remark}

\subsection{Disregarding failures and policies}
We repeat the definition of $D_{ij}(q)$:
\[
D_{ij}^t(q):=\mathbb{E}[e^{-\beta T_t(q)}\mathds{1}\{S(t+T_t(q))=j\}|S(t)=i].
\]
We will now prove a few properties regarding $D_{ij}(q)$:
\begin{lemma}\label{lemma:MmfmDiscountsExponentLinear}
	\[
	D_{ij}(A+B)=\sum\limits_k D_{ik}(A)D_{kj}(B)
	\]
	\begin{proof}
		At the time when the fluid level has decreased by $A$, the process must be in some CTMC-state $s_k$.
		Furthermore, the time until the fluid is decreased by $A$ is independent of the time until the fluid is decreased by $A$ because of the Markov property.
	\end{proof}
\end{lemma}
\begin{lemma}\label{lemma:MmfmDiscountsInfinitisimal}
	For small $\delta$, $D_{ij}(\delta r_i)$ is given by
	\[
	D_{ij}(\delta r_i)=(1-\delta\lambda_i-\delta\beta)\mathds{1}\{i=j\}+\sum\limits_{k\neq i}\delta\lambda_{ij}D_{kj}(J_{ik})+o(\delta^2).
	\]
	\begin{proof}
		In a time period of length $\delta$, either a transition occurs to some state $s_k$ ($k\neq i$) or no transition occurs.
		These have probabilities $\delta\lambda_{ik}+o(\delta^2)$ and $1-\delta\lambda_i+o(\delta^2)$ respectively.
		When a transition from $s_i$ to $s_k$ occurs, the fluid level increases by $J_{ik}$.
		Over this time interval, the discount factor is $1-\delta\beta$.
		Furthermore, $D_{ij}(0)=\mathds{1}\{i=j\}$.
		Putting these together results in
		\[
		\begin{split}
		D_{ij}(\delta r_i)&=(1-\delta\lambda_i)(1-\delta\beta)\mathds{1}\{i=j\}+\sum\limits_{k\neq i}\delta(1-\delta\beta)\lambda_{ik}D_{kj}(J_{ik})+o(\delta^2)\\
		&(1-\delta\lambda_i-\delta\beta)\mathds{1}\{i=j\}+\sum\limits_{k\neq i}\delta\lambda_{ik}D_{kj}(J_{ik})+o(\delta^2).
		\end{split}
		\]
	\end{proof}
\end{lemma}
\begin{lemma}
	$D_{ij}$ adheres to the following differential equation:
	\begin{equation}\label{eq:MmfmDiscountDifferential}
	r_i\frac{d}{dq}D_{ij}(q)=\sum\limits_m\left[\sum\limits_{k\neq i}\lambda_{ik}D_{km}(J_{ik})\right] D_{mj}(q)-(\lambda_i+\beta)D_{ij}(q)
	\end{equation}
	\begin{proof}
		First we write
		\[
		\begin{split}
		D_{ij}(q+\delta r_i)&=\sum_mD_{im}(\delta r_i)D_{mj}(q)\\
		&=\sum\limits_m\left[(1-\delta\lambda_i-\delta\beta)\mathds{1}\{i=j\}+\sum\limits_{k\neq i}\delta\lambda_{ik}D_{kj}(J_{ik})+o(\delta^2)\right]D_{mj}(q)\\
		&=(1-\delta\lambda_i-\delta\beta)D_{ij}(q)+\sum\limits_m\left[\sum\limits_{k\neq i}\delta\lambda_{ij}D_{kj}(J_{ik})\right]D_{mj}(q)+o(\delta^2)\\
		&=D_{ij}(q)+\delta\left( \sum\limits_m\left[\sum\limits_{k\neq i}\lambda_{ik}D_{km}(J_{ik})\right] D_{mj}(q)-(\lambda_i+\beta)D_{ij}(q) \right)+o(\delta^2).\\
		\end{split}
		\]
		If we then subtract $D_{ij}(q)$ from both sides, divide by $\delta$ and let $\delta\rightarrow 0$, we get
		\[
		\begin{split}
		\lim\limits_{\delta\rightarrow 0}\frac{D_{ij}(q+\delta r_i)-D_{ij}(q)}{\delta}&=r_i\frac{d}{dq}D_{ij}(q)\\
		&=\sum\limits_m\left[\sum\limits_{k\neq i}\lambda_{ik}D_{km}(J_{ik})\right] D_{mj}(q)-(\lambda_i+\beta)D_{ij}(q).
		\end{split}
		\]
	\end{proof}
\end{lemma}
Hence, the derivative of $D_{ij}$ is a linear combination of $D_{kj}$.
This suggests defining the following matrix:
\begin{equation}\label{eq:MmfmDiscountGenerator}
\begin{split}
\Lambda^D_{im}:=\begin{cases}
\sum\limits_{k\neq i}\frac{\lambda_{ik}}{r_i}D_{km}(J_{ik})-\frac{(\lambda_i+\beta)}{r_i}&\text{ if }i=m\\
\sum\limits_{k\neq i}\frac{\lambda_{ik}}{r_i}D_{km}(J_{ik})&\text{ else.}
\end{cases}
\end{split}
\end{equation}
Furthermore, if we let $D(q)$ be the matrix with entries $D_{ij}(q)$, we can solve the differential equation \eqref{eq:MmfmDiscountDifferential} in the following way:
\begin{theorem}
	For $\Lambda^D$ as defined above, the solution to differential equation \eqref{eq:MmfmDiscountDifferential} is given by
	\begin{equation}\label{eq:MmfmDiscountMatrixSolution}
	D(q)=e^{\Lambda^D q}.
	\end{equation}
	So that the discounted probability of going from state $i$ to $j$ while $Q(t)$ decreases by $q$, is given by
	\begin{equation}\label{eq:MmfmDiscountEntrySolution}
	D_{ij}(q)=\left(e^{\Lambda^D q}\right)_{ij}.
	\end{equation}
	\begin{proof}
		The differential equation \eqref{eq:MmfmDiscountDifferential} can be rewritten to
		\[
			\frac{d}{dq}D_{ij}(q)=\sum\limits_m\Lambda^D_{im}D_{mj}(q).
		\]
		So that for the matrix $D(q)$, we have the following matrix differential equation
		\[
		\frac{d}{dq}D(q)=\Lambda^D D(q),
		\]
		of which \eqref{eq:MmfmDiscountMatrixSolution} is a solution.
		This combined with $D(0)=I$ completes the proof.
	\end{proof}
\end{theorem}

\begin{remark}
	To compute \eqref{eq:MmfmDiscountMatrixSolution}, we still need the constants $D_{km}(J_{ik})$.
	These $N^3$ values can be estimated using a method of successive approximation where iteratively these values $D_{km}(J_{ik})$ are calculated using \eqref{eq:MmfmDiscountMatrixSolution}.
	For the problem parameters that we used, ten iterations were enough to make these values converge for up to five decimals.
\end{remark}

\begin{remark}\label{remark:MmfmStochasticShortestPath}
	As $L_0(t)$ increases continuously when $L_c(t)=0$ and is constant when $L_c(t)>0$, we know that in each run of the asset for each value $l_0\geq0$, there exists a $t$ so that $L_0(t)=l_0$.
	If we omit all time intervals where $L_c(t)>0$, we can view the asset as a CTMC over $L_0$.
	That is, for the first run of the asset, we consider a CTMC $S^*(l_0)$ where
	\[
	S^*(l_0):=S(\min\{\tau|L_c(\tau)=0\wedge L_0(\tau)=l_0\}).
	\]
	When we would start in a state $s_i$ and pay a certain cost when $X(t)=(i,l_0,0)$, the expected value at which this cost would be discounted, would correspond to \eqref{eq:MmfmDiscountEntrySolution}.
	Transition probabilities of $S^*$ are given by generator $e^{\Lambda^D}$ for $\beta=0$.
	Viewing the process as this adjusted CTMC helps understanding the problem.
\end{remark}

\subsection{Taking policies into account}
When we take policies into account, the following complication arises in computing $D_{ij}(q, \pi,l)$:
In the path from $s_i$ to $s_j$, no state $s_k$ must be visited when $L_0(t)>\mu_k$.
This is summarized by the following lemma.
\begin{lemma}
	Similar to lemma \ref{lemma:MmfmDiscountsExponentLinear}, we have
	\[
	D_{ij}(A+B,\pi,l)=\sum\limits_{\mu_k>l+A} D_{ik}(A,\pi,l)D_{kj}(B,\pi,l+A)
	\]
	\begin{proof}
		The reasoning is the same as for lemma \ref{lemma:MmfmDiscountsExponentLinear}, but with the addition that we also need to keep into account that $L_0$ has been increased by $A$, this explains the $l+A$ on the right hand side.
	\end{proof}
\end{lemma}

\begin{lemma}
	For small $\delta$, we have
	\[
	D_{ij}(r_i\delta,\pi,l)=(1-\delta\lambda_i)\mathds{1}\{i=j\}+\sum\limits_{\mu_k>l+r_i\delta}\delta\lambda_{ij}D_{kj}(J_{ik})+o(\delta^2).
	\]
	\begin{proof}
		The reasoning is the same as in lemma \ref{lemma:MmfmDiscountsInfinitisimal}, but now we know that for $\mu_k\leq l+\delta r_i$, we have that $D_{kj}(J_{ik},\pi,l+\delta r_i)=0$.
	\end{proof}
\end{lemma}
Which suggests that we should replace the generator $\Lambda^D$ by a $\Lambda^D(l_0)$ dependent of the amount of used fluid $l_0$:
\begin{equation}
\begin{split}
\Lambda^D_{im}(l_0,\pi):=\begin{cases}
0&\text{ if }\mu_i<l_0\\
\Lambda^D_{im}&\text{ else.}
\end{cases}
\end{split}
\end{equation}
$D_{ij}(q, \pi,l)$ can now be calculated in the following straightforward way:
\begin{theorem}
	The discounted probabilities $D_{ij}(q, \pi,l)$ are given by
	\[
	D_{ij}(q, \pi,l)=\left(e^{\int_l^{l+q}\Lambda^D(x,\pi)dx}\right)_{ij}.
	\]
\end{theorem}
Now we will calculate the discounted density $\Gamma_i(q,\pi,l)$:

$\Gamma_i^t(q,\pi,l)$ corresponds to repairing when the fluid level $Q(t)$ reaches $Q(t)-q$.
The discounted probability of reaching fluid level $Q(t)-q$ in state $s_j$ equals $D_{ij}(q, \pi,l)$.
When the process reaches this state, the asset can be repaired by transitioning to a CTMC-state $s_k$ where the control limit has already been exceeded.
However, the presence of jumps complicates this:
If the transition from $s_j$ to $s_k$ has a fluid jump, then repair won't be chosen immediately.
This problem is solved by using transition rates $\Lambda^D_kj$ instead of $\lambda_{kj}$ since, referring back to remark \ref{remark:MmfmStochasticShortestPath}, are not interested in time intervals where the buffer $L_c$ is nonempty.
Concluding:

\begin{theorem}
	The discounted density corresponding to repairing when $L_0(t)=l+q$ given that initially the process is in CTMC-state $s_i$ with used initial fluid $l$ is given by
	\[
	\Gamma_i(q,\pi,l)=\sum\limits_{\mu_j>l+q}D_{ij}(q,\pi,l)\sum\limits_{\mu_k<l+q}\Lambda_{jk}^D.
	\]
\end{theorem}


%\begin{remark}
%	Referring back to remark \ref{remark:MmfmStochasticShortestPath}, if we view the process as this adjusted CTMC, the problem simplifies to a stochastic shortest path problem where in each state $s_i$, there is a transition to a terminating state with rate
%	\[
%	-\sum\limits_j \Lambda^D_{ij},
%	\]
%	and a terminating cost of 0.
%\end{remark}
% The optimal policy
\documentclass[a4paper]{thesis}
\usepackage[english]{babel}
\usepackage[utf8x]{inputenc}
\usepackage{amsfonts}
\usepackage{amsmath} % Equation numbering
\usepackage[square,sort,comma,numbers]{natbib}
\usepackage[toc,page]{appendix} % For appendices

\usepackage{graphicx} % Figures
\usepackage{grffile} % For recognising the png-extension
\usepackage{float} % For floating figures

\usepackage{changes} % For track changed extensions
\usepackage{todonotes}

\usepackage{dsfont} % For indicator function

\usepackage{hyperref}
\bibliographystyle{plainnat}

% For example environment
\usepackage{amsthm}
\theoremstyle{definition}
\let\example\relax
\newtheorem{example}{Example}[chapter]
\newtheorem{remark}{Remark}[chapter]
\newtheorem{corollary}{Corollary}[chapter]

% For theorem environment
\newtheorem{theorem}{Theorem}[section]

% For lemma environment
\newtheorem{lemma}{Lemma}

\begin{document}
\chapter{MMFM Optimal Policy}
In this section, the structure of the optimal policy for a machine which deteriorates according to the Markov Modulated Fluid Model will be determined.
We will start with the simple model from earlier and will incrementally add complexity.

\section{One CTMC-state, no jumps}
The initial fluid level is given by $Q_0\sim F(q)$ and the fluid rate in this single state equals $r$.
The hazard rate then equals
\begin{equation}
\begin{split}
\lim\limits_{\delta\rightarrow0}\frac{1}{\delta}\mathbb{P}(rx<Q_0<r(x+\delta)|rx<Q_0)&=\lim\limits_{\delta\rightarrow0}\frac{\mathbb{P}(rx<Q_0<r(x+\delta))}{\delta\mathbb{P}(rx<Q_0)}\\
&=rh(rx).
\end{split}
\end{equation}
The rest of the problem can be solved in the same way as the simple problem.
Hence, the optimal control limit $\mu^*$ is the first $\mu$ such that
\begin{equation}
rh(r\mu)=\beta\frac{c+V_\mu}{a}
\end{equation}
(where $V_\mu$ is the expected total discounted cost when using control limit $\mu$), or $\mu^*=\infty$ else.
\section{One CTMC-state, constant jumps at exponentially distributed time intervals}
We extend the previous model by adding constant fluid jumps to the model at exponentially distributed intervals.
The time intervals between jumps have rate $\lambda$ and fluid quantity $J$.
As earlier, we denote the state of the machine at time $t$ by
$$
X(t)=(S(t),L_0(t),L_c(t)),
$$
but in this simple one-state problem $S(t)=s_0$ for all $t$.
The current state is completely observable, so at time $t$, $L_0(t)$ and $L_c(t)$ are known.
The hazard rate at time $t$ is given by
\begin{equation}
h(t)=\begin{cases}
0&\text{if }L_c(t)>0\\
rh(rL_0(t))&\text{else.}
\end{cases}
\end{equation} 
This suggests that preventive maintenance should be chosen at time $t$ when $L_c(t)=0$ and
\begin{equation}
rh(rL_0(t))=\beta\frac{c+V_{L_0(t)}}{a},
\end{equation}
where $V_{l}$ equals the expected total discounted cost when the machine is repaired whenever $L_0(t)$, the observed lower bound of $Q_0$, equals $l$.
Note that when there are no jumps ($J=0$), this corresponds to the optimal policy (and cost) of the previous problem.
Hence, we will refer to this limit for the lower bound as the control limit $\mu$ and use the notation $V_\mu$ for the corresponding cost.

In the simple problem, we could easily calculate at each time $t$, how long it takes until preventive repair should take place ($\mu-t$, assuming $\mu<Q_0$).
But in this case, the presence of fluid jumps complicates this.
A lower bound for the time at which preventive maintenance should be scheduled is given by $L_c(t)+\mu-L_0(t)$ and this is also the optimal time to do preventive maintenance if no jumps occur in the meantime.

\subsection{Calculation of the expected total discounted cost}
For a given $\mu$, $V_\mu$ is still difficult to compute.
The presence of jumps complicates the calculation of the time between repairs.
Let $T_t(q)$ be the random variable denoting the time until the fluid level is $q$ lower than it was at time $t$, i.e.
$$
T_t(q)=\min\{\tau\geq0|Q(t+\tau)\leq Q(t)-q\}.
$$
Note that, using this definition, $T_0(\mu)$ equals the time until the control limit is reached ($L_0(t)=\mu$) and $T(Q_0)$ equals the time until the machine fails.
\begin{lemma}
$Q_0\leq\mu\Leftrightarrow T_t(Q_0)\leq T_t(\mu)$
\end{lemma}
\begin{proof}
$\Rightarrow$: 
\begin{equation}
\begin{split}
Q_0\leq\mu&\Rightarrow Q(t)-\mu\leq Q(t)-Q_0\\
&\Rightarrow (Q(t+\tau)\leq Q(t)-\mu\Rightarrow Q(t+\tau)\leq Q(t)-Q_0)\\
&\Rightarrow T_t(Q_0)\leq T_t(\mu)\\
\end{split}
\end{equation}
$\Leftarrow$: We will prove that $Q_0>\mu\Rightarrow T_t(Q_0)> T_t(\mu)$:\\
We know that
$$
Q_0>\mu\Rightarrow Q(t)-\mu > Q(t)-Q_0.
$$
Since $Q(t)$ is piecewise continuous and does not decrease at the discontinuities, we know that 
$$
Q(t+T_t(\mu))=Q(t)-\mu>Q(t)-Q_0\Rightarrow T_t(Q_0)> T_t(\mu).
$$
\end{proof}
Also, let $N_t(q)$ be the random variable denoting the number of jumps that occur in the interval $(t,t+T_t(q)]$.
We will now compute its distribution.
For simplicity, we will assume that $r=1$.
The probability that zero jumps occur equals the probability that the exponentially distributed time interval is larger than $q$:
$$
\mathbb{P}(N_t(q)=0)=e^{-\lambda q}.
$$
The probability that exactly one jump occurs equals the probability that exactly one Poisson event happens in the interval $(t,t+q]$ while none happen in $(t+q,t+q+J]$. Resulting in
$$
\mathbb{P}(N_t(q)=1)=\lambda q e^{-\lambda q} e^{-\lambda J}=\lambda q e^{-\lambda (q+J)}.
$$

For each $k\geq0$, by conditioning on the time until the first jump, we get the following recursion
\begin{equation}\label{eq:recursionFormula}
\begin{split}
\mathbb{P}(N_t(q)=k+1)&=\int\limits_0^{q}\lambda e^{-\lambda x}\mathbb{P}(N_t(q-x+J)=k)dx\\
&=\int\limits_0^{q}\lambda e^{-\lambda (q-y)}\mathbb{P}(N_t(y+J)=k)dy,
\end{split}
\end{equation}
since after this first jump, the fluid level equals $q-x+J$ and $k$ jumps should occur. The second equality follows after the substition $y=q-x$.
The solution of this recursion, is given by
$$
\mathbb{P}(N_t(q)=k)=\lambda q\frac{(\lambda (q+kJ))^{k-1}}{k!}e^{-\lambda(q+kJ)}.
$$
Which can be seen by substituting it into \eqref{eq:recursionFormula} and setting $k=1$ to see that it also satisfies the calculated expression for $\mathbb{P}(N_t(q)=1)$.

Now we know the distribution of the amount of jumps that occur in a run where the initial fluid level equals $q$, we can derive the distribution of the time length of a run with initial fluid level $Q_0$ random with density function $f_{Q_0}(q)$.
This distribution $f(x)$, is given by
\begin{equation}
\begin{split}
f(x)&=\lim\limits_{\delta\rightarrow 0}\frac{1}{\delta}\mathbb{P}(x<T_0(Q_0)<x+\delta)\\
&=\sum\limits_{k=0}^{\lfloor\frac{x}{J}\rfloor}\mathbb{P}(N_0(q-kJ)=k)\lim\limits_{\delta\rightarrow 0}\frac{1}{\delta}\mathbb{P}(x-kJ<Q_0<x-kJ+\delta)\\
&=\sum\limits_{k=0}^{\lfloor\frac{x}{J}\rfloor}\mathbb{P}(N_0(q-kJ)=k)f_{Q_0}(x-kJ)\\
\end{split}
\end{equation}

The total discounted cost $V_\mu$ when using control limit $\mu$ can, similarly as with the simple problem, be calculated in the following way
\begin{equation}
V_\mu=\frac{(c+a)\mathbb{P}(Q_0<\mu)\mathbb{E}[e^{-\beta Q_0}|Q_0<\mu]+c\mathbb{P}(Q_0\geq \mu)e^{-\beta\mu}}{1-\mathbb{E}e^{-\beta T(\mu\wedge Q_0)}}.
\end{equation}
The same iteration approaches can also be applied.

\section{Multiple CTMC-states with equal fluid rates}
TODO
\section{Multiple CTMC-states with different fluid rates}
TODO
\end{document}
% Structural properties
\section{Structural properties}
In this section, the effect of changing the parameters to the expected total discounted cost and the control limits are investigated.
These structural properties are mostly similar to the simple fluid problem.
The main difference is that there are multiple control limits for the various CTMC-states and that the control limit in a certain state is also influenced by the costs in other states.

\begin{remark}
	Referring back to the equation for the optimal control limit $\mu_i^*$ for a CTMS-state $s_i$ \eqref{eq:MmfmHazardBoundsShort}: if some change of the parameters would cause an increase in the expected remaining cost for some state $s_j$ that neighbors $s_i$ in the CTMC defined by the generator $\Lambda^D$ (i.e. if $\Lambda^D_{ij}>0$), then $\Lambda^D_{ij}V(j,\mu_i^*,0,\pi^*)$ would increase so that the hazard at which repair is chosen must decrease.
	This results in a lower control limit.
\end{remark}

Furthermore, there are also different fluid rates for different CTMC-states:
\begin{remark}
	An increase in the fluid rate $r_i$ for some state $s_i$ increases the hazard in that state.
	This results in a lower control limit, as one would expect as a higher fluid rate corresponds to the machine deteriorating quicker in that state.
\end{remark}

Again, appendix \ref{AppendixComputationsTable} contains computed values of the optimal control limit and the corresponding expected total discounted cost.
% Heuristic policies
\section{Heuristic policies}\label{section:MmfmHeuristics}
As it is difficult to find an optimal policy that satisfies \eqref{eq:MmfmHazardBoundsShort}, it might be useful to find heuristic policies that minimize the expected total discounted cost reasonably well.

\subsection{Uniform control limit}
If the CTMC-states are similar to each other (i.e. similar fluid rates, transition rates and jump sizes), then we could also just use the same control limit $\mu$ for all the CTMC-states.
This would simplify the expressions 
Finding the policy that minimizes the cost within this class of control limit policies would be relatively easy.
The expected total discounted cost would be
\[
V(x_{NEW},\mu)=\int\limits_0^\mu f(x)D_0(x)dx (c+a+V(x_{NEW},\mu))+\bar{F}(\mu)D_0(\mu)(c+V(x_{NEW},\mu)),
\]
which is easier to minimize numerically than \eqref{eq:MmfmTDC}.
This heuristic would be a crude estimation of the optimal policy if the  CTMC-states are not very similar. 
The heuristic was implemented in Matlab and the resulting policies and total discounted costs were compared with the exact solutions.
The results can be found in appendix \ref{AppendixComputationsTable}.

\subsection{Assuming no jumps before the next failure}
When we compare the equation for the optimal policy of age-based maintenance \eqref{eq:AgeBasedHazardBound} with that of the MMFM problem \eqref{eq:MmfmHazardBounds},
we see that these two differ mostly by the term 
\[
\sum\limits_{j\neq i}\lambda_{ij}V(j,\mu_i^*,J_{ij},\pi^*).
\]
This term is caused by the possibility that a jump would occur around the time the control limit is reached.
If we would simply assume that no jump would occur before the next failure, we could omit this term and the problem would be easier to solve.
This heuristic results in an adjusted equation for the optimal control limits:
\[
r_ih(\mu_i^*)a=\beta(c+V(x_{NEW},\pi^*)).
\]
\begin{remark}
	Note that using this heuristic, all states with the same fluid rate would have the same control limit, regardless of their outgoing edges.
\end{remark}
This heuristic was implemented in Matlab and the resulting policies and total discounted costs were compared with the exact solutions.
The results can be found in appendix \ref{AppendixComputationsTable}.
It turns out that the performance of this heuristic depends a lot on the size and frequencies that jumps would otherwise occur at.
For instance, if transitions would occur frequently and the jump sizes are large, then this heuristic is would be crude and the difference in control limit and cost would be significant.
% Computing the optimal control limits
\section{Computing the optimal control limits}
In this section, a numeric method will be introduced to compute control limits that satisfy \eqref{eq:MmfmHazardBounds}.
The expected total discounted cost will also be computed.
The method is similar to the successive approximation method that has been presented in section \ref{section:AgeBasedOptimalPolicyComputation} for the problem of age-based maintenance.

From \eqref{eq:MmfmHazardBoundsShort}, we know that if a policy $\hat\pi=[\hat\mu_1,...,\hat\mu_N]$ $\hat{\mu}$ satisfies
\[
h(\hat\mu_i)a+\sum\limits_j\Lambda^D_{ij}V(j,\mu_i^*,0,\pi^*)=0,\]
then $\hat{\pi}=\pi^*$.
The total discounted cost would then be given by \eqref{eq:MmfmTDC}.
This suggests the following iteration method:
At the $k+1$'th iteration, we will update the estimates of the optimal control limits $\mu_i^*$ by finding the $\hat{\pi}_i^{(k+1)}=[\hat{\mu}_i^{(k+1)},...,\hat{\mu}_i^{(k+1)}]$ that minimizes
\[
\begin{split}
\hat{V}^{(k+1)}=&\mathbb{E}[D_{i}(Q_0,\hat\pi^{(k+1)},0)](c+a+\hat{V}^{(k)})\\
&+\left[\int\limits_0^\infty \bar{F}(q)\Gamma_0(q,\hat\pi^{(k+1)},0)dq+\sum\limits_i\bar{F}(\hat\mu_i^{(k+1)})D_{i}(\hat\mu_i,\hat\pi^{(k+1)},0)\right](c+\hat{V}^{(k)}),
\end{split}
\]
where $\hat{V}^{(k)}$ is the current estimate of $V(0^+,\pi^*)$.
This could be found by looking for the control limits that satisfy
\begin{equation}
h(\hat\mu_i)a+\sum\limits_j\Lambda^D_{ij}\hat V(j,\mu_i^{(k)},0,\pi^{(k)})=0.
\end{equation}
For this iteration, we also need an initial value of the expected total discounted cost $\hat{V}^{(0)}$.
Although we do not have a proof for the convergence of this iteration method, for the problem parameters that were used, it did converge to solutions similar to those attained via value iteration.

\chapter{Data analysis}
\section{Data description}
\subsection{Data format}
[trace data]
\subsection{Cleaning}
[negative transition times because of the transition from summer to winter time]
\subsection{Visualization}
[plot CDF, PDF and hazard rate of lifetime distribution. Also some plots for the memorylessness of the state transitions.]
\section{Fitting lifetime distributions}
[Weibull, Erlang etc]
\subsection{Phase-type}
[mention that phase-type can fit any positive distribution but that the required number of parameters is huge.]

\chapter{Parameter estimation}
\section{CTMC Estimation}
[Mention the standard way to determine the parameters in a CTMC.]
\section{Estimating fluid rates and jump quantities}
\subsection{Maximum Likelihood?}
[Derive likelihood of parameters and comment on the difficulty to maximize this and that it requires knowledge about the distribution of the initial fluid level.]
\subsection{Justification of minimizing variance}
[Explain that the machines are likely made by the same manufacturer and that this manufacturer probably strives for constant quality (continuity), such that the different machines likely had the same quality initially.]
\subsection{Implementation}
\subsection{Results}
[Mention that this depends on the variance of the distribution of the initial fluid level. If this variance is too large, the parameter estimation has likely minimized the variance too much.]

\chapter{Conclusion}
[Summarize results]
\section{Further research}
[Random jump sizes, better parameter estimation methods, zero or negative fluid rates, nonincreasing hazard rates, ordering the CTMC states depending only on the MMFM]

\chapter{Discussion}
\section{Assumptions}
[List assumptions, their consequences and alternatives]
\section{Robustness}
[Explain what happens to the resulting policy and total discounted cost when one of the problem parameters or MMFM parameters changes a little.]


\bibliography{Sections/bibliography}

\begin{appendices}
	\chapter{List of symbols and notation}\label{AppendixSymbolsAndNotation}
The symbols and notation that are used in chapters \ref{chapter:AgeBased}, \ref{chapter:SimpleFluid} and \ref{chapter:Mmfm} are listed in the table below.
%\begin{table}[H]\label{table:Symbols}
	\begin{tabularx}{\linewidth}{l|L}\label{table:Symbols}
		Symbol        & Meaning         \\
		\hline                       
		$X$			& State space of the MDP. \\
		$X(t)$ & State of the process at time $t$ \\
		$\omega_k$ & Random variable for the evolution of the stochastic process at time $t_k$.\\   
		$x_{BREAK}$ & State for when the asset is broken.           \\
		$x_{NEW}$ & State for when the asset is new.\\   
		$U(x)$& Set of available actions in state $x$.\\
		$a_W$& Action corresponding to doing nothing.\\
		$a_R$& Action corresponding to repairing (performing maintenance). \\
		$g(x,u)$    & Cost of performing action $u$ in state $x\in X$.\\
		$c$& Cost for maintenance.\\
		$a$& Additional cost for corrective maintenance.\\
		$\alpha_\delta$ & Discount over a period $\delta$. In this thesis, we use $\alpha_\delta=e^{-\beta \delta}$ for discount coefficient $\beta>0$.\\       
		$\mu$ & Control limit. $\mu^*$ denotes the optimal control limit.\\
		$\pi$& The policy for the MMFM-problem. $\pi=\{\mu_1,...,\mu_N\}$ where $\mu_i$ denotes the control limit in CTMC-state $s_i$. $\pi^*$ denotes the optimal policy.\\
	    $x\wedge y$   & The minimum of $x$ and $y$.          \\       
        $x\vee y$     & The maximum of $x$ and $y$.\\
        %\pagebreak
        $Q_0$ & Distribution of the initial fitness (lifetime or fluid level) of the asset.\\
        $Q(t)$& Fluid level at time $t$.\\
        $L_0(t)$& Amount of used initial fluid at time $t$.\\
        $L_c(t)$& Amount of fluid inside the buffer at time $t$.\\
        $F$ & CDF of $Q_0$.\\
        $F_{X(t)}$ & CDF of $Q(t)$\\
        $\bar F(x)$& Reliability function corresponding to $F$. Furthermore $\bar F(x;y)=\frac{\bar F(x)}{\bar F(y)}$.\\
        $h$& Hazard rate of $Q_0$.\\                       
		$V_\delta(x,\mu)$ & Expected total discounted cost at state $x$ with time-step $\delta$, using control limit $\mu$. $\delta$ is omitted than for the continuous MDP. For the MMFM we write $V_\delta(x,\pi)$ with policy $\pi$.            \\
		$\lambda_{ij}$& Transition rate from state $s_i$ to $s_j$. Also $\lambda_i=-\sum_{j\neq i}\lambda_{ij}$. $\lambda$ is used for the simple fluid model.\\
		$J_{ij}$& Jump quantity when transitioning from $s_i$ to $s_j$. $J$ for the simple fluid model.\\
		$N$& Number of CTMC-states for the MMFM problem.\\
		$T_t(q)$&Random variable denoting the time until the fluid level has decreased by $q$, i.e. $\min\{\tau-t>0|Q(\tau)=Q(t)-q\}$.\\
		$N_t(q)$&The number of jumps that occur during the time that the fluid decreases by $q$.\\
		$D_{ij}^t(q,\pi,l)$&Expected value of the discount factor over the time period that the process reaches state $s_j$ while the fluid level decreases by $q$ without a control limit from $\pi$ being encountered, given that at time $t$ the process was in $X(t)= (i,l,0)$. For the simple fluid problem, $D(q)$ denotes the discount over the time that the fluid decreases by $q$.\\
		$\Lambda^D$&The generator matrix for $D_{ij}$.\\
	\end{tabularx}
%\end{table}
	\chapter{Proof of properties of optimal age-based control limits}\label{AppendixAgeBasedControlLimit}
Without the assumption of increasing hazard rates, it can also be proven that if an optimal control limit $\mu^*$ exists, $\mu^*$ must satisfy \eqref{eq:AgeBasedHazardBound} and the hazard rate must be increasing at $\mu^*$.
This can be proven using the Bellman equations.
For this, we will briefly return to discretized time:
If the control limit equals $\mu^*$, then one time interval earlier, $c+V_\delta(\delta,\mu^*)\geq V_\delta(\mu^*-\delta,\mu^*)$ holds since else the control limit would be smaller than $\mu^*$.
Using \eqref{eq:gatheredDelta}, we get

\[\begin{split}
&c+V_\delta(0^+,\mu^*)\\
\geq& V_\delta(\mu^*-\delta,\mu^*)\\
=&\delta h(\mu^*)(c+a+ V_\delta(0^+,\mu^*))+(1-\delta\beta-\delta h(\mu^*)) V_\delta(\mu^*,\mu^*)+o(\delta^2).
\end{split}
\]
Since we repair at age $\mu^*$, $V_\delta(\mu^*,\mu^*)=V_\delta(0^+,\mu^*)+c$ and we can write
$$
c+V_\delta(0^+,\mu^*)\geq \delta h(\mu^*)(c+a+ V_\delta(0^+))+(1-\delta\beta-\delta h(\mu^*)) (c+V_\delta(0^+))+o(\delta^2).
$$
Which simplifies to
$$
0\geq ah(\mu^*)-\beta (c+V_\delta(0^+,\mu^*))+o(\delta^2)
$$
and can be rewritten as
$$
h(\mu^*)\leq \beta\frac{c+V_\delta(0^+,\mu^*)}{a} +o(\delta^2)\rightarrow\beta\frac{c+V_\delta(0^+,\mu^*)}{a}.
$$
If instead of looking a decision stage before the control limit, we now look at the decision stage where the control limit $\mu^*$ is reached, the Bellman equations yield
$$
c+V_\delta(0^+,\mu^*)\leq V_\delta(\mu^*-\delta,\mu^*)=\delta h(\mu^*)(c+a+ V_\delta(0^+,\mu^*))+(1-\delta\beta-\delta h(\mu^*)) V_\delta(\mu^*+\delta)+o(\delta^2)
$$
And using the same steps, we get
$$
h(\mu^*)\geq \beta\frac{c+V_\delta(0^+,\mu^*)}{a} +o(\delta^2)\rightarrow\beta\frac{c+V_\delta(0^+,\mu^*)}{a}
$$
such that the result is proven when $\delta$ approaches zero.
From the above, it also follows that the hazard rate is increasing at the control limit.
This can be summarized in the following theorem:

\begin{theorem}
	Whenever the optimal policy is to repair when the age reaches control limit $\mu^*<\infty$, it holds that the hazard rate is increasing at $\mu^*$ and
	\[h(\mu^*)=\beta\frac{c+V(0^+,\mu^*)}{a}.\]
\end{theorem}

\begin{corollary}
	If the hazard rate of the lifetime $Q_0$ of the asset is monotonously decreasing, preventive repair will never be the optimal choice.
\end{corollary}
\chapter{CTMC analysis}
\section{Value iteration}
[Derive Bellman equations, mention implementation]
\section{State clustering}
[Explain the state clustering that was employed, mention that repair states tend to be clustered and explain]
\chapter{Total discounted costs for various problem parameters and policies?}\label{AppendixComputationsTable}
[Tables containing the total discounted cost and control limit that resulted from various heuristic policies and parameters.]
\chapter{Matlab scripts and ProM plugins}
[references to the code]
\end{appendices}
\end{document}